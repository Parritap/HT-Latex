\ChapterImagePrelim[cap:resumen]{Resumen}{./images/fondo.png}\label{cap:resumen}
\mbox{}\\

La presente tesis desarrolla los esfuerzos de expansión de un universo (\ie un contexto de ejecución computacional) de la infraestructura HTCondor actual para el Grupo de Investigación en Redes, Información y Distribución (\GRID) de la universidad del Quindío. Este trabajo responde a la necesidad estratégica del grupo \GRID~de fortalecer sus capacidades en computación distribuida y paralela, con el objetivo de incrementar su competitividad académica y científica. La expansión propuesta facilitaría la colaboración con otras universidades y centros de investigación, permitiendo la consolidación de recursos computacionales compartidos que potencien el desarrollo científico y la investigación de alto impacto, lo que a su vez se alinea con los objetivos misionales de \textit{investigación y extensión} de la universidad del Quindío.
\\\\
Metodológicamente, el proyecto comprende: (\textbf{a}) la exploración y documentación de las necesidades, problemas y oportunidades (\NPO) del \GRID, (\textbf{b}) un estudio de mapeo sistemático (\SMS) para la identificación de la literatura relacionada a los universos HTCondor con el fin de propender por la toma de decisiones informadas, (\textbf{c}) la aplicación de la metodología de Análisis de Decisiones y Resolución (\DAR) del modelo \CMMI. Los resultados señalan a los universos \textbf{parallel} y \textbf{GRID} como los más adecuados. Finalmente, se desarrolla un diseño arquitectónico utilizando el framework de modelado ArchiMate, el cual articula la integración entre los universos HTCondor seleccionados y la infraestructura existente del grupo \GRID. Este diseño se complementa con el desarrollo de prototipos funcionales para cada universo seleccionado, implementados bajo el enfoque de producto mínimo viable (\PMV) para validar la viabilidad técnica y funcional de las soluciones propuestas. Considerando que la solución propuesta debe trascender las necesidades específicas del \GRID~y beneficiar a otros grupos de investigación, se desarrolló adicionalmente una aplicación web que funciona como interfaz de usuario intuitiva, facilitando el acceso de los investigadores a la infraestructura computacional sin requerir conocimientos técnicos especializados.






\ChapterImagePrelim[cap:abstract]{Abstract}{./images/fondo.png}\label{cap:abstract}
\mbox{}\\

This thesis develops the expansion efforts of a universe (\ie a computational execution context) of the current HTCondor infrastructure for the Research Group on Networks, Information and Distribution (\GRID) at Universidad del Quindío. This work responds to the strategic need of the \GRID~group to strengthen its capabilities in distributed and parallel computing, with the objective of increasing its academic and scientific competitiveness. The proposed expansion would facilitate collaboration with other universities and research centers, enabling the consolidation of shared computational resources that enhance scientific development and high-impact research, which in turn aligns with the institutional mission objectives of \textit{research and extension} at Universidad del Quindío.
\\\\
Methodologically, the project comprises: (\textbf{a}) the exploration and documentation of the needs, problems and opportunities (\NPO) of \GRID, (\textbf{b}) a systematic mapping study (\SMS) for the identification of literature related to HTCondor universes in order to promote informed decision-making, (\textbf{c}) the application of the Decision Analysis and Resolution (\DAR) methodology from the \CMMI~model. The results indicate the parallel and \GRID~universes as the most suitable. Finally, an architectural design is developed using the ArchiMate modeling framework, which articulates the integration between the selected HTCondor universes and the existing infrastructure of the \GRID~group. This design is complemented by the development of functional prototypes for each selected universe, implemented under the minimum viable product (\PMV) approach to validate the technical and functional viability of the proposed solutions. Considering that the proposed solution must transcend the specific needs of \GRID~and benefit other research groups, a web application was additionally developed that functions as an intuitive user interface, facilitating researchers' access to the computational infrastructure without requiring specialized technical knowledge.