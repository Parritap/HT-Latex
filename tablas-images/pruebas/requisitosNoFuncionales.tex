\begin{table}[H]
	\centering
	\sffamily\scriptsize
	\setlength{\tabcolsep}{4pt}
	\renewcommand{\arraystretch}{1.3}
	\caption{Requisitos no-funcionales para \textit{Grid App}}
	\label{table:requisitosNoFuncionales}
	\begin{tabular}{|p{0.1\textwidth}|p{0.2\textwidth}|p{0.7\textwidth}|}
		\toprule
		\textbf{ID}                              & \textbf{Título}                               & \textbf{Descripción} \\
		\midrule
		RNF1 & Transparencia de uso & El usuario no debe preocuparse por las configuraciones internas de cada clúster; el Grid Manager abstrae el destino mediante una interfaz más amigable. \\
		\midrule
		RNF2 & Usabilidad & La configuración de los nuevos Universos debe estar documentada y ser accesible mediante manual de despliegue para los usuarios del Grupo GRID. \\
		\midrule
		RNF3 & Disponibilidad & En caso de que un clúster esté inactivo, el Grid Manager debe registrar el error y permitir redirección a otro clúster disponible si el Universo es compatible. \\
		\bottomrule
	\end{tabular}
\end{table}