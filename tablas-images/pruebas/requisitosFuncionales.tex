\begin{table}[H]
	\centering
	\renewcommand{\arraystretch}{1.2} % Espaciado reducido
	\fontsize{9pt}{10pt}\selectfont % Tamaño de fuente 8pt
	\begin{tabular}{|p{1cm}|p{4cm}|p{8cm}|} % Total: 14cm
		\hline
		\textbf{ID}                              & \textbf{Título}                               & \textbf{Descripción} \\
        \hline
		RF1 & Soporte para Universos adicionales & El sistema debe permitir la ejecución de trabajos en al menos dos Universos adicionales a Vanilla (Grid y Parallel) \\
		RF2 & Orquestación de múltiples clústeres & El sistema debe permitir enviar trabajos hacia diferentes clústeres (Parallel, Vanilla) a través del Universo Grid. \\
		RF3 & Selección de destino & El sistema debe permitir que el usuario especifique el clúster de destino. \\
		RF4 & Monitoreo de trabajos & El sistema debe permitir el monitoreo de los trabajos enviados a los clústeres, mostrando su estado al usuario. \\
		RF5 & Ejecución MPI en Parallel & Un clúster debe soportar ejecución de trabajos basados en MPI en múltiples nodos (N nodos configurados). \\
		RF6 & Redirección Grid-Parallel & El Grid Manager debe aceptar trabajos enviados al Universo Grid y, si la naturaleza del trabajo es paralelo, redirigirlos correctamente al Universo Parallel. \\
		RF7 & Redirección Grid-Vanilla & El Grid Manager debe aceptar trabajos enviados al Universo Grid y, si la naturaleza del trabajo es distribuido, redirigirlos correctamente al Universo Vanilla. \\
		RF8 & Registro centralizado de errores & El Grid Manager debe centralizar \textit{logs} de fallos de ejecución provenientes de cada clúster. \\
        \hline
	\end{tabular}
	\caption{Requisitos funcionales para \textit{Grid App}}
	\label{table:requisitosFuncionales}
\end{table}