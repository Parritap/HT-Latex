\begin{table}[H]
	\centering
	\renewcommand{\arraystretch}{1.2} % Espaciado reducido
	\fontsize{9pt}{10pt}\selectfont % Tamaño de fuente 8pt
	\begin{tabular}{|p{2cm}|p{4cm}|p{2.5cm}|p{4.7cm}|} % Total: 14cm
		\hline
		\textbf{ID del escenario de pruebas} & ESC-01 & \textbf{ID de requisito} & RF1, RF2, RF3, RF4, RF7, RF8, RNF1, RNF2, RNF3 \\
        \textbf{Título de la prueba} & Ejecución de trabajos en Universo Grid con repeticiones de iguales parámetros & \textbf{Prerrequisitos} & Grid Manager configurado con al menos un cluster soportando Vanilla \\
        \textbf{Descripción del caso de prueba} & Validar que el sistema soporta ejecución en Universo Grid hacia el Universo Vanilla & \textbf{Prioridad de la prueba} & Alta \\
        \hline
        \multicolumn{4}{|c|}{\textbf{Pasos de ejecución de las pruebas}} \\
        \hline
        \textbf{ID de paso} & \textbf{Acción} & \textbf{Datos de prueba} & \textbf{Resultado esperado} \\
		1 & Ingresar a la página & Link de la aplicación & Página principal de la aplicación mostrada \\
        2 & Subir binario a través del selector de archivos & Binario de ejecución & Binario almacenado en el sistema de envío \\
        3 & Escribir argumentos adicionales (Opcional) & Argumentos adicionales & Argumentos seteados en el submit file dinámico \\
        4 & Seleccionar la naturaleza del trabajo (Distribuido) & Naturalezas permitidas & Universo seteado en el submit file dinámico \\
        5 & Seleccionar la información de distribución (Repeticiones con mismos parámetros) & Distribuciones permitidas & Agregada estructura necesaria al Submit file \\
        6 & Seleccionar cantidad de repeticiones & Repeticiones necesarias & Ciclo de envío en el archivo de shell creado \\
        7 & Seleccionar cluster más favorable & Métricas de clústeres & Recurso remoto del grid en el submit file establecido \\
        8 & Enviar trabajo & Submit file generado & Shell de envío y submit file ejecutado \\
        9 & Ver resultados a medida que van terminando & Resultados de trabajos & Salida de cada ejecución mostrada de forma organizada \\
        \hline
	\end{tabular}
	\caption{Información ESC-01}
	\label{table:esc-01}
\end{table}

\begin{table}[H]
	\centering
	\renewcommand{\arraystretch}{1.2} % Espaciado reducido
	\fontsize{9pt}{10pt}\selectfont % Tamaño de fuente 8pt
	\begin{tabular}{|p{2cm}|p{4cm}|p{2.5cm}|p{4.7cm}|} % Total: 14cm
		\hline
		\textbf{ID del escenario de pruebas} & ESC-02 & \textbf{ID de requisito} & RF1, RF2, RF3, RF4, RF7, RF8, RNF1, RNF2, RNF3 \\
        \textbf{Título de la prueba} & Ejecución de trabajos en Universo Grid con variables parametrizables & \textbf{Prerrequisitos} & Grid Manager configurado con al menos un cluster soportando Vanilla \\
        \textbf{Descripción del caso de prueba} & Validar que el sistema soporta ejecución en Universo Grid hacia el Universo Vanilla & \textbf{Prioridad de la prueba} & Alta \\
        \hline
        \multicolumn{4}{|c|}{\textbf{Pasos de ejecución de las pruebas}} \\
        \hline
        \textbf{ID de paso} & \textbf{Acción} & \textbf{Datos de prueba} & \textbf{Resultado esperado} \\
		1 & Ingresar a la página & Link de la aplicación & Página principal de la aplicación mostrada \\
        2 & Subir binario a través del selector de archivos & Binario de ejecución & Binario almacenado en el sistema de envío \\
        3 & Escribir argumentos adicionales (Opcional) & Argumentos adicionales & Argumentos seteados en el submit file dinámico \\
        4 & Seleccionar la naturaleza del trabajo (Distribuido) & Naturalezas permitidas & Universo seteado en el submit file dinámico \\
        5 & Seleccionar la información de distribución (Repeticiones con mismos parámetros) & Distribuciones permitidas & Agregada estructura necesaria al Submit file \\
        6 & Seleccionar variable parametrizable (De los argumentos adicionales) & Variables parametrizables & Ciclo de envío en el archivo de shell creado \\
        7 & Establecer valor inicial & Valor inicial & Valor inicial del ciclo establecido \\
        8 & Establecer valor final & Valor final & Valor final del ciclo establecido \\
        9 & Establecer incremento & Incremento & Incremento del ciclo establecido \\
        10 & Ver cantidad de repeticiones & Repeticiones necesarias & Cantidad de repeticiones calculadas \\
        11 & Seleccionar clúster más favorable & Métricas de clústeres & Recurso remoto del grid en el submit file establecido \\
        12 & Enviar trabajo & Submit file generado & Shell de envío y submit file ejecutado \\
        13 & Ver resultados a medida que van terminando & Resultados de trabajos & Salida de cada ejecución mostrada de forma organizada \\
        \hline
	\end{tabular}
	\caption{Información ESC-02}
	\label{table:esc-02}
\end{table}

\begin{table}[H]
	\centering
	\renewcommand{\arraystretch}{1.2} % Espaciado reducido
	\fontsize{9pt}{10pt}\selectfont % Tamaño de fuente 9pt
	\begin{tabular}{|p{2cm}|p{4cm}|p{2.5cm}|p{4.7cm}|} % Total: 14cm
		\hline
		\textbf{ID del escenario de pruebas} & ESC-03 & \textbf{ID de requisito} & RF1, RF2, RF3, RF4, RF5, RF6, RF8, RNF1, RNF2, RNF3 \\
		\textbf{Título de la prueba} & Ejecución de trabajos en Universo Parallel & \textbf{Prerrequisitos} & Grid Manager configurado con al menos un clúster soportando Parallel \\
		\textbf{Descripción del caso de prueba} & Validar que el sistema soporta ejecución en Universo Parallel & \textbf{Prioridad de la prueba} & Alta \\
		\hline
		\multicolumn{4}{|c|}{\textbf{Pasos de ejecución de las pruebas}} \\
		\hline
		\textbf{ID de paso} & \textbf{Acción} & \textbf{Datos de prueba} & \textbf{Resultado esperado} \\
		\hline
		1 & Ingresar a la página & Link de la aplicación & Página principal de la aplicación mostrada \\
		2 & Subir binario a través del selector de archivos & Binario de ejecución & Binario almacenado en el sistema de envío \\
		3 & Escribir argumentos adicionales (Opcional) & Argumentos adicionales & Argumentos establecidos en el \textit{submit file} dinámico \\
		4 & Seleccionar la naturaleza del trabajo (Paralelo) & Naturalezas permitidas & Universo establecido en el \textit{submit file} dinámico \\
		5 & Escribir cantidad de máquinas requeridas & Máquinas requeridas & Parámetro ``\textit{machine\_count}'' del \textit{submit file} establecido \\
		6 & Escribir cantidad de núcleos solicitados por máquina & Núcleos requeridos & Parámetro ``\textit{required\_cpus}'' del \textit{submit file} establecido \\
		7 & Seleccionar clúster más favorable & Métricas de clústeres & Recurso remoto establecido \\
		8 & Enviar trabajo & \textit{Submit file} generado & \textit{Script} de envío y \textit{submit file} ejecutado \\
		9 & Ver resultados a medida que van terminando & Resultados de trabajos & Salida de cada ejecución mostrada de forma organizada \\
		\hline
	\end{tabular}
	\caption{Información ESC-03}
	\label{table:esc-03}
\end{table}