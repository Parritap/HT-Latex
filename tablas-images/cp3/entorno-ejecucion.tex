\begin{table}[H]
\centering
\scriptsize
\setlength{\tabcolsep}{3pt}
\renewcommand{\arraystretch}{1.1}
\begin{tabularx}{\textwidth}{|p{0.2\textwidth}|X|}
\hline
\textbf{Tecnología} & \textbf{Ambiente de Ejecución} \\
\hline
Docker & Sistemas Linux, Windows y macOS. Entornos de desarrollo, pruebas y producción, incluyendo la nube pública (\AWS, Google Cloud, Azure). \\
\hline
Podman & Sistemas Linux y macOS, con soporte experimental en Windows. Usado en entornos de desarrollo y producción sin necesidad de un daemon. \\
\hline
Udocker & Sistemas Linux, en entornos de computación de alto rendimiento (\HPC) y servidores compartidos, permitiendo ejecutar contenedores sin privilegios de root. \\
\hline
Wasm (WebAssembly) & Navegadores web (Chrome, Firefox, Safari, Edge). Ejecución en aplicaciones web y entornos de alto rendimiento sin dependencia del sistema operativo subyacente. \\
\hline
LXC & Sistemas Linux. Utilizado para crear contenedores ligeros que actúan como máquinas virtuales, con aplicaciones aisladas en servidores y plataformas de nube privada. \\
\hline
Containerd & Sistemas Linux y Windows. Usado en plataformas de orquestación como Kubernetes, y en infraestructuras de contenedores a gran escala. \\
\hline
LXD & Sistemas Linux. Virtualización ligera de contenedores como máquinas virtuales completas, ideal para servidores y plataformas de virtualización en la nube privada. \\
\hline
Rkt & Sistemas Linux. Anteriormente usado en entornos de orquestación de contenedores y en infraestructuras de nube privada. (Descontinuado actualmente). \\
\hline
Singularity & Sistemas Linux, especialmente en entornos de computación científica y \HPC\ . Portabilidad de aplicaciones científicas sin necesidad de privilegios de root. \\
\hline
runC & Sistemas Linux. Runtime ligero de contenedores compatible con los estándares de la Open Container Initiative (\OCI), utilizado por plataformas como Docker y Kubernetes. \\
\hline
CRI-O & Sistemas Linux. Integrado con Kubernetes para la gestión eficiente de contenedores en plataformas de orquestación en la nube y servidores locales. \\
\hline
Hyper-V containers & Sistemas Windows (Windows Server y Windows 10 con Hyper-V habilitado). Usado para contenedores con aislamiento mediante micro-VMs, ideal para entornos híbridos. \\
\hline
OpenVZ & Sistemas Linux. Virtualización a nivel de contenedor en proveedores de hosting para ofrecer entornos aislados y eficientes. \\
\hline
Linux VServer & Sistemas Linux. Contenedores que funcionan como servidores virtuales, usados en entornos de hosting y administración de servidores de alta disponibilidad. \\
\hline
Google gVisor & Plataformas de nube, especialmente Google Cloud. Capa adicional de seguridad para contenedores en entornos multitenant. \\
\hline
Kata Containers & Sistemas Linux. Virtualización ligera con aislamiento similar a máquinas virtuales, utilizado en plataformas de contenedores en la nube y servidores donde se requiere seguridad. \\
\hline
Firecracker & Plataformas de computación en la nube, como \AWS\ . Optimizado para micro-VMs ultra ligeras y de alto rendimiento en entornos serverless. \\
\hline
Sarus & Sistemas Linux. Entornos de computación de alto rendimiento (\HPC) y clústeres de supercomputación, proporcionando portabilidad y eficiencia sin privilegios de root. \\
\hline
\end{tabularx}
\caption{Entornos de ejecución de cada VBC}
\label{tab:entornos-ejecucion-vbc}
\end{table}