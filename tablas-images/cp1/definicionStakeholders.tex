\begin{table}[H]
	\centering
	\fontsize{7}{7}\selectfont % Cambia el primer número (9) al tamaño deseado, el segundo es el interlineado
	\setlength{\tabcolsep}{3pt} % reduce el espacio horizontal entre columnas
	\renewcommand{\arraystretch}{1.4} % espacio entre filas
	\begin{tabularx}{\textwidth}{%
		>{\raggedright\arraybackslash}p{1.7cm}  % Columna 1: Interesado
		>{\raggedright\arraybackslash}p{1.5cm}    % Columna 2: Rol
		>{\raggedright\arraybackslash}p{2.3cm}  % Columna 3: Relación
		>{\raggedright\arraybackslash}p{1.4cm}  % Columna 4: Impacto
		>{\raggedright\arraybackslash}p{2.0cm}  % Columna 5: Poder de influencia
		>{\raggedright\arraybackslash}p{2.5cm}    % Columna 6: Interés
		>{\raggedright\arraybackslash}p{2.0cm}} % Columna 7: Compromiso
		\toprule
		\textbf{Interesado}                              & \textbf{Rol}                               & \textbf{Relación}                                                                                                                                      & \textbf{Impacto} & \textbf{Poder de influencia}                                                                                            & \textbf{Interés}                                                                                                                          & \textbf{Compromiso}                                                                                \\
		\midrule
		Docentes de Ingeniería de Sistemas               & Usuarios clave                             & Utilizarán los entregables del proyecto en actividades de enseñanza e investigación                                                                    & Medio-Alto       & Medio, pueden proponer mejoras pero no tienen poder de decisión sobre la implementación                                 & Medio-Alto, requieren soliuciones tecnológicas para integrarlas en procesos académicos e investigativos                                   & Medio, condicionado a la utilidad práctica y aplicabilidad de la solución                          \\
		\midrule
		Estudiantes de Ingeniería de Sistemas            & Beneficiarios potenciales                  & Podrían utilizar la infraestructura en proyectos académicos y acceder a información sobre el estudio                                                   & Bajo             & Bajo, carecen de poder de decisión aunque su adopción validará la efectividad de la solución                            & Bajo, dado que la computación distribuida es un área especializada con uso poco frecuente en el contexto académico general                & Bajo                                                                                               \\
		\midrule
		Grupo de Investigación GRID                      & Beneficiario principal                     & Proporciona la infraestructura base, evalúa la solución propuesta y mide su impacto                                                                    & Alto             & Alto, tiene autoridad para decidir la adopción e implementación de la tecnología                                        & Alto, busca optimizar sus servicios y fortalecer su posicionamiento en el ámbito de la investigación y la colaboración interuniversitaria & Alto, la solución potenciará directamente sus capacidades de infraestructura                       \\
		\midrule
		Grupos de Investigación Locales y Externos       & Beneficiarios principales                  & Identificarían oportunidades significativas para potenciar sus líneas de investigación                                                                 & Medio            & Medio-Bajo, pueden influir mediante solicitudes específicas de funcionalidades o mejoras                                & Alto, la solución podría reducir significativamente los tiempos de procesamiento y obtención de resultados en sus proyectos               & Medio-Alto, conformarían uno de los segmentos principales de usuarios finales                      \\
		\midrule
		Programa de Ingeniería de Sistemas y Computación & Facilitador institucional                  & Puede proveer respaldo institucional, recursos y normativas que faciliten la adopción                                                                  & Alto             & Alto, posee autoridad para aprobar asignación de recursos e invitar a otros programas académicos a utilizar la solución & Medio, su interés se centra en aspectos institucionales y estratégicos más que operativos                                                 & Bajo-Medio, especialmente si la solución no impacta directamente sus procesos de gestión académica \\
		\midrule
		Comunidad HTCondor e investigadores en HTC y HPC & Proveedores y consumidores de conocimiento & Suministran documentación técnica esencial para la implementación y se benefician de la documentación generada a partir de la experiencia del proyecto & Alto             & Bajo, no participan en decisiones sobre el proyecto                                                                     & Bajo-Medio, la implementación exitosa fortalecería la adopción de HTCondor y ampliaría el impacto de la tecnología                        & Bajo, sujeto a la alineación de resultados con sus objetivos comunitarios                          \\
		\bottomrule
	\end{tabularx}
	\caption{Análisis de stakeholders}\label{tab:stakeholders}
\end{table}
