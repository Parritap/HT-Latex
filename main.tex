\documentclass[12pt]{report}
\usepackage[utf8]{inputenc}
\usepackage[spanish]{babel}
\usepackage[T1]{fontenc}
\usepackage{geometry}
\geometry{a4paper, margin=3cm}

\usepackage{setspace}
\onehalfspacing\usepackage{graphicx}
\usepackage{float}
\usepackage{caption}
\usepackage{subcaption}
\usepackage{amsmath, amssymb}
\usepackage{array}
\usepackage{tabularx}
\usepackage{multirow}
\usepackage{booktabs}
\usepackage[most]{tcolorbox}
\usepackage{verbatim}
\usepackage{placeins}

% ####################################
\usepackage[absolute,overlay]{textpos}
\usepackage{babel}
\usepackage{lmodern}
\usepackage{iflang}
\usepackage{xspace}
\usepackage[table,xcdraw]{xcolor}
\usepackage{hyperref}
\usepackage{tabularx}
\usepackage{array}
\usepackage{titlesec}
\usepackage{eso-pic}

% ####################################

% Citas
\usepackage{natbib}

% Hyperlinks: siempre debe ir antes de acronym
\usepackage{hyperref}

% Encabezados y pies de página personalizados
\usepackage{fancyhdr}

% Acrónimos (Después de hyperref)
\usepackage[printonlyused]{acronym}

\usepackage[absolute,overlay]{textpos}
\usepackage{babel}
\usepackage{lmodern}
\usepackage{iflang}
\usepackage{xspace}
\usepackage[table,xcdraw]{xcolor}
\usepackage{hyperref}
\usepackage{tabularx}
\usepackage{array}
\usepackage{titlesec}
\usepackage{eso-pic}

% Configuración del tamaño de secciones y subsecciones
\titleformat{\section}
  {\large\bfseries}{\thesection}{1em}{}
\titleformat{\subsection}
  {\normalsize\bfseries}{\thesubsection}{1em}{}
\titleformat{\subsubsection}
  {\normalsize\bfseries}{\thesubsubsection}{1em}{}

\usepackage[utf8]{inputenc}
\usepackage[spanish]{babel}
\usepackage[T1]{fontenc}
\usepackage{geometry}
\geometry{a4paper, margin=3cm}

\usepackage{setspace}
\onehalfspacing\usepackage{graphicx}
\usepackage{float}
\usepackage{caption}
\usepackage{subcaption}
\usepackage{amsmath, amssymb}
\usepackage{array}
\usepackage{tabularx}
\usepackage{multirow}
\usepackage{booktabs}
\usepackage[most]{tcolorbox}
\usepackage{verbatim}
\usepackage{placeins}

% ####################################
\usepackage[absolute,overlay]{textpos}
\usepackage{babel}
\usepackage{lmodern}
\usepackage{iflang}
\usepackage{xspace}
\usepackage[table,xcdraw]{xcolor}
\usepackage{hyperref}
\usepackage{tabularx}
\usepackage{array}
\usepackage{titlesec}
\usepackage{eso-pic}

% ####################################

% Citas
\usepackage{natbib}

% Hyperlinks: siempre debe ir antes de acronym
\usepackage{hyperref}

% Encabezados y pies de página personalizados
\usepackage{fancyhdr}

% Acrónimos (Después de hyperref)
\usepackage[printonlyused]{acronym}

\newcommand{\mytype}{Tesis de pregrado}
\newcommand{\myname}{José Alejandro Arias Pinzón}
\newcommand{\matricle}{1002652342}
\newcommand{\mytitlea}{Solución arquitectónica de}
\newcommand{\mytitleb}{tecnologías de virtualización basada en }
\newcommand{\mytitlec}{contenedores para el grupo de investigación}
\newcommand{\mytitled}{ en redes, información y distribución}
\newcommand{\reviewerone}{Dra. Diana Marcela Rivera Valencia}
\newcommand{\advisor}{Ph.D. Luis Eduardo Sepúlveda Rodríguez}
\newcommand{\timeend}{Colombia, 2025}
\newcommand{\submissiontime}{16.07.2025}
\newcommand{\mynameb}{Anubis Haxard Correa Urbano}
\newcommand{\matricleb}{1004871385} 

% Configuración de encabezados y pies de página
\pagestyle{fancy}
\fancyhf{} % Limpia todos los encabezados y pies
\renewcommand{\headrulewidth}{0pt} % Sin línea en el encabezado
\renewcommand{\footrulewidth}{0pt} % Sin línea en el pie

% Configuración del banner inferior desde la página 1
% Ajustes de posición en puntos (1pt ≈ 1 pixel a 72 DPI)
\newcommand{\bannerHeight}{68.16pt} % 2.4cm = 68.16pt
\newcommand{\bannerXOffset}{0pt} % Desplazamiento horizontal del banner
\newcommand{\bannerYOffset}{0pt} % Desplazamiento vertical del banner
\newcommand{\pageNumXOffset}{28.35pt} % 1cm = 28.35pt - posición X del número
\newcommand{\pageNumYOffset}{35.05pt} % 3cm = 85.05pt - posición Y del número

\fancyfoot[C]{%
  \begin{tikzpicture}[remember picture, overlay]
    % Banner que ocupa todo el ancho de la página
    \node[anchor=south, inner sep=0pt] at ([xshift=\bannerXOffset, yshift=\bannerYOffset]current page.south) {%
      \includegraphics[width=\paperwidth, height=\bannerHeight]{./images/banner-inferior.png}%
    };
    % Número de página sobre la imagen en la esquina inferior izquierda
    \node[anchor=south west, text=black, font=\bfseries] 
      at ([xshift=\pageNumXOffset, yshift=\pageNumYOffset]current page.south west) {\thepage};
  \end{tikzpicture}%
}


% Marcador para la sección de Siglas
\newcommand{\siglaref}{\hyperref[cap:siglas]{\nameref{cap:siglas}}}

% Ahora definimos cada sigla con enlace hacia la lista
\newcommand{\API}{\hyperref[cap:siglas]{API}\xspace}
\newcommand{\CMMI}{\hyperref[cap:siglas]{CMMI}\xspace}
\newcommand{\CPU}{\hyperref[cap:siglas]{CPU}\xspace}
\newcommand{\DAR}{\hyperref[cap:siglas]{DAR}\xspace}
\newcommand{\GRID}{\hyperref[cap:siglas]{GRID}\xspace}
\newcommand{\IT}{\hyperref[cap:siglas]{IT}\xspace}
\newcommand{\SMS}{\hyperref[cap:siglas]{SMS}\xspace}
\newcommand{\VBC}{\hyperref[cap:siglas]{VBC}\xspace}
\newcommand{\TI}{\hyperref[cap:siglas]{TI}\xspace}
\newcommand{\PMV}{\hyperref[cap:siglas]{PMV}\xspace}
\newcommand{\PMBOK}{\hyperref[cap:siglas]{PMBOK}\xspace}
\newcommand{\ISO}{\hyperref[cap:siglas]{ISO}\xspace}
\newcommand{\CNCF}{\hyperref[cap:siglas]{CNCF}\xspace}
\newcommand{\TOGAF}{\hyperref[cap:siglas]{TOGAF}\xspace}
\newcommand{\IEC}{\hyperref[cap:siglas]{IEC}\xspace}
\newcommand{\CS}{\hyperref[cap:siglas]{CS}\xspace}
\newcommand{\CLI}{\hyperref[cap:siglas]{CLI}\xspace}
\newcommand{\UI}{\hyperref[cap:siglas]{UI}\xspace}
\newcommand{\HPC}{\hyperref[cap:siglas]{HPC}\xspace}
\newcommand{\AWS}{\hyperref[cap:siglas]{AWS}\xspace}
\newcommand{\OCI}{\hyperref[cap:siglas]{OCI}\xspace}
\newcommand{\VM}{\hyperref[cap:siglas]{VM}\xspace}
\newcommand{\OS}{\hyperref[cap:siglas]{OS}\xspace}
\newcommand{\PMI}{\hyperref[cap:siglas]{PMI}\xspace}
\newcommand{\CPD}{\hyperref[cap:siglas]{CPD}\xspace}

% Comando para capítulos SIN numeración de página (páginas especiales)
\newcommand{\ChapterImageEmpty}[3][]{%
  \cleardoublepage
  \thispagestyle{fancy}%
  \refstepcounter{chapter}%
  \AddToShipoutPictureBG*{%
    \begin{tikzpicture}[remember picture,overlay]
      % Imagen de fondo al ancho de la página
      \node[inner sep=0pt, anchor=north west] at (current page.north west)
        {\includegraphics[width=\paperwidth]{#3}};
      % Caja blanca que se adapta al ancho del texto
      \node[
        anchor=north west,
        xshift=1.5cm, yshift=-1.0cm,
        text=black,
        fill=white, fill opacity=0.8,
        rounded corners=6pt,
        inner xsep=10pt, inner ysep=6pt
      ] at (current page.north west)
        {\Large\bfseries \thechapter\quad #2\strut};
    \end{tikzpicture}
  }%
}

% Comando para secciones preliminares (sin numeración pero en índice)
\newcommand{\ChapterImagePrelim}[3][]{%
  \cleardoublepage%
  \phantomsection%
  \addcontentsline{toc}{chapter}{#2}%
  \AddToShipoutPictureBG*{%
    \begin{tikzpicture}[remember picture,overlay]
      % Imagen de fondo al ancho de la página
      \node[inner sep=0pt, anchor=north west] at (current page.north west)
        {\includegraphics[width=\paperwidth]{#3}};
      % Caja blanca que se adapta al ancho del texto
      \node[
        anchor=north west,
        xshift=1.5cm, yshift=-1.0cm,
        text=black,
        fill=white, fill opacity=0.8,
        rounded corners=6pt,
        inner xsep=10pt, inner ysep=6pt
      ] at (current page.north west)
        {\Large\bfseries #2\strut};
    \end{tikzpicture}
  }%
}

% Comando para capítulos CON numeración de página (capítulos normales)
\newcommand{\ChapterImageStar}[3][]{%
  \cleardoublepage%
  \refstepcounter{chapter}%
  \addcontentsline{toc}{chapter}{\protect\numberline{\thechapter}#2}%
  \AddToShipoutPictureBG*{%
    \begin{tikzpicture}[remember picture,overlay]
      % Imagen de fondo al ancho de la página
      \node[inner sep=0pt, anchor=north west] at (current page.north west)
        {\includegraphics[width=\paperwidth]{#3}};
      % Caja blanca que se adapta al ancho del texto
      \node[
        anchor=north west,
        xshift=1.5cm, yshift=-1.0cm,
        text=black,
        fill=white, fill opacity=0.8,
        rounded corners=6pt,
        inner xsep=10pt, inner ysep=6pt
      ] at (current page.north west)
        {\Large\bfseries \thechapter\quad #2\strut};
    \end{tikzpicture}
  }%
}

\newcommand\BackgroundPic{%
    \put(-25,-2){%
        \parbox[b][\paperheight]{\paperwidth}{%
            \vfill
            \centering
            \includegraphics[scale=1,height=\paperheight]{./images/fondo-anteportada.png}%
            \vfill
}}}

\newcommand\PortadaPic{%
    \put(0,0){%
        \parbox[b][\paperheight]{\paperwidth}{%
            \vfill
            \centering
            \includegraphics[width=\paperwidth,height=\paperheight]{images/portada.png}%
            \vfill
}}}

\begin{document}


\AddToShipoutPicture*{\BackgroundPic}
\begin{titlepage}
	\thispagestyle{empty} % quita encabezados/pies de página

	\vspace*{-0.5cm}
	\noindent
	\begin{minipage}{\textwidth}
		\centering
		{\LARGE\bfseries \mytitlea}\\[1em]
		{\LARGE\bfseries \mytitleb}\\[1em]
		{\LARGE\bfseries \mytitlec}\\[1em]
		{\LARGE\bfseries \mytitled}
	\end{minipage}

	\begin{tikzpicture}[remember picture,overlay]
		\node[anchor=west] at ([xshift=8cm,yshift=-19cm]current page.north west)
		{\normalsize \myname };
	\end{tikzpicture}
	\begin{tikzpicture}[remember picture,overlay]
		\node[anchor=west] at ([xshift=9cm,yshift=-19.5cm]current page.north west)
		{\small Cc: \matricle};
	\end{tikzpicture}

	\begin{tikzpicture}[remember picture,overlay]
		\node[anchor=west] at ([xshift=8cm,yshift=-21cm]current page.north west)
		{\normalsize \mynameb};
	\end{tikzpicture}

	\begin{tikzpicture}[remember picture,overlay]
		\node[anchor=west] at ([xshift=9cm,yshift=-21.5cm]current page.north west)
		{\small Cc: \matricleb};
	\end{tikzpicture}
	\clearpage % fuerza nueva página después
\end{titlepage}
\ClearShipoutPicture 

\AddToShipoutPicture*{\PortadaPic} % agrega la imagen de fondo
\begin{titlepage}
    \thispagestyle{empty} % sin encabezado ni pie

    % Usamos TikZ para colocar texto en posiciones absolutas
    \begin{tikzpicture}[remember picture,overlay]

        % Logo en (x,y) desde esquina superior izquierda

        % Título en coordenadas absolutas
        

        % Tipo de trabajo
        \node[anchor=west] at ([xshift=14cm,yshift=-6.5cm]current page.north west)
            {\Large \mytype};


        % Revisor y asesor
        \node[anchor=west] at ([xshift=8cm,yshift=-21.5cm]current page.north west)
            {\begin{tabular}{rl}
                Revisor: & \reviewerone \\
                Asesor: & \advisor
              \end{tabular}};

        % Fecha
        \node[anchor=west] at ([xshift=11cm,yshift=-24cm]current page.north west)
            {\normalsize \timeend};

    \end{tikzpicture}

\end{titlepage}
\ClearShipoutPicture%

% Página en blanco antes de la dedicatoria
\cleardoublepage%
\thispagestyle{empty}
\vspace*{\fill}
\begin{center}
\small\textit{Página en blanco intencionalmente}
\end{center}
\vspace*{\fill}
\newpage

\cleardoublepage%
\phantomsection%
\thispagestyle{empty}

\vspace*{\fill}

\begin{center}
{\Large\bfseries Dedicatoria}
\end{center}

\vspace{2cm}

\begin{center}
\textit{
    A mi madre, por sus esfuerzos y sacrificios para brindarme una educación, además de enseñarme el valor del trabajo duro y la perseverancia. \\
    A mi padre, por enseñarme casi todo lo que sé y por ser un ejemplo de esfuerzo. \\
    A mi hermano, por su apoyo constante y por ser una fuente de inspiración.\\
    A mi esposa, Natalie Arias, por su amor, paciencia y comprensión, y por apoyarme de todas las formas posibles en esta etapa de mi vida.\\
}

\vspace{2cm}

\hfill \textit{--- José Alejandro Arias Pinzón}

\vspace{3cm}

\textit{
A mi madre, por su paciencia, comprensión, por creer siempre en mí y motivarme
a alcanzar mis metas.
A mi hermano, por apoyarme y motivarme a ser mejor cada día.
A mis compañeros, por ser un apoyo incondicional en todo este proceso y
permitirme crecer como persona.
}

\vspace{2cm}

\hfill \textit{--- Anubis Haxard Correa Urbano}
\end{center}


\input{capitulos/agradecimientos.tex}

% Página en blanco después de agradecimientos
\cleardoublepage%
\thispagestyle{empty}
\vspace*{\fill}
\begin{center}
\small\textit{Página en blanco intencionalmente}
\end{center}
\vspace*{\fill}
\newpage

% redefinimos temporalmente el nombre del TOC
\renewcommand{\contentsname}{} % sin título automático

% portada bonita sin numeración de página
\ChapterImagePrelim{Índice general}{./images/fondo.png}
\vspace*{-3cm}
% ahora el índice sin entrada ni salto extra
\begingroup
  \renewcommand{\addcontentsline}[3]{} % no añade entrada
  \let\clearpage\relax
  \let\cleardoublepage\relax
  \tableofcontents
\endgroup
\ChapterImagePrelim[cap:resumen]{Resumen}{./images/fondo.png}\label{cap:resumen}
\mbox{}\\
La presente tesis desarrolla la especificación de una solución arquitectónica basada en tecnologías de virtualización por contenedores (\VBC) para el Grupo de Investigación en Redes, Información y Distribución (\GRID) de la Universidad del Quindío. El estudio surge como respuesta a la necesidad de ampliar el portafolio de servicios tecnológicos mediante instancias computacionales más ligeras que complementen la infraestructura existente basada en máquinas virtuales. 
Metodológicamente, la investigación comprende: (a) un estudio de mapeo sistemático (\SMS) para identificar tecnologías de \VBC; (b) un \textit{benchmarking} técnico de consumo de \CPU, memoria, tiempo de arranque, \textit{throughput} y latencia de E/S; y (c) la aplicación de la metodología de Análisis de Decisiones y Resolución (\DAR) del modelo \CMMI. Los resultados señalan a Containerd como la tecnología más adecuada, mientras que K3S se identificó como el motor de orquestación más viable. 
Finalmente, se propone un diseño arquitectónico modelado en Archimate que articula la infraestructura existente con la capa de virtualización (Containerd + K3S) y la capa de aplicación. La solución busca ofrecer un servicio escalable y mantenible alineado con las necesidades académicas e investigativas del \GRID. El trabajo concluye con la implementación de un producto mínimo viable (\PMV), que valida la pertinencia de la propuesta y establece un referente metodológico para decisiones tecnológicas en contextos académicos e institucionales. 

\ChapterImagePrelim[cap:abstract]{Abstract}{./images/fondo.png}\label{cap:abstract}
\mbox{}\\
This thesis develops the specification of an architectural solution based on container virtualization technologies (CVT) for the Research Group on Networks, Information, and Distribution (\GRID) at the University of Quindío. The study arises in response to the need to expand the technological service portfolio through lighter computational instances that complement the existing infrastructure, which is based on virtual machines.
Methodologically, the research includes: (a) a systematic mapping study (\SMS) to identify CVT technologies; (b) a technical benchmarking of CPU consumption, memory, boot time, throughput, and I/O latency; and (c) the application of the Decision Analysis and Resolution (\DAR) methodology from the CMMI model. The results indicate Containerd as the most suitable technology, while K3S was identified as the most viable orchestration engine.
Finally, an architectural design modeled in Archimate is proposed, which integrates the existing infrastructure with the virtualization layer (Containerd + K3S) and the application layer. The solution aims to offer a scalable and maintainable service aligned with \GRID’s academic and research needs. The work concludes with the implementation of a minimum viable product (MVP), which validates the relevance of the proposal and establishes a methodological reference for technological decisions in academic and institutional contexts.

% portada bonita
\ChapterImagePrelim{Índice de figuras}{./images/fondo.png}

\mbox{}\\
% pegamos la lista sin salto extra
\begingroup
  \renewcommand{\addcontentsline}[3]{} % no añade entrada
  \let\clearpage\relax
  \let\cleardoublepage\relax
  \makeatletter
    \@starttoc{lof}% genera la lista de figuras directamente
  \makeatother
\endgroup
% portada bonita
\ChapterImagePrelim{Índice de tablas}{./images/fondo.png}

\mbox{}\\
% pegamos la lista sin salto extra
\begingroup
  \renewcommand{\addcontentsline}[3]{} % no añade entrada
  \let\clearpage\relax
  \let\cleardoublepage\relax
  \makeatletter
    \@starttoc{lot}% genera la lista de tablas directamente
  \makeatother
\endgroup

% Reiniciar numeración de capítulos para el contenido principal
\setcounter{chapter}{0}
\renewcommand{\thechapter}{\arabic{chapter}}
\ChapterImagePrelim[cap:introduccion]{Introducción}{./images/fondo.png}\label{cap:introduccion}
\mbox{}\\
% La computación en la nube \textit{(Cloud Computing)} es uno de los conceptos con más crecimiento en la industria de la tecnología~\citep{Jayaweera2024}. Las organizaciones han identificado en esta forma de computación una manera de aprovisionamiento de recursos informáticos rápida y según la demanda. Entre sus principales beneficios se incluyen la flexibilidad, la escalabilidad y la eficiencia en costos~\citep{Ahmadi2024}. La adopción de estos recursos ha transformado el desarrollo de soluciones tecnológicas, lo cual ha posibilitado que la planificación, el análisis, el diseño, el desarrollo, las pruebas y el mantenimiento se realicen completamente en la nube. Esto ha dado origen a aplicaciones nativas de este entorno, conocidas como \textit{cloud native apps}.\\
%Las \textit{cloud native apps} permiten a las organizaciones implementar soluciones complejas con un rendimiento mejorado, distribuyendo sus cargas de trabajo en múltiples entornos de nube y optimizando el retorno de inversión~\citep{Alonso2023}. Con el aumento en el uso de estas aplicaciones nativas, ha aumentado también la demanda por por infraestructura que las soporte. Para mitigar los costos que implica el crecimiento de estos equipos físicos se busca la posibilidad de consolidar cada vez más los recursos de \TI. La virtualización es útil debido a que permite una consolidación de recursos según las necesidades organizacionales. Anteriormente el despliegue de aplicaciones se realizaba directamente sobre el sistema de la máquina física; actualmente, la gran mayoría se ejecuta sobre sistemas virtualizados~\citep{Jain2016}. Las máquinas virtuales, o de sistema completo, han sido hasta ahora el estándar de facto para la segmentación de infraestructura de \TI;\@ sin embargo, la virtualización ligera, también conocida como Virtualización Basada en Contenedores (\textbf{\VBC})\footnote{Las siglas utilizadas en este documento se explican en el capítulo~\ref{cap:siglas}.}, se ha ido posicionando como una alternativa moderna a las máquinas virtuales.\\ \\
%En este contexto, desde la aparición de Docker en 2013, la virtualización ligera ha transformado el desarrollo de software, fortaleciendo prácticas como DevOps, donde la escalabilidad y la replicabilidad son fundamentales~\citep{Docker2021}. Docker ha experimentado un notable crecimiento en su adopción, debido a su capacidad para ejecutar aplicaciones en el mismo entorno en el que fueron construidas, sin importar el lugar donde se implementen. El crecimiento de Docker se ve evidenciado en el uso de \textit{Docker images} por parte de los desarrolladores. En 2023 se registraron 130 mil millones de descargas, cifra que aumentó a 242 mil millones en 2024~\citep{Docker2024}. A partir del auge de Docker, surgieron nuevas tecnologías de contenerización, la aparición de estas puede percibirse inicialmente como una ventaja para organizaciones, desarrolladores y demás actores de \TI\;\@sin embargo, la proliferación de estas herramientas puede representar un reto al momento de elegir la idónea en una arquitectura de solución.\\
%Debido al creciente interés en la \VBC\ surge la pregunta ¿Cuál es la tecnología idonea según el contexto del \GRID? y ¿Qué elementos tecnológicos son necesarios para una posible implementación de esta tecnología?  Este trabajo aborda la situación ya expuesta, cuyo objetivo principal es proponer una arquitectura de solución basada en contenedores para el Grupo de Investigación en Redes, Información y Distribución (\GRID) de la Universidad del Quindío. Inicialmente, se realiza una valoración de necesidades de la organización cliente, destacando sus objetivos misionales enfocados en el apoyo a la docencia, la investigación y la extensión. El desafío consiste en el aprovechamiento de la infraestructura actual del \GRID\ aportando al cumplimiento de sus objetivos misionales. Lo anterior, haciendo uso de los aportes del presente trabajo. Posteriormente, se profundiza en una revisión del estado del arte mediante un estudio de mapeo sistemático \textit{(Systematic Mapping Study --- \SMS\ )}, con el objetivo de comprender las tecnologías de \VBC\ y los dominios de \TI\ en los que se desarrollan. Paso seguido, se realiza un análisis \DAR\ \textit{(Decision Analysis and Resolution)} basado en el modelo de \CMMI\, el cual permite definir la tecnología de contenedores adecuada
% en la implementación de una solución. A partir de este análisis, se desarrolla la arquitectura de solución con base en las necesidades del grupo de investigación.

\ChapterImagePrelim[cap:glosario]{Glosario}{./images/fondo.png}\label{cap:glosario}
\mbox{}\\
En este apartado se encuentran términos clave y conceptos relevantes utilizados a lo largo de este proyecto.

\section*{B}
\begin{description}
  \item[Benchmarking:] Mide el rendimiento o el grado de éxito alcanzado en comparación con otras empresas para una actividad, flujo de valor u otros factores de interés determinados. Estas medidas se convierten en la base para el análisis y el rediseño \citep{PeterWootton2024}.
\end{description}

\section*{C}
\begin{description}
  \item[Cloud Computing:] La computación en la nube es un modelo que permite el acceso a la red, ubicuo, práctico y bajo demanda, a un conjunto compartido de recursos informáticos configurables que pueden aprovisionarse y liberarse rápidamente con un mínimo esfuerzo de gestión o interacción con el proveedor de servicios \citep{Mell2011}.
\end{description}

\section*{E}
\begin{description}
  \item[Escalabilidad:] Capacidad de un sistema de ajustar automáticamente sus recursos informáticos en función de la carga de trabajo, mediante la asignación de servidores, memoria o red de manera dinámica \citep{TARI2024100650}.
\end{description}

\section*{H}
\begin{description}
  \item[Hypervisor:] Es responsable de crear, administrar y programar máquinas virtuales, que representan máquinas reales para los sistemas operativos que se ejecutan en ellas \citep{Cinque2024}.
\end{description}

\section*{P}
\begin{description}
  \item[Private Cloud:] Una nube privada virtual se refiere a una nube privada alojada en un entorno de nube pública o compartida. Permite la conexión entre la infraestructura heredada y los servicios en la nube mediante una conexión de red virtual segura \citep{Collins2016}.
  
  \item[Producto mínimo viable:] El producto mínimo viable es aquella versión de un nuevo producto que permite a un equipo recopilar la máxima cantidad de aprendizaje validado sobre los clientes con el menor esfuerzo \citep{Ries2020}.
\end{description}

\section*{V}
\begin{description}
  \item[Virtualización:] Proceso de crear versiones virtuales de recursos físicos, como servidores, sistemas operativos o redes, con el fin de mejorar la eficiencia y flexibilidad en la gestión de la infraestructura \citep{Meena2021}.
\end{description} 
\ChapterImagePrelim[cap:siglas]{Siglas y Abreviaturas}{./images/fondo.png}\label{cap:siglas}
\mbox{}\\
A continuación, se presentan las siglas y abreviaturas utilizadas en este documento, junto con su significado completo para facilitar la comprensión.
\begin{description}
	\item[HTC] Computación de Alta Productividad (\textit{High Throughput Computing})
	\item[CMMI] \textit{Capability Maturity Model Integration}
	\item[DAR] \textit{Decision Analysis and Resolution}
	\item[GRID] Grupo de Investigación en Redes, Información y Distribución
	\item[HPC] Computación de Alto Rendimiento (\textit{High Performance Computing})
	\item[PMBOK] Project Management Body of Knowledge
	\item[PMI] Instituto de Gestión de Proyectos (\textit{Project Management Institute})
	\item[SMS] Systematic Mapping Study
	\item[TI] Tecnologías de la Información
\end{description}

\ChapterImageStar[cap:objetivos]{Objetivos}{./images/fondo.png}\label{cap:objetivos}
\mbox{}\\
En este capítulo se establece un conjunto de objetivos que orientan el desarrollo del trabajo, articulando el propósito general con metas específicas que permiten su cumplimiento de manera sistemática. Estos objetivos se centran en la definición, análisis y validación de una arquitectura basada en tecnologías de virtualización con contenedores (\VBC), con el fin de responder a las necesidades y oportunidades del Grupo de Investigación en Redes, Información y Distribución (\GRID). 

\section{Objetivo general}\label{cap:objetivoGeneral}

Especificar una arquitectura de tecnologías de virtualización basadas en contenedores (\VBC), evaluando sus características a través de un benchmarking, seleccionando la que mejor se adapte a la necesidad, problema y oportunidad del \GRID\ (Grupo de Investigación en Redes, Información y Distribución), haciendo un análisis \DAR\ e implementando un producto mínimo viable (\PMV).

\section{Objetivos específicos}\label{cap:objetivosEspecificos}
\begin{itemize}
    \item Reconocer necesidades del \GRID\ con relación a las tecnologías de virtualización basadas en contenedores.
    \item Identificar las tecnologías de virtualización basadas en contenedores.
    \item Caracterizar tecnologías de virtualización basadas en contenedores.
    \item Seleccionar un conjunto de tecnologías de contenedores para realizar pruebas de concepto.
    \item Diseñar una especificación arquitectónica para las herramientas seleccionadas.
    \item Implementar el prototipo funcional.
    \item Validar casos con relación a la necesidad del cliente.
\end{itemize}
\ChapterImageStar[cap:justificacion]{Justificación}{./images/fondo.png}\label{cap:justificacion}
\mbox{}\\
Actualmente, el Grupo de Investigación en Redes, Información y Distribución 
(\GRID) enfrenta diversas necesidades y oportunidades en relación con los servicios 
tecnológicos que ofrece a la Universidad del Quindío, en apoyo a sus objetivos 
misionales de docencia, investigación y extensión. En este contexto, el \GRID\ 
orienta sus esfuerzos hacia la identificación de tecnologías emergentes que fortalezcan 
su capacidad de ofrecer servicios tecnológicos avanzados, tanto para beneficio propio 
como para la comunidad académica de su área de influencia. En este marco, la virtualización basada en procesos se proyecta como una alternativa 
estratégica para la gestión de recursos y servicios de tecnología informática 
(\TI). Si bien el \GRID\ dispone de una infraestructura sustentada en máquinas virtuales, 
gestionadas mediante un hipervisor tipo I, persiste la necesidad de contar con instancias 
computacionales más livianas que permitan ampliar la oferta de servicios hacia la comunidad 
académica, particularmente a los estudiantes del programa de Ingeniería de Sistemas y 
Computación de la Universidad del Quindío.\\
Como señalan diversos autores~(\citep{Chingo2021} y~\citep{DOGANI2023120}), las tecnologías de virtualización han experimentado un 
crecimiento acelerado en los últimos años, consolidándose como la base fundamental de 
infraestructuras modernas, entre ellas el cloud computing~\citep{Sepulveda-Rodriguez2022}. 
En este sentido, la virtualización basada en contenedores (\VBC) se presenta como una opción 
al requerir recursos computacionales más ligeros para su operación~\citep{Xavier2013}. 
Su incorporación, en complemento a las máquinas virtuales ya desplegadas en el \GRID, 
posibilita el diseño de un portafolio de servicios de \TI\ más diversificado, escalable, 
flexible y de fácil mantenimiento. De este modo, la adopción de tecnologías de virtualización basadas en contenedores no solo 
responde a las demandas del grupo de investigación, sino que también abre la 
puerta a nuevas oportunidades de innovación y transferencia de conocimiento dentro del 
contexto académico. Con ello, el \GRID\ podría consolidarse como un referente institucional 
en el aprovechamiento de tecnologías de virtualización, incrementando su capacidad para 
atender demandas crecientes de infraestructura computacional y apoyando con mayor solidez 
la misión universitaria.
\input{capitulos/metodologia.tex}
\ChapterImageStar[cap:marcoConceptual]{Marco Conceptual}{./images/fondo.png}\label{cap:marcoConceptual}
\mbox{}\\
Los conceptos a continuación no solo delimitan el ámbito de estudio, sino que también proporcionan las bases terminológicas y estructurales necesarias para la evaluación, comparación e implementación de las tecnologías consideradas. 

\section{Virtualización}
La virtualización constituye el punto de partida de este marco. Se entiende como una tecnología que permite la creación de representaciones virtuales de recursos físicos de computación, como servidores, almacenamiento o redes, facilitando la ejecución de múltiples entornos aislados sobre una misma infraestructura física~\citep{AmazonWebServicesInc2023}. Tradicionalmente, la virtualización se ha implementado mediante hipervisores, software que posibilita la creación y gestión de máquinas virtuales (\VM), las cuales emulan hardware completo y ejecutan sistemas operativos independientes (\OS)~\citep{KLEIDERMACHER201225}. Si bien esta aproximación ofrece un alto grado de aislamiento, conlleva una sobrecarga significativa en términos de consumo de recursos, debido a la duplicación de sistemas operativos y la necesidad de reservar recursos de forma estática para cada instancia~\citep{bauman2015survey}. Actualmente, en el grupo de investigación se aprovisionan servicios mediante máquinas virtuales.

\section{Virtualización basada en contenedores}
Como alternativa al modelo anterior, emerge la \VBC. A diferencia de las máquinas virtuales, los contenedores no virtualizan el hardware subyacente, sino que comparten el kernel del sistema operativo anfitrión~\citep{eder2016hypervisor}. Esto se logra mediante mecanismos de aislamiento a nivel de sistema operativo, como \textit{namespaces} y \textit{cgroups} en Linux, que permiten aislar procesos, sistemas de archivos, redes y recursos computacionales para cada contenedor~\citep{jain2020linux}. El resultado son instancias ligeras, con tiempos de arranque reducidos y un consumo de recursos inferior en el uso de \CPU\;, memoria y almacenamiento, en comparación con las \VM\ tradicionales~\citep{6903537}. Se puede identificar como un complemento a la virtualización completa ya disponible en el grupo.

\section{Registro de contenedores}
Un concepto central en el ecosistema de contenedores es el de la imagen de contenedor. Una imagen es un paquete inmutable que incluye todo lo necesario para ejecutar una aplicación: código, bibliotecas, herramientas del sistema y configuraciones~\citep{straesser2023empirical}. Las imágenes se construyen en capas a partir de un archivo de instrucciones (Dockerfile, por ejemplo), lo que favorece la reutilización y el control de versiones~\citep{dahlmanns2023secrets}. Estas imágenes se almacenan en un \textit{registry} (como Docker Hub), desde donde pueden ser descargadas y ejecutadas en cualquier entorno compatible~\citep{anwar2018improving}. El uso de registros populares permite al grupo de investigación aprovisionar software estándar de manera rápida y sencilla. Agregando a esto, se pueden crear imágenes personalizadas que encapsulen aplicaciones específicas del grupo o sus dependencias, facilitando su despliegue y distribución en repositorios ya existentes o privados.

\section{Orquestación de contenedores}
La orquestación de contenedores representa otro pilar conceptual indispensable. Gestionar manualmente unos pocos contenedores es factible, pero a escala, se requiere de herramientas automatizadas para desplegar, escalar, monitorizar y mantener la salud de cientos o miles de contenedores~\citep{al2019container}. Aquí, Kubernetes se erige como el estándar \textit{de facto}~\citep{zhou2021container}. Es una plataforma \textit{open-source} que automatiza la gestión de aplicaciones contenerizadas, proporcionando mecanismos para el descubrimiento de servicios, balanceo de carga, auto-escalamiento y despliegues automatizados~\citep{carrion2022kubernetes}. Para el caso del \GRID, Kubernetes no solo facilita la administración de contenedores, sino que también permite afinar el uso de recursos en un entorno compartido, propendiendo por la alta disponibilidad y resiliencia frente a fallos.

\section{Runtime de contenedores}
Kubernetes no interactúa directamente con los contenedores, sino a través de un \textit{runtime} de contenedores que cumple con la \textit{Open Container Initiative} (\OCI), un estándar industrial que especifica formatos y rutinas para contenedores~\citep{girma2018evaluation}. Es en este nivel donde se sitúan las tecnologías de \VBC\ evaluadas en este trabajo, como Docker, Podman, containerd y LXC. \@Cada una ofrece una implementación distinta para crear y ejecutar contenedores \OCI. Docker, por ejemplo, popularizó el uso de contenedores al ofrecer una herramienta todo-en-uno que incluye un \textit{daemon}, un \CLI\ y herramientas de construcción de imágenes~\citep{Buchanan2020}. Containerd, por su parte, es un \textit{runtime} más minimalista y enfocado, que se diseñó para ser embebido en sistemas más grandes como Kubernetes, manejando el ciclo de vida de los contenedores a bajo nivel~\citep{protogeros2024cargosync}. Containerd es el \textit{runtime} que finalmente se seleccionó para la solución del \GRID\, esta decisión se fundamentó en un análisis comparativo que consideró aspectos técnicos, de licencia, soporte y comunidad, avalado por la metodología \DAR\ del modelo \CMMI\;\@.

\section{Cloud native}
La transición hacia infraestructuras definidas por \textit{software} y aplicaciones \textit{cloud-native} está intrínsecamente ligada a estos conceptos. Una aplicación \textit{cloud-native} está diseñada específicamente para aprovechar las ventajas de la nube, como la escalabilidad y la alta disponibilidad, y suele estar compuesta por microservicios empaquetados en contenedores y orquestados mediante plataformas como Kubernetes~\citep{gannon2017cloud}. Este paradigma arquitectónico permite una entrega continua y una resiliencia superior frente a modelos monolíticos tradicionales~\citep{oyeniran2024comprehensive}. Para el \GRID, la adopción de estas tecnologías representa una evolución natural de su infraestructura. El grupo ya cuenta con una base sólida de virtualización tradicional mediante el hipervisor XCP-ng. La incorporación de \VBC\ no busca reemplazar esta infraestructura, sino complementarla, ofreciendo un portafolio de servicios más diversificado. Los contenedores pueden coexistir con las máquinas virtuales, permitiendo asignar la tecnología más adecuada según la carga de trabajo. Además, en la implementación de un \textit{cluster} de Kubernetes se puede utilizar \VM\ como nodos, que sirve de base para el posterior despliegue de servicios.

\section{GitOps}
GitOps, por otro lado, es un paradigma operativo que utiliza repositorios Git como fuente de verdad para la definición de la infraestructura y las aplicaciones, automatizando los despliegues para que el estado del sistema siempre refleje el estado declarado en el repositorio~\citep{kormanik2023exploring}. En conjunto, estos conceptos delinean un panorama tecnológico moderno y robusto sobre el cual se diseña la solución arquitectónica para el \GRID. Proporcionan el vocabulario técnico y las bases estructurales necesarias para entender la evaluación, selección e integración de las tecnologías de \VBC\ en el contexto específico del grupo de investigación, con sus particulares restricciones de recursos y sus objetivos misionales de docencia, investigación y extensión.



\ChapterImageStar[cap:Marco-Teorico]{Marco Teórico}{./images/fondo.png}\label{cap:marcoTeorico}
\mbox{}\\
En el contexto de la gestión de proyectos tecnológicos y el desarrollo de software, los marcos de referencia resultan fundamentales para afrontar los desafíos actuales con metodologías claras y estructuras probadas. Estos marcos permiten ubicar el proyecto dentro de corrientes de pensamiento aceptadas y, al mismo tiempo, ofrecen herramientas prácticas para su aplicación en contextos reales. Para el Grupo de Investigación en Redes, Información y Distribución (GRID), su utilización busca mejorar la organización, la calidad y la pertinencia de las soluciones tecnológicas, particularmente en el diseño de arquitecturas basadas en contenedores.

\section{PMBOK}
Uno de los referentes más reconocidos en la gestión de proyectos es el \PMBOK\ (\textit{Project Management Body of Knowledge}), establecido por el Project Management Institute. Este estándar reúne un conjunto amplio de buenas prácticas aplicables a la mayoría de los proyectos, organizando el trabajo en áreas clave como el alcance, tiempo, costos, calidad, riesgos y recursos\citep{project2017guia}. La utilización del \PMBOK\ no solo mejora la gestión y control de los proyectos, sino que también permite alinearlos con los objetivos estratégicos de la organización, propendiendo la entrega de valor y la reducción de riesgos durante su ejecución\citep{Monday2022}.

\section{ISO 9000}
Complementariamente, la norma \ISO\ 9000 aporta una perspectiva centrada en la calidad, promoviendo la estandarización de procesos y la mejora continua\citep{ISO9001}. Esta serie de normas internacionales busca garantizar que las organizaciones respondan de manera consistente a las expectativas de los clientes, mediante la implementación de principios que abarcan desde el liderazgo hasta la gestión de la información y el conocimiento. Aplicar este marco no solo mejora la operación, sino que también fortalece la confianza del cliente y asegura la calidad en los productos y servicios ofrecidos\citep{Gray2022}. Así, se establece una conexión directa entre la gestión de proyectos y los sistemas de calidad, lo que resulta especialmente útil cuando se busca garantizar la sostenibilidad de los resultados.

\section{Modelo por capas}
Para abordar la complejidad técnica de los sistemas desarrollados, se recurre al modelo por capas, una arquitectura que permite dividir el sistema en distintos niveles con funciones específicas y autónomas. Esta forma de organización contribuye a una mayor claridad y modularidad, permitiendo que los componentes de una capa puedan ser modificados sin afectar el resto del sistema\citep{Spray2023}. De este modo, se facilita el mantenimiento, la escalabilidad y la gestión de cambios, cualidades esenciales en el desarrollo de software moderno. La interoperabilidad también se ve fortalecida, dado que esta arquitectura permite una integración más fluida entre distintos módulos y servicios.

\section{CNCF}
En ese mismo sentido, la Cloud Native Computing Foundation (\CNCF) introduce un enfoque moderno para el desarrollo de aplicaciones, orientado a tecnologías nativas de la nube. Este marco promueve prácticas como el uso de contenedores, microservicios y la automatización continua, con el objetivo de construir soluciones más eficientes, escalables y resilientes\citep{CNCF2023}. La \CNCF\ también proporciona herramientas que buscan la portabilidad y la interoperabilidad entre diferentes entornos de nube, lo que permite a las organizaciones adaptarse con mayor agilidad a un entorno cambiante y competitivo. Su enfoque abierto e interoperable lo convierte en un aliado clave para iniciativas que busquen aprovechar al máximo las capacidades de la nube.

\section{Design thinking}
Junto a estas herramientas técnicas y de gestión, el Design Thinking aporta una perspectiva centrada en las personas, enfocándose en comprender profundamente las necesidades del usuario para proponer soluciones innovadoras\citep{CombellesC.LucenaP.2020}. Esta metodología fomenta la empatía, la experimentación y la colaboración interdisciplinaria, promoviendo la creación de productos y servicios que se ajusten con mayor precisión a las demandas reales del contexto. Su inclusión en proyectos tecnológicos no solo impulsa la innovación, sino que también fortalece la toma de decisiones ágiles y adaptativas, favoreciendo entornos flexibles en constante evolución.

\section{TOGAF}
Por su parte, \TOGAF\ (The Open Group Architecture Framework) complementa este conjunto de marcos al enfocarse en la alineación entre la estrategia del negocio y los procesos de tecnología de la información. Mediante su enfoque estructurado por fases —que abarca desde la planificación hasta la implementación y el monitoreo— \TOGAF\ permite gestionar arquitecturas empresariales de forma coherente y flexible. Su aplicación ayuda en el uso recursos, integración de sistemas y toma de decisiones estratégicas con una visión holística de la organización\citep{Mumtaza2025}.

\section{ISO/IEC 25010}
Finalmente, la norma \ISO/\IEC\ 25010 establece un modelo integral para la evaluación de la calidad del software, considerando atributos como la funcionalidad, usabilidad, seguridad, mantenibilidad y portabilidad\citep{ISO25010}. Este marco teórico es fundamental para asegurar que los sistemas desarrollados cumplan con los requisitos tanto del negocio como del usuario final, proporcionando un enfoque riguroso que permite identificar áreas de mejora en las distintas etapas del ciclo de vida del software. Su adopción permite fortalecer la confianza en los productos desarrollados y garantizar su robustez en contextos dinámicos.\\

Todos estos marcos, aunque distintos en su enfoque, se complementan entre sí y permiten establecer una base para la formulación y ejecución de proyectos tecnológicos. Su integración permite abordar los retos desde múltiples dimensiones —estratégica, técnica, organizacional y humana—, ayudando al diseño de soluciones innovadoras y sostenibles.
\ChapterImageStar[cap:desarrollo-metodologico]{Desarrollo metodológico}{./images/fondo.png}\label{cap:desarrolloMetodologico}
\mbox{}\\
A continuación, se describe el procedimiento metodológico seguido para alcanzar los objetivos planteados en esta investigación. La metodología se estructuró en fases sucesivas y complementarias que permiten pasar de la caracterización del contexto institucional y tecnológico, hacia la selección, diseño, implementación y validación de una arquitectura basada en tecnologías de virtualización por contenedores (\VBC).

\section{Caracterización del GRID}
El Grupo de Investigación en Redes, Información y Distribución (\GRID) de la Universidad del Quindío desarrolla actividades en los ejes misionales de la institución: educación, investigación y extensión. En el marco de esta investigación, se caracterizó el \GRID\ con el propósito de identificar sus capacidades actuales, necesidades y oportunidades relacionadas con la adopción de tecnologías de virtualización. Este diagnóstico inicial permitió contextualizar la pertinencia de las \VBC\ como una alternativa tecnológica para fortalecer los servicios académicos y de investigación, especialmente en beneficio de los estudiantes de Ingeniería de Sistemas y Computación.

\section{Revisión de la literatura}
Con el fin de fundamentar la investigación, se realizó un mapeo sistemático de estudios (\SMS). Este consistió en la búsqueda, filtrado, selección y análisis de literatura académica, artículos técnicos y reportes de caso relacionados con las \VBC. El objetivo fue obtener una visión global y estructurada sobre las tecnologías disponibles, sus tendencias de adopción y las principales dimensiones de análisis empleadas en la comunidad científica y profesional.

\section{Identificación y caracterización de tecnologías VBC}
A partir de los resultados del \SMS, se seleccionaron las tecnologías de \VBC\ con mayor relevancia e impacto en la literatura y la práctica. Para cada una de ellas se realizó una caracterización técnica, evaluando aspectos como arquitectura interna, facilidad de integración, integración con la nube y comunidad de soporte. Esta fase permitió construir un marco comparativo preliminar que orienta la elección de herramientas candidatas para el \GRID.

\section{Benchmarking de tecnologías VBC}
Posteriormente, se diseñó y ejecutó un proceso de \textit{benchmarking} enfocado en medir y contrastar el desempeño de un conjunto de tecnologías seleccionadas bajo condiciones controladas. Los criterios de evaluación incluyeron consumo de \CPU, uso de memoria, throughput de red y operaciones de entrada/salida (I/O). Los resultados permitieron establecer métricas que evidencian fortalezas y limitaciones de cada tecnología, facilitando la selección informada de la alternativa adecuada para el contexto institucional.

\section{Análisis de Decisión y Resolución (DAR)}
Con base en los resultados del \textit{benchmarking}, se aplicó un análisis de Decisión y Resolución (\DAR). Este método permitió ponderar los beneficios, riesgos y oportunidades asociados con la adopción de las \VBC\ en el \GRID. El \DAR\ integró tanto los criterios técnicos como los organizacionales, priorizando aquellos que propenden por la sostenibilidad de la solución a mediano y largo plazo.

\section{Diseño de la solución arquitectónica}
En esta fase se elaboró la propuesta de arquitectura tecnológica que articula la infraestructura existente en el \GRID\ con las capacidades de la tecnología seleccionada. El diseño incluyó la definición de componentes, interacciones, flujos de información y políticas de gestión, buscando escalabilidad, resiliencia y facilidad de administración de la solución.

\section{Implementación de la solución}
Con el diseño arquitectónico como guía, se procedió a implementar un producto mínimo viable (\PMV) que materializa la adopción de la tecnología seleccionada. La implementación se llevó a cabo en el entorno del \GRID, integrando las configuraciones necesarias y desplegando servicios básicos que permiten evaluar la funcionalidad del sistema en condiciones reales.

\section{Validación de la solución}
Finalmente, se realizó la validación del \PMV\ mediante pruebas de desempeño, disponibilidad y escalabilidad, contrastando los resultados con los requerimientos definidos en la fase de caracterización del \GRID. Adicionalmente, se consideraron percepciones de los usuarios del grupo de investigación como insumo para verificar la pertinencia y aplicabilidad de la solución propuesta.

\ChapterImageStar[cap:caracterizacionGRID]{Caracterización del GRID}{./images/fondo.png}\label{cap:caracterizacionGRID}
\mbox{}\\
\noindent
El Grupo de Investigación en Redes, Información y Distribución (\GRID) de la Universidad del Quindío se enmarca en los objetivos misionales de la institución: educación, investigación y extensión. Uno del los intereses particulares del \GRID es ofrecer servicios tecnológicos avanzados a la comunidad académica, con énfasis en los estudiantes de Ingeniería de Sistemas y Computación, quienes encuentran en este grupo un espacio de formación e innovación en temas de infraestructura, ingeniería de software y tecnologías emergentes.\\
La caracterización del \GRID\ resulta esencial para comprender su estructura, capacidades y necesidades en relación con la expansión de un nuevo universo HTCondor. A continuación, se presenta un análisis de los diferentes aspectos que definen el contexto institucional y tecnológico del grupo.

\section{Análisis de stakeholders del GRID}
\noindent
Con el fin de identificar los actores internos y externos que influyen en el desarrollo de las actividades del grupo, se realizó un análisis de \textit{stakeholders}. Este ejercicio permitió reconocer los diferentes intereses, roles y niveles de influencia que cada actor tiene en relación con la expansión de un nuevo universo HTCondor. Los principales \textit{stakeholders} identificados incluyen: investigadores del grupo, estudiantes de pregrado y posgrado, docentes de la Facultad de Ingeniería, y en un nivel más amplio, la comunidad académica de la Universidad del Quindío.
\\
\noindent
La tabla~\ref{tab:stakeholders} presenta el análisis de los principales interesados en la expansión de los universos HTCondor dentro del contexto institucional. Se identifican actores internos y externos, especificando su rol, el tipo de relación con el proyecto, el nivel de impacto esperado, así como su poder de influencia, interés y compromiso frente a la iniciativa.

El análisis de interesados revela una estructura compleja de actores con distintos niveles de influencia y compromiso frente al proyecto. El Grupo de Investigación GRID emerge como el interesado crítico, concentrando simultáneamente el mayor impacto, poder de influencia, interés y compromiso. Esta posición central le otorga un rol determinante como beneficiario principal y tomador de decisiones, lo que subraya la necesidad de mantener una comunicación estrecha y continua con este grupo para asegurar la alineación del proyecto con sus expectativas estratégicas y operativas.

Los docentes de Ingeniería de Sistemas y los investigadores locales y externos constituyen el segundo nivel de importancia, caracterizándose por un alto interés en la solución y un potencial significativo como usuarios finales. Aunque su poder de decisión es limitado, su adopción efectiva de la tecnología será fundamental para validar el éxito del proyecto. Este segmento requiere especial atención en términos de usabilidad, documentación y capacitación, dado que su compromiso está condicionado a la utilidad práctica y los beneficios tangibles que puedan obtener. La diferencia principal entre ambos grupos radica en que los investigadores externos representarían el segmento de usuarios más activo e intensivo de la infraestructura.

El Programa de Ingeniería de Sistemas y Computación desempeña un rol estratégico como facilitador institucional con alto poder de influencia, particularmente en la provisión de recursos y la posibilidad de escalar la solución hacia otros programas académicos. Sin embargo, su compromiso relativamente bajo sugiere que será necesario demostrar claramente el valor estratégico e institucional del proyecto para asegurar su respaldo sostenido. Por otro lado, los estudiantes de pregrado presentan bajo impacto, poder e interés debido a la naturaleza especializada de la computación distribuida en el contexto académico de pregrado, lo que los posiciona como beneficiarios secundarios que validarán la solución más por uso ocasional que por dependencia operativa.

Finalmente, la comunidad HTCondor y los investigadores en HTC y HPC representan un interesado externo con una relación simbiótica particular: aunque no tienen poder de decisión sobre el proyecto y su compromiso es bajo, su rol como proveedores de conocimiento técnico es de alto impacto. La documentación y experiencias generadas por el proyecto podrían contribuir al fortalecimiento del ecosistema HTCondor, creando un valor agregado que trasciende los objetivos locales del proyecto. Esta relación sugiere la conveniencia de establecer canales de comunicación con esta comunidad para compartir hallazgos y mejores prácticas.

\begin{table}[H]
\centering
\scriptsize % reduce el tamaño de la letra en 2 puntos aprox.
\renewcommand{\arraystretch}{1.3} % espacio entre filas
\begin{tabularx}{\textwidth}{%
    >{\raggedright\arraybackslash}X
    >{\raggedright\arraybackslash}X
    >{\raggedright\arraybackslash}X
    >{\raggedright\arraybackslash}X
    >{\raggedright\arraybackslash}X
    >{\raggedright\arraybackslash}X
    >{\raggedright\arraybackslash}X}
\toprule
\textbf{Interesado} & \textbf{Rol} & \textbf{Relación} & \textbf{Impacto} & \textbf{Poder de influencia} & \textbf{Interés} & \textbf{Compromiso} \\
\midrule
Grupo de Investigación GRID & Beneficiario principal & Provee infraestructura, evalúa la solución y su impacto & Alto & Alto, decide la adopción de la tecnología & Alto, busca mejorar sus servicios & Alto, ya que su infraestructura será potenciada \\
\midrule
Docentes de Ingeniería de Sistemas & Usuarios clave & Harán uso de los entregables para proyectos y enseñanza & Medio-Alto & Medio, pueden sugerir mejoras pero no decidir implementación & Medio-Alto, esperan decisiones sobre tecnología para enseñanza e investigación & Medio, dependerá de la utilidad de la solución \\
\midrule
Estudiantes de Ingeniería de Sistemas & Usuarios finales & Usarán los servicios en sus cursos y proyectos. Podrán informarse sobre el estudio. & Medio & Bajo, no tienen poder de decisión, pero su uso validará la solución & Alto, necesitan un entorno estable y eficiente & Medio-Alto, dependiendo de la accesibilidad y usabilidad \\
\midrule
Programa de ingeniería de sistemas y computación & Facilitador & Puede apoyar con recursos y normativas para la adopción & Alto & Alto, puede aprobar recursos & Medio, su interés es institucional y estratégico & Bajo-Medio, si la solución no afecta directamente a su gestión \\
\midrule
Proveedores de tecnología (Docker, Kubernetes, etc.) & Proveedores de herramientas & Proveen la tecnología de virtualización a utilizar & Bajo & Bajo, la decisión de uso recae en el GRID y la universidad & Bajo aunque buscan ampliar su base de usuarios & Bajo, su involucramiento es indirecto \\
\midrule
Investigadores y otros grupos de investigación & Potenciales beneficiarios & Pueden usar los resultados en búsqueda de mejoras para sus proyectos & Medio & Medio, pueden influir con solicitudes de mejora & Medio, dependiendo de su relación con GRID & Bajo, solo si ven beneficios concretos \\
\midrule
Sector empresarial & Potencial inversor o socio & Podría apoyar la solución si ve ventajas en la adopción de TVBC & Bajo-Medio & Bajo, no decide en la universidad, pero puede ofrecer incentivos & Bajo-Medio, si la tecnología ofrece valor comercial & Bajo, depende de la alineación con sus intereses \\
\bottomrule
\end{tabularx}
\caption{Análisis de stakeholders}\label{tab:stakeholders}
\end{table}

\section{Priorización de stakeholders}
Una vez realizada la identificación de los \textit{stakeholders}, se emprendió un proceso de priorización para determinar cuáles poseen mayor impacto y poder de decisión en el proyecto. Esta clasificación resulta crucial para establecer estrategias de comunicación, gestión de expectativas y participación activa en la definición de requerimientos. De esta manera, se busca que los actores más influyentes en la toma de decisiones y en la adopción tecnológica sean atendidos de forma prioritaria, aumentando las probabilidades de éxito en la implementación.

\begin{figure}[H]
    \centering
    \includegraphics[width=\textwidth] {tablas-images/cp1/priorizacionStakeholders.png}
    \caption{Priorización de stakeholders del proyecto}\label{fig:tabla-priorizacion-stakeholders}
\end{figure}

\section{Integrantes y áreas de trabajo del GRID}
El \GRID\ está conformado por un equipo multidisciplinario de investigadores y profesionales que, desde sus diferentes áreas de experticia, contribuyen al avance en campos como computación de alto rendimiento, \textit{big data}, inteligencia artificial, redes de computadoras y desarrollo de software. A continuación, se presenta una lista de los integrantes actuales del grupo, junto con sus respectivas líneas de trabajo y áreas de especialización:
\begin{itemize}
	\item \href{https://scienti.minciencias.gov.co/cvlac/visualizador/generarCurriculoCv.do?cod_rh=0000210897}{\underline{{\textbf{Christian Andrés Candela Uribe}}}}: Microservicios, desarrollo de software, minería de datos, infraestructura TI.\@
	\item \href{https://scienti.minciencias.gov.co/cvlac/visualizador/generarCurriculoCv.do?cod_rh=0001383939}{\underline{{\textbf{Luis Eduardo Sepúlveda Rodríguez}}}}: Infraestructura de TI, HPC, computación paralela.
	\item \href{https://scienti.minciencias.gov.co/cvlac/visualizador/generarCurriculoCv.do?cod_rh=0001638854}{\underline{{\textbf{Carlos Andrés Flórez Villarraga}}}}: Programación y algoritmia, inteligencia artificial.
	\item \href{https://scienti.minciencias.gov.co/cvlac/visualizador/generarCurriculoCv.do?cod_rh=0001343801}{\underline{{\textbf{Carlos Eduardo Gómez Montoya}}}}: Redes, ingeniería de software, cloud computing.
	\item \href{https://scienti.minciencias.gov.co/cvlac/visualizador/generarCurriculoCv.do?cod_rh=0001398775}{\underline{{\textbf{Sergio Augusto Cardona Torres}}}}: Big data y análisis de datos, ingeniería de software, sistemas adaptativos, informática educativa.
	\item \href{https://scienti.minciencias.gov.co/cvlac/visualizador/generarCurriculoCv.do?cod_rh=0000193550}{\underline{{\textbf{Sonia Jaramillo Valbuena}}}}: Big data, electroquímica, inteligencia artificial.
	\item \href{https://scienti.minciencias.gov.co/cvlac/visualizador/generarCurriculoCv.do?cod_rh=0000283495}{\underline{{\textbf{Julián Esteban Gutiérrez Posada}}}}: Microservicios, desarrollo de software, minería de datos, infraestructura TI, HPC, computación paralela, redes, ingeniería de software.
\end{itemize}

La diversidad de líneas de trabajo de los integrantes fortalece la capacidad del grupo para abordar proyectos de carácter transversal y multidisciplinario, lo cual resulta particularmente relevante para el diseño e implementación de soluciones arquitectónicas soportadas en tecnologías de virtualización.

\section{Misión del GRID}
La misión del GRID es heredada de la Universidad del Quindío. A continuación se presenta la misión del GRID:\@

\begin{quote}
	\textit{La Universidad del Quindío contribuye a la transformación de la sociedad, mediante la formación integral desde el ser, el saber y el hacer, de líderes reflexivos y gestores del cambio; con estándares de calidad, a través de una oferta de formación en diferentes metodologías, que responda a una sociedad basada en el conocimiento; una investigación pertinente, que aporte a la solución de las problemáticas del desarrollo e integrada con la extensión y proyección social; educando en tiempos del posconflicto y de la consolidación de la paz, apoyada en una gestión creativa y con estándares de calidad.}
\end{quote}

A partir de esta misión, se identifican los siguientes pilares fundamentales:

\begin{itemize}
	\item \textbf{Docencia:} La Universidad del Quindío contribuye a la transformación de la sociedad, mediante la formación integral desde el ser, el saber y el hacer, de líderes reflexivos y gestores del cambio; con estándares de calidad, a través de una oferta de formación en diferentes metodologías, que responda a una sociedad basada en el conocimiento.

	\item \textbf{Investigación:} Una investigación pertinente, que aporte a la solución de las problemáticas del desarrollo e integrada con la extensión y proyección social.

	\item \textbf{Extensión y Desarrollo Social:} Apoyada en una gestión creativa y con estándares de calidad.

	\item \textbf{Responsabilidad Social:} Educando en tiempos del posconflicto y de la consolidación de la paz.
\end{itemize}

\section{Visión del GRID}
La misión de la Universidad del Quindío se complementa con su visión institucional, la cual también es adoptada por el \GRID. A continuación se presenta la visión del \GRID:\@

\begin{quote}
	\textit{En el año 2025, la Universidad del Quindío estará consolidada como una institución \textit{Pertinente --- Creativa --- Integradora}, acreditada de alta calidad, con reconocimiento nacional e internacional en sus procesos de formación a través de diferentes metodologías, de investigación, extensión, proyección y responsabilidad social.}
\end{quote}

A partir de esta visión, se destacan los siguientes enfoques estratégicos:

\begin{itemize}
	\item \textbf{Gestión:} La Universidad del Quindío estará consolidada como una institución \textit{Pertinente --- Creativa --- Integradora}.

	\item \textbf{Docencia:} Acreditada de alta calidad en sus procesos de formación a través de diferentes metodologías.

	\item \textbf{Investigación:} Consolidada como pertinente y de alta calidad en sus procesos de investigación.

	\item \textbf{Extensión y Desarrollo Social:} Procesos creativos e integradores en proyección social.

	\item \textbf{Responsabilidad Social:} Reconocimientos en sus procesos de responsabilidad social.
\end{itemize}

\section{Impacto del proyecto en el GRID}

El proyecto tiene como objetivo apoyar los procesos de docencia, investigación
y extensión mediante la especificación de una arquitectura de tecnologías de
virtualización basada en contenedores (\VBC).
Este trabajo se enfoca en la identificación de una tecnología de contenerización que \textbf{agregue valor a los procesos del grupo}, beneficiando a docentes, estudiantes y cualquier parte interesada que participe en los proyectos y actividades desarrolladas por este grupo de investigación.

\section{Caracterización de la infraestructura tecnológica del GRID}
En la presente sección se van a especificar las características técnicas de la infraestructura tecnológica del \GRID\ disponible para temas de virtualización. Esta caracterización se construyó usando una \href{https://docs.google.com/spreadsheets/d/14NBv72ucVTrLqGIldYdIsjdBGt3QlgwcblcVRis-DaQ/edit?usp=sharing}{macro} que permite facilitar la recolección de información técnica de los servidores y equipos de cómputo disponibles en el \GRID.

Los cuadros~\ref{tab:torre-hp-1},~\ref{tab:torre-2},~\ref{tab:torre-3} y~\ref{tab:torre-4} presentan las fichas técnicas de cuatro servidores tipo torre pertenecientes al inventario del Grupo de investigación. Se trata de equipos marca HP, sin modelo específico identificado, pero con características homogéneas en su configuración física y técnica: disponen de ocho puertos USB (cuatro frontales y cuatro posteriores), entradas y salidas de audio y micrófono, conexión HDMI, lector de DVD, tres interfaces Ethernet, puerto DisplayPort y conectores PS/2 para teclado y ratón. Su propósito principal es funcionar como hipervisores con XCP-ng, y se encuentran adaptados para soportar procesos de virtualización. Estos recursos, identificados con diferentes códigos de inventario y números en el \CPD\, hacen parte del \textit{cluster} actual, usados para el desarrollo de proyectos dentro del \GRID.
% Archivo de caracterización de infraestructura corregido

% Torre HP 1
\begin{table}[H]
\centering
\scriptsize % Tamaño de fuente más pequeño
\setlength{\tabcolsep}{2pt} % Menor espacio entre columnas
\renewcommand{\arraystretch}{1.0} % Espaciado más ajustado
\caption{Ficha técnica --- Torre 1}\label{tab:torre-hp-1}
\begin{tabular}{|p{0.5\textwidth}|p{0.2\textwidth}|} % Columnas más estrechas
\hline
\multicolumn{2}{|l|}{\textbf{DESCRIPCIÓN FÍSICA:} Servidor tipo torre} \\ \hline
\textbf{TIPO DE RECURSO:} Torre &
\multirow{5}{*}{\includegraphics[width=0.18\textwidth,keepaspectratio]{tablas-images/cp1/torres/torre-1.png}} \\ \cline{1-1}
\textbf{MODELO:} Desconocido & \\ \cline{1-1}
\textbf{MARCA:} HP & \\ \cline{1-1}
\textbf{CÓDIGO DE INVENTARIO:} 7 24390 49867 3 & \\ \cline{1-1}
\textbf{NÚMERO EN CPD:} 14 & \\ \hline
\multicolumn{2}{|l|}{\textbf{ESPECIFICACIONES TÉCNICAS}} \\ \hline
\multicolumn{2}{|p{0.7\textwidth}|}{ % Ancho ajustado
- 8 USB (4 frontal, 4 trasera)
- Audio y micrófono
- HDMI
- Lector DVD
- 3 Ethernet
- DisplayPort
- PS/2 (Teclado/Ratón)
} \\ \hline
\multicolumn{2}{|l|}{\textbf{PROPÓSITO:} Hipervisor XCP-ng} \\ \hline
\multicolumn{2}{|l|}{\textbf{OPORTUNIDAD DE USO:} Proyectos del \GRID} \\ \hline
\multicolumn{2}{|p{0.7\textwidth}|}{\textbf{OBSERVACIONES:} Sin modelo. Equipo adaptado para virtualización.} \\ \hline
\end{tabular}
\end{table}

% Torre 2
\begin{table}[H]
\centering
\scriptsize
\setlength{\tabcolsep}{2pt}
\renewcommand{\arraystretch}{1.0}
\caption{Ficha técnica --- Torre 2}\label{tab:torre-2}
\begin{tabular}{|p{0.5\textwidth}|p{0.2\textwidth}|}
\hline
\multicolumn{2}{|l|}{\textbf{DESCRIPCIÓN FÍSICA:} Servidor tipo torre} \\ \hline
\textbf{TIPO DE RECURSO:} Torre & 
\multirow{5}{*}{\includegraphics[width=0.18\textwidth,keepaspectratio]{tablas-images/cp1/torres/torre-1.png}} \\ \cline{1-1}
\textbf{MODELO:} Desconocido & \\ \cline{1-1}
\textbf{MARCA:} HP & \\ \cline{1-1}
\textbf{CÓDIGO DE INVENTARIO:} 7 24390 49861 1 & \\ \cline{1-1}
\textbf{NÚMERO EN CPD:} 12 & \\ \hline
\multicolumn{2}{|l|}{\textbf{ESPECIFICACIONES TÉCNICAS}} \\ \hline
\multicolumn{2}{|p{0.7\textwidth}|}{
- 8 USB (4 frontal, 4 trasera)
- Audio y micrófono
- HDMI
- Lector DVD
- 3 Ethernet
- DisplayPort
- PS/2 (Teclado/Ratón)
} \\ \hline
\multicolumn{2}{|l|}{\textbf{PROPÓSITO:} Hipervisor XCP-ng} \\ \hline
\multicolumn{2}{|l|}{\textbf{OPORTUNIDAD DE USO:} Proyectos del \GRID} \\ \hline
\multicolumn{2}{|p{0.7\textwidth}|}{\textbf{OBSERVACIONES:} Sin modelo. Equipo adaptado para virtualización.} \\ \hline
\end{tabular}
\end{table}

% Torre 3
\begin{table}[H]
\centering
\scriptsize
\setlength{\tabcolsep}{2pt}
\renewcommand{\arraystretch}{1.0}
\caption{Ficha técnica -- Torre 3}\label{tab:torre-3}
\begin{tabular}{|p{0.5\textwidth}|p{0.2\textwidth}|}
\hline
\multicolumn{2}{|l|}{\textbf{DESCRIPCIÓN FÍSICA:} Servidor tipo torre} \\ \hline
\textbf{TIPO DE RECURSO:} Torre & 
\multirow{5}{*}{\includegraphics[width=0.18\textwidth,keepaspectratio]{tablas-images/cp1/torres/torre-1.png}} \\ \cline{1-1}
\textbf{MODELO:} Desconocido & \\ \cline{1-1}
\textbf{MARCA:} HP & \\ \cline{1-1}
\textbf{CÓDIGO DE INVENTARIO:} 7 24390 49969 4 & \\ \cline{1-1}
\textbf{NÚMERO EN CPD:} 13 & \\ \hline
\multicolumn{2}{|l|}{\textbf{ESPECIFICACIONES TÉCNICAS}} \\ \hline
\multicolumn{2}{|p{0.7\textwidth}|}{
- 8 USB (4 frontal, 4 trasera)
- Audio y micrófono
- HDMI
- Lector DVD
- 3 Ethernet
- DisplayPort
- PS/2 (Teclado/Ratón)
} \\ \hline
\multicolumn{2}{|l|}{\textbf{PROPÓSITO:} Hipervisor XCP-ng} \\ \hline
\multicolumn{2}{|l|}{\textbf{OPORTUNIDAD DE USO:} Proyectos del \GRID} \\ \hline
\multicolumn{2}{|p{0.7\textwidth}|}{\textbf{OBSERVACIONES:} Sin modelo. Equipo adaptado para virtualización.} \\ \hline
\end{tabular}
\end{table}

% Torre 4
\begin{table}[H]
\centering
\scriptsize
\setlength{\tabcolsep}{2pt}
\renewcommand{\arraystretch}{1.0}
\caption{Ficha técnica --- Torre 4}\label{tab:torre-4}
\begin{tabular}{|p{0.5\textwidth}|p{0.2\textwidth}|}
\hline
\multicolumn{2}{|l|}{\textbf{DESCRIPCIÓN FÍSICA:} Servidor tipo torre} \\ \hline
\textbf{TIPO DE RECURSO:} Torre & 
\multirow{5}{*}{\includegraphics[width=0.18\textwidth,keepaspectratio]{tablas-images/cp1/torres/torre-1.png}} \\ \cline{1-1}
\textbf{MODELO:} Desconocido & \\ \cline{1-1}
\textbf{MARCA:} HP & \\ \cline{1-1}
\textbf{CÓDIGO DE INVENTARIO:} 7 24390 49879 4 & \\ \cline{1-1}
\textbf{NÚMERO EN CPD:} 14 & \\ \hline
\multicolumn{2}{|l|}{\textbf{ESPECIFICACIONES TÉCNICAS}} \\ \hline
\multicolumn{2}{|p{0.7\textwidth}|}{
- 8 USB (4 frontal, 4 trasera)
- Audio y micrófono
- HDMI
- Lector DVD
- 3 Ethernet
- DisplayPort
- PS/2 (Teclado/Ratón)
} \\ \hline
\multicolumn{2}{|l|}{\textbf{PROPÓSITO:} Hipervisor XCP-ng} \\ \hline
\multicolumn{2}{|l|}{\textbf{OPORTUNIDAD DE USO:} Proyectos del \GRID} \\ \hline
\multicolumn{2}{|p{0.7\textwidth}|}{\textbf{OBSERVACIONES:} Sin modelo. Equipo adaptado para virtualización.} \\ \hline
\end{tabular}
\end{table}


Los cuadros~\ref{tab:torre-5} y~\ref{tab:torre-6} detallan las características de dos servidores tipo torre HP, modelo G9, destinados a operar como hipervisores bajo XCP-ng. Ambos equipos, identificados en el inventario con los códigos 72992 y 72976 y ubicados en los puestos 22 y 21 del CPD, respectivamente, cuentan con especificaciones técnicas similares: nueve puertos USB (cuatro frontales y cinco posteriores), entradas y salidas de audio y micrófono, conexión HDMI, lector de DVD, una interfaz Ethernet, dos puertos DisplayPort y un procesador Intel vPro i9, lo que les otorga un mayor rendimiento frente a los servidores del mismo entorno. Estos equipos han sido adaptados para virtualización, reforzando la infraestructura tecnológica del GRID y ampliando las capacidades de cómputo necesarias para el desarrollo de entornos virtualizados de alto desempeño.
% Torre 5
\begin{table}[H]
\centering
\scriptsize
\setlength{\tabcolsep}{2pt}
\renewcommand{\arraystretch}{1.0}
\caption{Ficha técnica --- Torre 5}\label{tab:torre-5}
\begin{tabular}{|p{0.5\textwidth}|p{0.2\textwidth}|}
\hline
\multicolumn{2}{|l|}{\textbf{DESCRIPCIÓN FÍSICA:} Servidor tipo torre} \\ \hline
\textbf{TIPO DE RECURSO:} Torre & 
\multirow{5}{*}{\includegraphics[width=0.18\textwidth,keepaspectratio]{tablas-images/cp1/torres/torre-2.png}} \\ \cline{1-1}
\textbf{MODELO:} G9 & \\ \cline{1-1}
\textbf{MARCA:} HP & \\ \cline{1-1}
\textbf{CÓDIGO DE INVENTARIO:} 72992 & \\ \cline{1-1}
\textbf{NÚMERO EN CPD:} 22 & \\ \hline
\multicolumn{2}{|l|}{\textbf{ESPECIFICACIONES TÉCNICAS}} \\ \hline
\multicolumn{2}{|p{0.7\textwidth}|}{
- 9 USB (4 frontal, 5 trasera)
- Audio y micrófono
- HDMI
- Lector DVD
- 1 Ethernet
- 2 DisplayPort
- Procesador Intel vPro i9
} \\ \hline
\multicolumn{2}{|l|}{\textbf{PROPÓSITO:} Hipervisor XCP-ng} \\ \hline
\multicolumn{2}{|l|}{\textbf{OPORTUNIDAD DE USO:} Proyectos del \GRID} \\ \hline
\multicolumn{2}{|p{0.7\textwidth}|}{\textbf{OBSERVACIONES:} Equipo adaptado para virtualización.} \\ \hline
\end{tabular}
\end{table}

% Torre 6
\begin{table}[H]
\centering
\scriptsize
\setlength{\tabcolsep}{2pt}
\renewcommand{\arraystretch}{1.0}
\caption{Ficha técnica --- Torre 6}
\label{tab:torre-6}
\begin{tabular}{|p{0.5\textwidth}|p{0.2\textwidth}|}
\hline
\multicolumn{2}{|l|}{\textbf{DESCRIPCIÓN FÍSICA:} Servidor tipo torre} \\ \hline
\textbf{TIPO DE RECURSO:} Torre & 
\multirow{5}{*}{\includegraphics[width=0.18\textwidth,keepaspectratio]{tablas-images/cp1/torres/torre-2.png}} \\ \cline{1-1}
\textbf{MODELO:} G9 & \\ \cline{1-1}
\textbf{MARCA:} HP & \\ \cline{1-1}
\textbf{CÓDIGO DE INVENTARIO:} 72976 & \\ \cline{1-1}
\textbf{NÚMERO EN CPD:} 21 & \\ \hline
\multicolumn{2}{|l|}{\textbf{ESPECIFICACIONES TÉCNICAS}} \\ \hline
\multicolumn{2}{|p{0.7\textwidth}|}{
- 9 USB (4 frontal, 5 trasera)
- Audio y micrófono
- HDMI
- Lector DVD
- 1 Ethernet
- 2 DisplayPort
- Procesador Intel vPro i9
} \\ \hline
\multicolumn{2}{|l|}{\textbf{PROPÓSITO:} Hipervisor XCP-ng} \\ \hline
\multicolumn{2}{|l|}{\textbf{OPORTUNIDAD DE USO:} Proyectos del \GRID} \\ \hline
\multicolumn{2}{|p{0.7\textwidth}|}{\textbf{OBSERVACIONES:} Equipo adaptado para virtualización.} \\ \hline
\end{tabular}
\end{table}

El cuadro~\ref{tab:torre-7} presenta la ficha técnica del servidor tipo torre identificado como Torre 7, perteneciente a la marca Argom Tech y sin código de inventario asignado, ubicado en el puesto 11 del \CPD\. Este equipo dispone de un procesador Intel Pentium 62030 de 3.00 GHz con arquitectura x64, memoria RAM de 16 GB, disco duro de 1 TB, unidad de CD/DVD y tarjetas de video y sonido integradas. Su propósito principal es operar como hipervisor bajo la plataforma XCP-ng. Cabe destacar que en este servidor se encuentra implementada la solución arquitectónica propuesta, lo que lo convierte en un recurso fundamental para la validación y consolidación del entorno de contenedores diseñado para el \GRID.
% Torre 7
\begin{table}[H]
\centering
\scriptsize
\setlength{\tabcolsep}{2pt}
\renewcommand{\arraystretch}{1.0}
\caption{Ficha técnica --- Torre 7}\label{tab:torre-7}
\begin{tabular}{|p{0.5\textwidth}|p{0.2\textwidth}|}
\hline
\multicolumn{2}{|l|}{\textbf{DESCRIPCIÓN FÍSICA:} Servidor tipo torre} \\ \hline
\textbf{TIPO DE RECURSO:} Torre & 
\multirow{5}{*}{\includegraphics[width=0.18\textwidth,keepaspectratio]{tablas-images/cp1/torres/ATX.png}} \\ \cline{1-1}
\textbf{MODELO:} Argom tech & \\ \cline{1-1}
\textbf{MARCA:} Argom tech & \\ \cline{1-1}
\textbf{CÓDIGO DE INVENTARIO:} Sin código & \\ \cline{1-1}
\textbf{NÚMERO EN CPD:} 11 & \\ \hline
\multicolumn{2}{|l|}{\textbf{ESPECIFICACIONES TÉCNICAS}} \\ \hline
\multicolumn{2}{|p{0.7\textwidth}|}{
- Procesador: Intel Pentium 62030 3.00GHz
- Arquitectura: X64
- RAM: 16GB
- Disco: 1024GB
- Unidad CD/DVD: Sí
- Tarjeta video: Integrada
- Tarjeta sonido: Integrada
} \\ \hline
\multicolumn{2}{|l|}{\textbf{PROPÓSITO:} Hipervisor XCP-ng} \\ \hline
\multicolumn{2}{|l|}{\textbf{OPORTUNIDAD DE USO:} Proyectos del \GRID} \\ \hline
\multicolumn{2}{|p{0.7\textwidth}|}{\textbf{OBSERVACIONES:} Equipo adaptado para virtualización.} \\ \hline
\end{tabular}
\end{table}

Los cuadros~\ref{tab:rack-1} al~\ref{tab:rack-5} describen cinco servidores tipo rack, modelo System x3250 M4 de la marca IBM, los cuales  se ubican en el rack 1 del \CPD. Estos equipos, identificados con diferentes códigos dentro de la topología de red se ubica en los puestos 55, 54, 53 y 52. Presentan características técnicas homogéneas: procesador Intel Xeon E3\-1220v2, controlador SATA integrado, dos ranuras PCI Express, cuatro unidades SAS/SATA con capacidad de intercambio en caliente, fuente redundante de 460W, sistema de gestión integrado, cuatro puertos USB (dos frontales y dos posteriores) y lector de DVD.\@Su propósito principal es la prestación de servicios de cómputo y la provisión de recursos tecnológicos para estudiantes, además de la generación de máquinas virtuales orientadas a prácticas académicas.
% Rack 1
\begin{table}[H]
\centering
\scriptsize
\setlength{\tabcolsep}{2pt}
\renewcommand{\arraystretch}{1.0}
\caption{Ficha técnica --- Rack 1}\label{tab:rack-1}
\begin{tabular}{|p{0.5\textwidth}|p{0.2\textwidth}|}
\hline
\multicolumn{2}{|l|}{\textbf{DESCRIPCIÓN FÍSICA:} Servidor tipo rack} \\ \hline
\textbf{TIPO DE RECURSO:} Servidor & 
\multirow{5}{*}{\includegraphics[width=0.18\textwidth,keepaspectratio]{tablas-images/cp1/racks/rack-1.png}} \\ \cline{1-1}
\textbf{MODELO:} System x3250 M4 & \\ \cline{1-1}
\textbf{MARCA:} IBM & \\ \cline{1-1}
\textbf{CÓDIGO DE INVENTARIO:} 7 24390 50981 & \\ \cline{1-1}
\textbf{NUMERO EN CPD:} 55 & \\ \hline
\multicolumn{2}{|l|}{\textbf{ESPECIFICACIONES TÉCNICAS}} \\ \hline
\multicolumn{2}{|p{0.7\textwidth}|}{
- Procesador Intel Xeon E3-1220v2
- Controlador SATA integrado
- 2 ranuras PCI Express
- 4 SAS/SATA intercambio en caliente
- Fuente redundante 460W
- Gestión del sistema
- 4 puertos USB (2 frontal, 2 trasero)
- Lector DVD
} \\ \hline
\multicolumn{2}{|l|}{\textbf{PROPÓSITO:} Prestación de servicios de cómputo} \\ \hline
\multicolumn{2}{|p{0.7\textwidth}|}{\textbf{IMPACTO:} 
- Servicios a estudiantes
- Máquinas virtuales para prácticas} \\ \hline
\multicolumn{2}{|p{0.7\textwidth}|}{\textbf{OBSERVACIONES:} Ninguna} \\ \hline
\end{tabular}
\end{table}

% Rack 2
\begin{table}[H]
\centering
\scriptsize
\setlength{\tabcolsep}{2pt}
\renewcommand{\arraystretch}{1.0}
\caption{Ficha técnica --- Rack 2}
\label{tab:rack-2}
\begin{tabular}{|p{0.5\textwidth}|p{0.2\textwidth}|}
\hline
\multicolumn{2}{|l|}{\textbf{DESCRIPCIÓN FÍSICA:} Servidor tipo rack} \\ \hline
\textbf{TIPO DE RECURSO:} Servidor & 
\multirow{5}{*}{\includegraphics[width=0.18\textwidth,keepaspectratio]{tablas-images/cp1/racks/rack-1.png}} \\ \cline{1-1}
\textbf{MODELO:} System x3250 M4 & \\ \cline{1-1}
\textbf{MARCA:} IBM & \\ \cline{1-1}
\textbf{CÓDIGO DE INVENTARIO:} 7 24390 50980 & \\ \cline{1-1}
\textbf{NUMERO EN CPD:} 54 & \\ \hline
\multicolumn{2}{|l|}{\textbf{ESPECIFICACIONES TÉCNICAS}} \\ \hline
\multicolumn{2}{|p{0.7\textwidth}|}{
- Procesador Intel Xeon E3-1220v2
- Controlador SATA integrado
- 2 ranuras PCI Express
- 4 SAS/SATA intercambio en caliente
- Fuente redundante 460W
- Gestión del sistema
- 4 puertos USB (2 frontal, 2 trasero)
- Lector DVD
} \\ \hline
\multicolumn{2}{|l|}{\textbf{PROPÓSITO:} Prestación de servicios de cómputo} \\ \hline
\multicolumn{2}{|p{0.7\textwidth}|}{\textbf{IMPACTO:} 
- Servicios a estudiantes
- Máquinas virtuales para prácticas} \\ \hline
\multicolumn{2}{|p{0.7\textwidth}|}{\textbf{OBSERVACIONES:} Ninguna} \\ \hline
\end{tabular}
\end{table}

% Rack 3
\begin{table}[H]
\centering
\scriptsize
\setlength{\tabcolsep}{2pt}
\renewcommand{\arraystretch}{1.0}
\caption{Ficha técnica --- Rack 3}
\label{tab:rack-3}
\begin{tabular}{|p{0.5\textwidth}|p{0.2\textwidth}|}
\hline
\multicolumn{2}{|l|}{\textbf{DESCRIPCIÓN FÍSICA:} Servidor tipo rack} \\ \hline
\textbf{TIPO DE RECURSO:} Servidor & 
\multirow{5}{*}{\includegraphics[width=0.18\textwidth,keepaspectratio]{tablas-images/cp1/racks/rack-1.png}} \\ \cline{1-1}
\textbf{MODELO:} System x3250 M4 & \\ \cline{1-1}
\textbf{MARCA:} IBM & \\ \cline{1-1}
\textbf{CÓDIGO DE INVENTARIO:} 7 24390 50980 & \\ \cline{1-1}
\textbf{NUMERO EN CPD:} 53 & \\ \hline
\multicolumn{2}{|l|}{\textbf{ESPECIFICACIONES TÉCNICAS}} \\ \hline
\multicolumn{2}{|p{0.7\textwidth}|}{
- Procesador Intel Xeon E3-1220v2
- Controlador SATA integrado
- 2 ranuras PCI Express
- 4 SAS/SATA intercambio en caliente
- Fuente redundante 460W
- Gestión del sistema
- 4 puertos USB (2 frontal, 2 trasero)
- Lector DVD
} \\ \hline
\multicolumn{2}{|l|}{\textbf{PROPÓSITO:} Prestación de servicios de cómputo} \\ \hline
\multicolumn{2}{|p{0.7\textwidth}|}{\textbf{IMPACTO:} 
- Servicios a estudiantes
- Máquinas virtuales para prácticas} \\ \hline
\multicolumn{2}{|p{0.7\textwidth}|}{\textbf{OBSERVACIONES:} Ninguna} \\ \hline
\end{tabular}
\end{table}

% Rack 4
\begin{table}[H]
\centering
\scriptsize
\setlength{\tabcolsep}{2pt}
\renewcommand{\arraystretch}{1.0}
\caption{Ficha técnica --- Rack 4}
\label{tab:rack-4}
\begin{tabular}{|p{0.5\textwidth}|p{0.2\textwidth}|}
\hline
\multicolumn{2}{|l|}{\textbf{DESCRIPCIÓN FÍSICA:} Servidor tipo rack} \\ \hline
\textbf{TIPO DE RECURSO:} Servidor & 
\multirow{5}{*}{\includegraphics[width=0.18\textwidth,keepaspectratio]{tablas-images/cp1/racks/rack-1.png}} \\ \cline{1-1}
\textbf{MODELO:} System x3250 M4 & \\ \cline{1-1}
\textbf{MARCA:} IBM & \\ \cline{1-1}
\textbf{CÓDIGO DE INVENTARIO:} 7 24390 48735 & \\ \cline{1-1}
\textbf{NUMERO EN CPD:} 52 & \\ \hline
\multicolumn{2}{|l|}{\textbf{ESPECIFICACIONES TÉCNICAS}} \\ \hline
\multicolumn{2}{|p{0.7\textwidth}|}{
- Procesador Intel Xeon E3-1220v2
- Controlador SATA integrado
- 2 ranuras PCI Express
- 4 SAS/SATA intercambio en caliente
- Fuente redundante 460W
- Gestión del sistema
- 4 puertos USB (2 frontal, 2 trasero)
- Lector DVD
} \\ \hline
\multicolumn{2}{|l|}{\textbf{PROPÓSITO:} Prestación de servicios de cómputo} \\ \hline
\multicolumn{2}{|p{0.7\textwidth}|}{\textbf{IMPACTO:} 
- Servicios a estudiantes
- Máquinas virtuales para prácticas} \\ \hline
\multicolumn{2}{|p{0.7\textwidth}|}{\textbf{OBSERVACIONES:} Ninguna} \\ \hline
\end{tabular}
\end{table}

% Rack 5
\begin{table}[H]
\centering
\scriptsize
\setlength{\tabcolsep}{2pt}
\renewcommand{\arraystretch}{1.0}
\caption{Ficha técnica --- Rack 5}\label{tab:rack-5}
\begin{tabular}{|p{0.5\textwidth}|p{0.2\textwidth}|}
\hline
\multicolumn{2}{|l|}{\textbf{DESCRIPCIÓN FÍSICA:} Servidor tipo rack} \\ \hline
\textbf{TIPO DE RECURSO:} Servidor & 
\multirow{5}{*}{\includegraphics[width=0.18\textwidth,keepaspectratio]{tablas-images/cp1/racks/rack-1.png}} \\ \cline{1-1}
\textbf{MODELO:} System x3250 M4 & \\ \cline{1-1}
\textbf{MARCA:} IBM & \\ \cline{1-1}
\textbf{CÓDIGO DE INVENTARIO:} 51474 & \\ \cline{1-1}
\textbf{NUMERO EN CPD:} 52 & \\ \hline
\multicolumn{2}{|l|}{\textbf{ESPECIFICACIONES TÉCNICAS}} \\ \hline
\multicolumn{2}{|p{0.7\textwidth}|}{
- Procesador Intel Xeon E3-1220v2
- Controlador SATA integrado
- 2 ranuras PCI Express
- 4 SAS/SATA intercambio en caliente
- Fuente redundante 460W
- Gestión del sistema
- 4 puertos USB (2 frontal, 2 trasero)
- Lector DVD
} \\ \hline
\multicolumn{2}{|l|}{\textbf{PROPÓSITO:} Prestación de servicios de cómputo} \\ \hline
\multicolumn{2}{|p{0.7\textwidth}|}{\textbf{IMPACTO:} 
- Servicios a estudiantes
- Máquinas virtuales para prácticas} \\ \hline
\multicolumn{2}{|p{0.7\textwidth}|}{\textbf{OBSERVACIONES:} Ninguna} \\ \hline
\end{tabular}
\end{table}

El cuadro~\ref{tab:nas-1} presenta la ficha técnica del NAS, un sistema de almacenamiento en red modelo TS-832PX-4G de la marca QNAP, el cual se integra como un recurso estratégico en la infraestructura tecnológica. Este dispositivo cuenta con un procesador Annapurna Labs Alpine AL-324 de cuatro núcleos, memoria RAM de 4 GB DDR4 expandible hasta 16 GB, ocho bahías para discos SATA de 3.5 o 2.5 pulgadas, dos puertos de red RJ45 de 2.5 GbE y dos de 10 GbE, además de tres puertos USB 3.2 Gen 1 y ranuras PCIe para expansión. Su consumo energético es de 50.8 W en funcionamiento y 27 W en reposo, lo que permite operación continua. El propósito principal de este NAS es ofrecer almacenamiento compartido y redundante en red, ofreciendo disponibilidad de respaldos y acceso confiable a archivos para proyectos académicos.
% NAS 1
\begin{table}[H]
\centering
\scriptsize
\setlength{\tabcolsep}{2pt}
\renewcommand{\arraystretch}{1.0}
\caption{Ficha técnica --- NAS 1}\label{tab:nas-1}
\begin{tabular}{|p{0.5\textwidth}|p{0.2\textwidth}|}
\hline
\multicolumn{2}{|l|}{\textbf{DESCRIPCIÓN FÍSICA:} Sistema de almacenamiento en red} \\ \hline
\textbf{TIPO DE RECURSO:} NAS & 
\multirow{5}{*}{\includegraphics[width=0.18\textwidth,keepaspectratio]{tablas-images/cp1/NAS/nas-1.png}} \\ \cline{1-1}
\textbf{MODELO:} TS-832PX-4G & \\ \cline{1-1}
\textbf{MARCA:} QNAP & \\ \cline{1-1}
\textbf{CÓDIGO DE INVENTARIO:} Por definir & \\ \cline{1-1}
\textbf{FECHA DE ADQUISICIÓN:} & \\ \hline
\multicolumn{2}{|l|}{\textbf{ESPECIFICACIONES TÉCNICAS}} \\ \hline
\multicolumn{2}{|p{0.7\textwidth}|}{
- Procesador: Annapurna Labs Alpine AL-324, 4 núcleos
- RAM: 4 GB DDR4 (exp. a 16 GB)
- Bahías: 8 para discos SATA 3.5"/2.5"
- Puertos Red: 2 x RJ45 2.5GbE, 2 x 10GbE
- Puertos USB: 3 x USB 3.2 Gen 1
- Consumo: 50.8 W (func.), 27 W (reposo)
- Expansión: Ranuras PCIe
} \\ \hline
\multicolumn{2}{|p{0.7\textwidth}|}{\textbf{PROPÓSITO:} Almacenamiento compartido y redundante en red} \\ \hline
\multicolumn{2}{|p{0.7\textwidth}|}{\textbf{IMPACTO:} - Sin NAS: no hay backups ni acceso a archivos} \\ \hline
\multicolumn{2}{|p{0.7\textwidth}|}{\textbf{OBSERVACIONES:} Ninguna} \\ \hline
\end{tabular}
\end{table}

El cuadro~\ref{tab:firewall-1} corresponde a la ficha técnica de un firewall implementado en el \CPD, el cual funciona como sistema de seguridad de red. Se trata de un equipo DELL, modelo PowerEdge T100, con código de inventario 7 24390 46288 9 y chasis en formato torre. Sus especificaciones técnicas incluyen un procesador Intel Xeon E3110 de 3 GHz, 4 GB de memoria DDR2 (2 x 2 GB), un disco duro de 1 TB tipo HDD de 3.5 pulgadas, fuente de alimentación de 305 W, unidad óptica DVD-ROM y conectividad Ethernet. Su propósito principal es la seguridad de la red de servidores, evitando accesos no autorizados y protegiendo los recursos computacionales. Actúa como primera barrera de defensa en el ecosistema de virtualización y servicios del grupo, contribuyendo a la integridad de los datos y a la continuidad de los procesos misionales.
% Firewall 1
\begin{table}[H]
\centering
\scriptsize
\setlength{\tabcolsep}{2pt}
\renewcommand{\arraystretch}{1.0}
\caption{Ficha técnica --- Firewall}\label{tab:firewall-1}
\begin{tabular}{|p{0.5\textwidth}|p{0.2\textwidth}|}
\hline
\multicolumn{2}{|l|}{\textbf{DESCRIPCIÓN FÍSICA:} Sistema de seguridad de red} \\ \hline
\textbf{TIPO DE RECURSO:} Firewall & 
\multirow{5}{*}{\includegraphics[width=0.18\textwidth,keepaspectratio]{tablas-images/cp1/firewall/firewall.png}} \\ \cline{1-1}
\textbf{MODELO:} PowerEdge T100 & \\ \cline{1-1}
\textbf{MARCA:} DELL & \\ \cline{1-1}
\textbf{CÓDIGO DE INVENTARIO:} 7 24390 46288 9 & \\ \cline{1-1}
\textbf{NÚMERO EN CPF:} No especificado & \\ \hline
\multicolumn{2}{|l|}{\textbf{ESPECIFICACIONES TÉCNICAS}} \\ \hline
\multicolumn{2}{|p{0.7\textwidth}|}{
- Procesador: Intel Xeon E3110 3 GHz
- Memoria: 4 GB DDR2 (2 x 2 GB)
- Almacenamiento: 1 TB HDD 3.5"
- Fuente alimentación: 305 W
- Unidad óptica: DVD-ROM
- Tipo chasis: Torre
- Ethernet
} \\ \hline
\multicolumn{2}{|l|}{\textbf{PROPÓSITO:} Seguridad de la red de servidores} \\ \hline
\multicolumn{2}{|p{0.7\textwidth}|}{\textbf{IMPACTO:} - Protege equipos de usuarios no autorizados} \\ \hline
\multicolumn{2}{|p{0.7\textwidth}|}{\textbf{OBSERVACIONES:} Ninguna} \\ \hline
\end{tabular}
\end{table}

\section{Caracterización de servicios del GRID}
El \GRID\ ofrece una variedad de servicios tecnológicos a la comunidad académica, especialmente a los estudiantes de Ingeniería de Sistemas y Computación. Estos servicios incluyen:

\subsection{Servicios actuales}
Los servicios actuales del GRID se centran en la provisión de infraestructura de \TI\, incluyendo máquinas virtuales, almacenamiento y redes. Estos servicios son utilizados principalmente por estudiantes y docentes, los cuales se especifican en el cuadro~\ref{tab:servicios-actuales}.

\begin{table}[H]
	\centering
	\renewcommand{\arraystretch}{1.2}
	\setlength{\tabcolsep}{3pt}
	\tiny
	\begin{tabularx}{\textwidth}{|>{\raggedright\arraybackslash}p{0.25\textwidth}|X|}
		\hline
		\textbf{NOMBRE DEL SERVICIO}    & Máquinas Virtuales para estudiantes y docentes                                                                    \\
		\hline
		\textbf{TIPO DE SERVICIO}       & Servicio educativo                                                                                                \\
		\hline
		\textbf{PROPÓSITO}              & Proveer máquinas virtuales a profesores, estudiantes y administrativos para prácticas académicas mediante XCP-ng. \\
		\hline
		\textbf{HORARIO DISPONIBILIDAD} & 24/7                                                                                                              \\
		\hline
		\textbf{TIEMPO FUNCIONAMIENTO}  & 3 años                                                                                                            \\
		\hline
		\textbf{RECURSOS}               & Servidores torre y rack                                                                                           \\
		\hline
		\textbf{TECNOLOGÍAS}            & Hipervisor tipo I (XCP-ng)                                                                                        \\
		\hline
		\textbf{IMPACTO}                & Indisponibilidad afecta actividades misionales del grupo de investigación y programa de Ingeniería de sistemas.   \\
		\hline
	\end{tabularx}
	\caption{Caracterización de los servicios actuales del GRID}\label{tab:servicios-actuales}
\end{table}

\subsection{Servicios esperados}
Los servicios esperados por el GRID se orientan al aprovisionamiento de contenedores a traves de tecnologias VBC. Se espera que los usuarios de distintos dominios ( educación, investigación, extensión ) puedan beneficiarse de este nuevo servicio que se especifica en el cuadro ~\ref{tab:servicios-esperados}

\begin{table}[H]
	\centering
	\renewcommand{\arraystretch}{1.2}
	\setlength{\tabcolsep}{3pt}
	\tiny
	\begin{tabularx}{\textwidth}{|>{\raggedright\arraybackslash}p{0.25\textwidth}|X|}
		\hline
		\textbf{NOMBRE DEL SERVICIO}    & Ambientes computacionales basados en \VBC                                                                       \\
		\hline
		\textbf{TIPO DE SERVICIO}       & Servicio de educación, investigación y extensión                                                                \\
		\hline
		\textbf{PROPÓSITO}              & Proveer ambientes computacionales mediante tecnologías \VBC\ y mecanismo de orquestación                        \\
		\hline
		\textbf{HORARIO DISPONIBILIDAD} & 24/7                                                                                                            \\
		\hline
		\textbf{RECURSOS}               & Servidores torre y rack                                                                                         \\
		\hline
		\textbf{TECNOLOGÍAS}            & Por determinar                                                                                                  \\
		\hline
		\textbf{IMPACTO}                & Indisponibilidad afecta actividades misionales del grupo de investigación y programa de Ingeniería de sistemas. \\
		\hline
	\end{tabularx}
	\caption{Caracterización de los servicios esperados del GRID}\label{tab:servicios-esperados}
\end{table}

\section{Descripción de la oportunidad}

Actualmente el Grupo de Investigación en Redes, Información y Distribución (\GRID) presenta diversas necesidades y oportunidades con relación a los servicios tecnológicos que ofrece a la Universidad del Quindío, en apoyo a sus objetivos misionales de docencia, investigación y extensión.

En este contexto, el \GRID\ busca identificar tecnologías emergentes que permitan potenciar su capacidad de brindar servicios tecnológicos avanzados, tanto para su propio beneficio como para la comunidad académica dentro de su ámbito de influencia.

Con relación a lo anterior, las \textbf{tecnologías de virtualización basadas en procesos} se presentan como una alternativa para optimizar la gestión de recursos y servicios de tecnología informática (\TI). Aunque el \GRID\ cuenta con una infraestructura basada en máquinas virtuales, gestionadas mediante el hipervisor tipo I XCP-ng, además de iniciativas de computación \textit{Desktop Cloud}, aún requiere de instancias computacionales más livianas para ampliar su oferta de servicios computacionales dirigidos a la comunidad académica, especialmente a los estudiantes del programa de Ingeniería de Sistemas y Computación de la Universidad del Quindío.

Como mencionan \textit{Sepúlveda et al.} (2022), las tecnologías de virtualización han proliferado en los últimos años y constituyen la base subyacente de infraestructuras modernas como el \textit{cloud computing}. A partir de esta proliferación, las tecnologías de \VBC\ se presentan como una alternativa que podría potenciar la gestión de los recursos relacionados con la infraestructura de \TI\ del \GRID.

Las tecnologías de \VBC\ representan una opción de virtualización que requiere menos recursos computacionales para su operación \citep{Xavier2013}, y que, en conjunto con las máquinas virtuales ya existentes en el \GRID\, podrían constituir una oferta de servicios de \TI\ con mayor diversificación, escalabilidad, flexibilidad y mantenibilidad, para satisfacer los requerimientos del contexto académico del grupo de investigación.

\section{Resumen de la entrevista con el cliente}

Para comprender mejor las necesidades y expectativas del \GRID\, se realizó una entrevista con el cliente.

\begin{itemize}
	\item \textbf{Entrevistado:} Luis Eduardo Sepúlveda Rodríguez
	\item \textbf{Fecha:} 6 de febrero de 2025
	\item \textbf{Duración:} 25 minutos
	\item \textbf{Link:} \href{https://drive.google.com/file/d/1rIc9xOsyDqumlTV-QXcw0inPyIbSEHLz/view?usp=sharing}{click aquí}
	\item \textbf{Asistentes:} Anubis Haxard Correa Urbano, José Alejandro Arias Pinzón
\end{itemize}

\subsection{Misión del grupo \GRID}
En el minuto 1:01 se menciona que: el grupo de investigación no declara una misión y visión distinta a la de su organización, la Universidad del Quindío. En consecuencia, estos elementos se heredan directamente de la institución.

\subsection{Actividades del grupo de investigación}
En el minuto 1:10 se menciona que: Aunque su nombre podría llevar al sesgo de pensar que se dedica exclusivamente a la investigación, el \GRID se desarrolla en los tres pilares misionales: docencia, investigación y extensión. Participa en actividades académicas como la enseñanza en el programa de Ingeniería de Sistemas y Computación, desarrolla investigaciones mediante el método científico, y realiza actividades de proyección social y formación complementaria.

\subsection{La virtualización basada en contenedores como una oportunidad}
En el minuto 3:01 se menciona que: Las tecnologías de \VBC pueden aportar al cumplimiento de la misión institucional. Actualmente se utiliza Docker por ser un estándar de facto, no por una evaluación formal. Existen alternativas como Podman, ContainerD y LXC que también podrían apoyar los tres pilares institucionales.

\subsection{El problema de la multitud de herramientas}
En el minuto 3:44 se menciona que: Existen muchas herramientas que podrían cumplir los objetivos institucionales. Escoger una tecnología adecuada no es trivial y requiere comprender el contexto organizacional. Por eso, este proyecto busca ofrecer una solución arquitectónica basada en \VBC\, que sirva a estudiantes y docentes para comprender el estado y las tendencias de estas tecnologías.

\subsection{Difusión para apoyar a otros grupos e instituciones}
En el minuto 5:32 se menciona que: Aunque el proyecto se enmarca en el \GRID\, sus resultados podrían ser útiles para otras universidades, grupos de investigación o incluso la industria. Elegir tecnologías \VBC\@estratégicamente puede tener gran valor, por lo que se plantea la necesidad de difundir los avances y resultados del proyecto.

\subsection{Restricción en los recursos}
En el minuto 7:08 se menciona que: El \GRID\ cuenta con infraestructura de \TI\, pero debe considerar su contexto y limitaciones. Soluciones que requieran licencias costosas o hardware especializado no son viables. Por tanto, las opciones de código abierto cobran especial relevancia.

\subsection{Impacto del proyecto en los campos de estudio del \GRID}
En el minuto 14:50 se menciona que: Los pilares misionales abarcan muchas actividades. El \GRID se enfoca en áreas como desarrollo de software, pensamiento computacional, computación paralela, análisis de datos, inteligencia artificial, redes, infraestructura de \TI, y HPC.\@Este proyecto busca fortalecer esas áreas mediante el uso de tecnologías \VBC.\@

\subsection{Necesidad de orquestación entre máquinas virtuales y contenedores}
Fuera del audio se menciona que: Actualmente los servicios se ejecutan sobre máquinas virtuales con XCP-ng. Se considera deseable —aunque no obligatorio— que la solución propuesta permita integrar contenedores con máquinas virtuales completas mediante un clúster, para maximizar el aprovechamiento de la infraestructura existente.

\bigskip\noindent \textit{Nota:} Este documento es solo un resumen de la entrevista. El audio incluye una explicación adicional del mapeo SMS que no se encuentra transcrita aquí.

\ChapterImageStar[cap:revisionLiteratura]{Revisión sistemática de la literatura}{./images/fondo.png}\label{cap:revisionLiteratura}

\mbox{}\\
\section{Construcción de la bitácora}

En búsqueda de una base teórica para la elección de una tecnología de virtualización basada en contenedores, 
se realizó una revisión del estado del arte. Esta revisión se completó en diferentes etapas:

\subsection{Planeación}

Esta etapa consistió en establecer el propósito general que se buscaba alcanzar con el \SMS\ (\textit{Systematic Mapping Study}). 
A su vez, definió aspectos como objetivos ver cuadro ~\ref{tab:metas}, preguntas de investigación ver cuadro~\ref{tab:preguntas} y métricas ver cuadro~\ref{tab:metricas}. Para ello, se siguió el modelo 
Objetivo-Pregunta-Métrica (\textit{Goal-Question-Metric}, GQM). A continuación, se definen los objetivos del \SMS\ aplicado 
a las tecnologías de virtualización basadas en contenedores en el cuadro.

\subsubsection{Definición de metas para el \SMS}

\begin{table}[H]
\centering
\renewcommand{\arraystretch}{1.2} % Espaciado reducido
\footnotesize % Texto más pequeño
\begin{tabular}{|c|p{13cm}|}  % Columna más ancha (12cm)
\hline
\textbf{Meta} & \textbf{Descripción} \\ \hline
G1 & Identificar trabajos con \VBC\ en docencia, investigación y extensión. \\ \hline
G2 & Clasificar trabajos con \VBC\ en dominios de \TI: desarrollo software, pensamiento computacional, computación paralela, análisis datos, IA, redes, infraestructura \TI, HPC, etc. \\ \hline
\end{tabular}
\caption{Definición de metas del SMS}
\label{tab:metas}
\end{table}

\subsubsection{Definición de preguntas de investigación}
\begin{table}[H]
\centering
\renewcommand{\arraystretch}{1.2} % Espaciado reducido
\scriptsize % Texto más pequeño
\begin{tabular}{|c|c|p{6cm}|p{6cm}|} % Columnas más estrechas
\hline
\textbf{Meta} & \textbf{Pregunta} & \textbf{Descripción} & \textbf{Motivación} \\ \hline

G1 & Q1 &
\textit{¿Cuáles trabajos con \VBC\ impactan positivamente en docencia, investigación y extensión?} &
\VBC\ ofrece transversalidad y reproducibilidad, facilitando transporte de soluciones TI entre dominios. \\ \hline

G2 & Q2 &
\textit{¿Cuáles trabajos con \VBC\ contribuyen en dominios de \TI?} &
Proporcionar base sólida para comprender estado del arte de \VBC\ sin análisis profundo. \\ \hline

\end{tabular}
\caption{Definición de preguntas de investigación del SMS}
\label{tab:preguntas}
\end{table}

\subsubsection{Definición de métricas}

\begin{table}[H]
\centering
\renewcommand{\arraystretch}{1.2} % Menor espaciado entre filas
\footnotesize % Texto más pequeño
\begin{tabular}{|c|p{9cm}|} % Columna de descripción más estrecha
\hline
\textbf{Métrica} & \textbf{Descripción} \\ \hline
M1 & Cantidad de trabajos identificados en cada dominio de \TI. \\ \hline
M2 & Cantidad de trabajos incluidos en educación. \\ \hline
M3 & Cantidad de trabajos incluidos en investigación. \\ \hline
M4 & Cantidad de trabajos incluidos en extensión. \\ \hline
\end{tabular}
\caption{Definición de métricas del SMS}
\label{tab:metricas}
\end{table}

\section{Búsqueda de estudios}

Esta etapa comprendió las siguientes secciones: 
\begin{enumerate}
  \item Estrategia de búsqueda, ya sea independiente o combinada;
  \item Identificación general de estudios;
  \item Revisión de estudios; y finalmente,
  \item Selección de estudios para incluir en el SMS.\@
\end{enumerate}

\subsection{Estrategia de búsqueda}

Este trabajo combinó las estrategias de búsqueda en bases de datos y búsqueda en bola de nieve. 
Para la estrategia de búsqueda en bases de datos, se seleccionaron las siguientes bases de datos: ACM, IEEE Xplore, Springer, Taylor \& Francis y Science Direct.

\subsection{Búsqueda en bases de datos}\label{subsec:busquedaBasesDatos}
Se seleccionaron las siguientes bases de datos para este propósito: ACM, IEEE Xplore, Springer, Taylor \& Francis y Science Direct.

\subsubsection{Identificación de estudios mediante búsqueda en bases de datos}\label{subsubsec:identificacionEstudios}
En esta etapa del proceso fue necesario establecer las palabras clave que serían útiles en las cadenas de búsqueda para cada una de las bases de datos seleccionadas. 
Los términos consideran los elementos identificados en la etapa de planificación, para lo cual también se utilizó el modelo PICOC ( \textit{Population}, \textit{Intervention}, \textit{Comparator}, \textit{Outcome}, and \textit{Context} ) como guía metodológica.

\begin{table}[H]
\centering
\renewcommand{\arraystretch}{1.2} % Espaciado reducido
\footnotesize % Texto más pequeño
\begin{tabularx}{\textwidth}{|p{0.18\textwidth}|X|} % Columna izquierda más estrecha
\hline
\textbf{Componente} & \textbf{Descripción} \\ \hline

Población & Trabajos sobre \VBC\ aplicadas en \TI, con énfasis en educación, investigación y extensión. \\ \hline

Intervención & Identificación y clasificación de trabajos \VBC\ en dominios de \TI. \\ \hline

Comparación & 
\textbf{1.} Comparación de proyectos \VBC\ por tasa de éxito en cada dominio \TI.\@        
\textbf{2.} Análisis de impacto de \VBC\ vs. otras soluciones en docencia, investigación y extensión. \\ \hline
Salida & Estructura de clasificación de trabajos \VBC\ que impactan en docencia, investigación y extensión. \\ \hline
Contexto & Docencia, investigación y extensión con apropiación de \VBC\ en \TI. \\ \hline
\end{tabularx}
\caption{Modelo PICOC}
\end{table}

\begin{table}[H]
\centering
\scriptsize
\setlength{\tabcolsep}{3pt}
\renewcommand{\arraystretch}{1.1}
\begin{tabular}{|p{3cm}|p{2.5cm}|p{2.5cm}|p{3cm}|p{3cm}|}
\hline
\textbf{Población} & \textbf{Intervención} & \textbf{Comparación} & \textbf{Salida} & \textbf{Contexto} \\
\hline
VBC \newline Dominios de \TI Educación Investigación Extensión & Identificación \newline Clasificación & Tasa de éxito \newline Evidencia de uso & Clasificación de trabajos \newline relacionados con VBC en cada dominio de \TI & Docencia Investigación Extensión \\
\hline
\end{tabular}
\caption{Palabras clave identificadas usando el modelo PICOC}
\label{tab:picoc}
\end{table}

\begin{table}[H]
\centering
\scriptsize
\setlength{\tabcolsep}{4pt}
\begin{tabular}{|p{5cm}|p{9.5cm}|}
\hline
\textbf{Palabras clave} & \textbf{Sinónimos} \\
\hline
Container-based virtualization & Application virtualization, Docker, Lightweight Virtualization \\
\hline
Education & Education System, Education Development, Higher Education \\
\hline
Research & Research Group, Research Proposal \\
\hline
Outreach & \IT\ Services, Technology Infrastructure, Cloud Computing \\
\hline
\end{tabular}
\caption{Palabras clave para la búsqueda en base de datos}
\label{tab:keywords}
\end{table}

\begin{table}[H]
\centering
\scriptsize
\setlength{\tabcolsep}{4pt}
\renewcommand{\arraystretch}{1.2}
\begin{tabular}{|p{4cm}|p{5cm}|p{5.5cm}|}
\hline
\textbf{Categoría} & \textbf{Inclusión} & \textbf{Exclusión} \\
\hline
Campos & Resumen & --- \\
\hline
Tipo de publicación & Artículos de revistas y conferencias & Tesis y capítulos de libros \\
\hline
Área/Disciplina & Management, \CS\, IT Management, engineering & Áreas no relacionadas con virtualización, \CS\ y \IT\ Management \\
\hline
Período & 2022 a 2024 & Antes de 2022 \\
\hline
Idioma & Inglés & --- \\
\hline
\end{tabular}
\caption{Criterios de Inclusión/Exclusión}\label{tab:criterios-inclusion-exclusion}
\end{table}

\subsubsection{Búsqueda en bases de datos}\label{par:busquedaBasesDatos}
Las cadenas de búsqueda se construyeron utilizando las palabras clave y sinónimos identificados en la tabla \ref{tab:keywords}.
Las cadenas de búsqueda específicas para cada base de datos se encuentran en~\ref{sec:cadenas-busqueda}.

\subsubsection{Resumen de la búsqueda en bases de datos sin criterios de inclusión/exclusión}\label{subsubsec:resumenBusqueda}
Este es el resultado antes de aplicar criterios de exclusión

\begin{table}[H]
	\centering
	\scriptsize
	\setlength{\tabcolsep}{4pt}
	\renewcommand{\arraystretch}{1.1}
	\begin{tabular}{|l|c|c|c|c|c|c|}
		\hline
		\textbf{Bases de datos} & \textbf{ACM} & \textbf{IEEE} & \textbf{Springer} & \textbf{Science Direct} & \textbf{Taylor \& Francis} & \textbf{Total} \\
		\hline
		Sin aplicar criterios   & 518          & 0             & 209               & 120                     & 0                          & 847            \\
		\hline
		Con criterios aplicados & 315          & 0             & 63                & 101                     & 0                          & 479            \\
		\hline
	\end{tabular}
	\caption{Resumen de la búsqueda en bases de datos con criterios de inclusión/exclusión}
	\label{tab:bases-sin-criterio}
\end{table}

\begin{figure}[H]
    \centering
    \includegraphics[scale=0.9]{tablas-images/cp2/bases-sin-criterio.png}
    \caption{Resumen de la búsqueda en bases de datos sin criterios de inclusión/exclusión}\label{fig:tabla-resumen-busqueda}
\end{figure}

\subsubsection{Aplicación de criterios de exclusión de las bases de datos}
Esta búsqueda se realizó considerando los criterios de exclusión e inclusión definidos previamente.

Las cadena de búsqueda son exactamente iguales que antes, este punto se diferencia por la aplicación de 
filtros. Para ver las capturas de pantalla veáse el apéndice B sección 2.

\subsection{Resumen de la búsqueda en bases de datos con criterios de inclusión/exclusión}\label{subsec:resumenBusquedaCriterios}

\begin{table}[H]
\centering
\scriptsize
\setlength{\tabcolsep}{4pt}
\renewcommand{\arraystretch}{1.1}
\begin{tabular}{|l|c|c|c|c|c|c|}
\hline
\textbf{Bases de datos} & \textbf{ACM} & \textbf{IEEE} & \textbf{Springer} & \textbf{Science Direct} & \textbf{Taylor \& Francis} & \textbf{Total} \\
\hline
Sin aplicar criterios & 189 & 426 & 4562 & 353 & 1000 & 6530 \\
\hline
Con criterios aplicados & 48 & 134 & 592 & 46 & 156 & 976 \\
\hline
\end{tabular}
\caption{Resumen de la búsqueda en bases de datos con criterios de inclusión/exclusión}\label{tab:resumen-busqueda}
\end{table}
\begin{figure}[H]
    \centering
    \includegraphics[scale=0.9] {tablas-images/cp2/bases-con-criterio.png}
    \caption{Resumen de la búsqueda en bases de datos con criterios de inclusión/exclusión}\label{fig:tabla-resumen-busqueda-con-criterio}
\end{figure}

\section{Eliminación de duplicados}\label{sec:eliminacionDuplicados}
La eliminación de duplicados se realizó haciendo uso de la herramienta de gestión de referencias Mendeley. Luego de obtener los artículos se agregaron a Mendeley y esta herramienta se encargó de eliminar duplicados. En este punto se eliminaron 274 artículos duplicados.

\section{Priorización de estudios}\label{sec:priorizacionEstudios}

Luego de la selección inicial de los artículos, se procedió a revisar el \textit{title}, \textit{abstract} y \textit{keywords} de cada uno. Como resultado de esta revisión, se generaron métricas de calidad para cada artículo, con el fin de priorizar aquellos más relevantes para la investigación. Las métricas utilizadas fueron las siguientes:

\begin{itemize}
    \item \textbf{SCI} (Science Citation Index)
    \item \textbf{CVI} (Core Value Index)
    \item \textbf{IRRQ} (Index Relation Research Question)
\end{itemize}

Este proceso inició con un total de 771 artículos, los cuales fueron evaluados según su alineamiento con los objetivos de la investigación. La evaluación temática permitió identificar un total de 110 artículos con una relación directa con el enfoque planteado.

\section{Estrategia de búsqueda usando bola de nieve}\label{sec:bolaDeNieve}

En esta etapa, se seleccionó el primer cuartil según el índice \textbf{IRRQ}, lo que resultó en un total de 24 artículos. Adicionalmente, se incluyeron dos artículos por criterio de inclusión directa, estableciendo así una línea base de \textbf{26 artículos}. 

Sobre esta base, se aplicó la estrategia de \textit{bola de nieve} en ambas direcciones: hacia adelante y hacia atrás. Como resultado, se obtuvieron \textbf{87 artículos} mediante la técnica hacia atrás y \textbf{495 artículos} mediante la técnica hacia adelante. 

Esto definió un nuevo conjunto de artículos para un proceso de selección adicional (\textit{screening}). En esta fase, se eliminaron \textbf{14 duplicados} y \textbf{452 artículos} fueron descartados por no estar alineados con la investigación. 

Finalmente, se obtuvo un total de \textbf{116 artículos} mediante esta estrategia de búsqueda ampliada.

\section{Diagrama de búsqueda}\label{sec:diagramaBusqueda}

\subsection{Usando cadenas de búsqueda}
En el diagrama~\ref{tab:tabla-diagrama-cadena-busqueda} se puede apreciar la estrategia de búsqueda de artículos por medio de base de datos, aplicando las cadenas de búsqueda, se consolidaron los resultados de distintas bases de datos para obtener un total de 6530 resultados, posteriormente y aplicando criterios de exclusión se redujo esta cantidad a menos de 1000 resultados. Adicional a los criterios de exclusión, también se hizo eliminación de artículos duplicados, 205 por parte del ~\textit{Reference Manager} (Mendeley), y 69 por parte del ~\textit{SMS-Builder} para un total de 274 artículos removidos. Finalmente, se realiza la etapa de screening, donde se leen las secciones claves de los artículos, como ~\textit{abstract}, ~\textit{keywords} e introducción, a través de esto se pudo descargar 671 artículos que no eran pertinentes para el estudio.
\begin{table}[H]
    \centering
    \includegraphics[scale=0.13]{tablas-images/cp2/diagrama-cadena-busqueda.png}
    \caption{Diagrama de la cadena de búsqueda}\label{tab:tabla-diagrama-cadena-busqueda}
\end{table}\label{img:busqueda-bd}

\subsection{Usando bola de nieve}
Como segunda estrategia de búsqueda dentro del SMS, se aplicó la técnica de ~\textit{Snowball}, la cual consiste en extraer artículos adicionales a partir de las referencias citadas en los estudios obtenidos en la estrategia anterior y de los estudios que citan a estos. De los 110 estudios del paso anterior, se seleccionan aquellos que tengan un SCI más relevante y se agrega uno por inclusión directa, con esto se obtiene un total de 25 artículos de línea base. Aplicando snowball hacia adelante (artículos referenciados) se obtienen 87 nuevos estudios, aplicando snowball hacia atrás (artículos que referencian el artículo base) se obtienen 495, para un total de 582. Finalmente, aplicando criterios de exclusión de remoción de duplicados y aplicando la técnica de screening, se obtiene un resultado de 116 artículos incluidos por bola de nieve. Todo este proceso se puede apreciar en la gráfica~\ref{tab:tabla-diagrama-bola-nieve-busqueda}.
\begin{table}[H]
    \centering
    \includegraphics[scale=0.14]{tablas-images/cp2/diagrama-bola-nieve-busqueda.png}
    \caption{Diagrama de la búsqueda en bola de nieve}\label{tab:tabla-diagrama-bola-nieve-busqueda}
\end{table}

\section{Identificación de estudios}

\subsection{Artículos por año y métricas}
A continuación se presentan las gráficas que resumen los resultados de las estrategias anteriores. En la figura~\ref{fig:diagrama-articulos-ano-metrica} se pueden visualizar las métricas de calidad, separadas por los últimos 3 años y mostrando el promedio en cada métrica de los artículos publicados en esos años. En la figura~\ref{fig:tipos-articulos} se muestra el conteo de artículos por su tipo específico, el cual puede ser uno de 3 opciones: 'Revista', 'Conferencia' o 'Genérico'. Vemos que la mayoría de artículos provienen de revistas. En la figura~\ref{fig:estrategia-busqueda-articulos} se detalla la cantidad de artículos que se extrajo de cada estrategia. Finalmente, en la figura~\ref{fig:diagrama-red-articulos} se puede apreciar un diagrama de red, que segrega por colores los tópicos más relacionados entre sí
En la figura~\ref{fig:diagrama-articulos-ano-metrica} se pueden visualizar las métricas de calidad, separadas por los últimos 3 años y mostrando el promedio en cada métrica de los artículos publicados en esos años, así como un apartado donde se calcula la sumatoria de estas métricas por cada año.

\begin{figure}[H]
    \centering
    \includegraphics[scale=0.7]{tablas-images/cp2/diagrama-articulos-ano-metrica.png}
    \caption{Artículos por métricas y año}\label{fig:diagrama-articulos-ano-metrica}
\end{figure}

En la figura~\ref{fig:tipos-articulos} se muestra el conteo de artículos por su tipo específico, el cual puede ser uno de 3 opciones: 'Revista', 'Conferencia' o 'Genérico'. Vemos que la mayoría de artículos provienen de revistas.
\begin{figure}[H]
    \centering
    \includegraphics[scale=0.5]{tablas-images/cp2/tipos-articulos.png}
    \caption{Artículos por tipo}\label{fig:tipos-articulos}
\end{figure}

En la figura~\ref{fig:estrategia-busqueda-articulos} se detalla la cantidad de artículos que se extrajeron de cada estrategia. Se puede observar que la estrategia que generó más artículos fue la técnica de ~\textit{Snowball}.
\begin{figure}[H]
    \centering
    \includegraphics[scale=0.8]{tablas-images/cp2/estrategia-busqueda-articulos.png}
    \caption{Estrategia de búsqueda de artículos}\label{fig:estrategia-busqueda-articulos}
\end{figure}

Finalmente, en la figura~\ref{fig:diagrama-red-articulos} se puede apreciar un diagrama de red, que segrega por colores los tópicos más relacionados entre sí, vemos 4 grandes grupos: IA, cloud computing, virtualización, desarrollo de software.
\begin{figure}[H]
    \centering
    \includegraphics[scale=0.9]{tablas-images/cp2/diagrama-red-busqueda.png}
    \caption{Diagrama de red de los artículos}\label{fig:diagrama-red-articulos}
\end{figure}

\section{Información de la herramienta}

\noindent
La herramienta utilizada para este proceso de revisión de la literatura fue \textbf{SMS-BUILDER}, la cual se encuentra disponible en \textit{Docker Hub}. El estudio realizado puede consultarse en el siguiente enlace:

\begin{center}
\href{https://sms-vbc.iti.grid.uniquindio.edu.co/}{\texttt{https://sms-vbc.iti.grid.uniquindio.edu.co/}}
\end{center}

\noindent
Adicionalmente, se implementaron procesos de respaldo como medida de seguridad. Estos \textit{backups} fueron almacenados en ubicaciones diferentes, siguiendo la estrategia de respaldo \textbf{3--2--1}.

\input{capitulos/desarrolloMetodologico/caracterizacionVBC.tex}
\input{capitulos/benchmarking.tex}
\ChapterImageStar[cap:dar]{Análisis de Decisiones y Resolución}{./images/fondo.png}\label{cap:dar}
\mbox{}\\
\section{Metodología de evaluación}

La metodología de evaluación que se aplicó para la elección de la tecnología de Virtualización Basada en Contenedores (VBC) fue DAR (Decision, analysis and resolution) de CMMI \citep{CMMIInstitute2010}. Esta metodología permitió evaluar las necesidades del grupo GRID a través de un proceso estructurado que consideró múltiples alternativas, criterios de evaluación bien definidos y un análisis comparativo. En este caso, se identificaron y analizaron diversas tecnologías VBC —como Docker, Podman, LXC o Kata Containers— aplicando criterios como el tipo de licencia, la compatibilidad con herramientas de orquestación, el rendimiento entre otros. Mediante el uso de matrices DOFA, se logró visualizar las fortalezas y debilidades de cada opción, facilitando una selección alineada con los objetivos estratégicos del sistema. Así, el uso de DAR no solo aportó transparencia al proceso, sino también trazabilidad y justificación técnica frente a una decisión para la arquitectura de infraestructura basada en contenedores.

El proceso de evaluación quedó registrado en un vídeo explicativo disponible en \href{https://youtu.be/xOmuQs2RX2c}{link}.

\section{Resultados de la evaluación}

\begin{table}[H]
    \centering
    \includegraphics[width=\textwidth] {tablas-images/cp5/DAR.png}
    \caption{Análisis de Decisiones y Resolución (DAR) aplicado a la selección de VBC}\label{tab:tabla-dar}
\end{table}

\section{Criterios de evaluación}

\subsection{VBC (¿Es una tecnología basada en contenedores?)}
Este criterio define si la tecnología analizada entra dentro de la categoría de virtualización basada en contenedores, lo cual es el punto de partida para que pueda ser considerada en el análisis. Se evalúa como Sí (SI) o No (NO).

\subsection{Tipo de licencia}
Se analiza el tipo de licencia bajo la cual se distribuye la tecnología, ya que esto afecta su adopción en proyectos académicos o comerciales. Las licencias permisivas (como Apache 2.0 o BSD) permiten mayor libertad de uso y modificación, mientras que licencias restrictivas (como AGPL o licencias propietarias) imponen ciertas limitaciones legales o técnicas.

\subsection{Posibilidad de orquestación}
Se refiere a la capacidad de la tecnología para integrarse con herramientas de orquestación como Kubernetes, Docker Swarm o Nomad, lo cual es clave para la gestión automatizada de contenedores a gran escala. Una mayor puntuación indica mejor compatibilidad y soporte para estas herramientas.

\subsection{Compatibilidad con imágenes de Docker Hub}
Evalúa si la tecnología puede ejecutar imágenes obtenidas directamente desde Docker Hub, el repositorio más utilizado para contenedores. Esto facilita la reutilización de contenedores existentes y la integración con flujos de trabajo ya establecidos.

\subsection{Soporte para redes personalizadas}
Determina si la tecnología permite la creación y gestión de redes personalizadas entre contenedores. Este aspecto es fundamental en arquitecturas distribuidas, donde la comunicación entre servicios debe configurarse de forma segura.

\subsection{Persistencia de datos / volúmenes}
Analiza si la solución permite la persistencia de datos, es decir, que los datos generados dentro de un contenedor puedan mantenerse incluso después de reiniciarlo o eliminarlo. Esto se logra mediante el uso de volúmenes o sistemas de almacenamiento externos.

\subsection{Documentación}
Se valora la calidad, profundidad y accesibilidad de la documentación oficial. Una buena documentación facilita el aprendizaje, la resolución de problemas y la implementación efectiva de la tecnología.

\subsection{Soporte al proyecto}
Considera el respaldo que tiene la tecnología por parte de la comunidad, empresas o fundaciones (como CNCF o Red Hat). Esto incluye mantenimiento activo, actualizaciones regulares, y foros o canales de ayuda disponibles.

\subsection{Popularidad}
Este criterio mide la adopción y visibilidad de la tecnología, lo cual puede reflejar su madurez, confianza del mercado y disponibilidad de talento capacitado. Se puede estimar por métricas como el número de estrellas en GitHub.

\subsection{Consumo de recursos}
Evalúa el nivel de consumo de recursos respecto al uso de CPU, memoria y almacenamiento. Se valora según lo que mencionan las organizaciones en este aspecto.

\subsection{Compatibilidad de orquestación}
Difiere levemente del punto 4.3, ya que aquí se mide qué tan bien se integra con los orquestadores, considerando estabilidad, plugins nativos y experiencia de uso. Un puntaje alto indica integración fluida y confiable.

\subsection{Costo de implementación y operación en ambientes productivos}
Este criterio analiza los costos asociados a poner en marcha la tecnología en un entorno real. Incluye licencias, infraestructura, tiempo de configuración y mantenimiento. Una puntuación alta significa bajo costo o costo nulo, lo cual es ideal para instituciones académicas o proyectos con presupuesto limitado.

\section{Tecnología VBC ganadora}

Del análisis comparativo realizado, Containerd se posiciona como la tecnología de virtualización basada en contenedores con mejor desempeño general. Destaca por su alta compatibilidad con Docker Hub, soporte para redes y volúmenes, excelente integración con orquestadores como Kubernetes, y una licencia permisiva que facilita su adopción. Además, cuenta con una sólida documentación y un respaldo activo de la comunidad. Estas características hacen de Containerd la opción adecuada para ser implementada en ambientes productivos del grupo de investigación GRID, combinando los diferentes criterios definidos desde el grupo de investigación.

\section{Análisis DAR del motor de Kubernetes}

\begin{table}[H]
    \centering
    \includegraphics[width=\textwidth] {tablas-images/cp5/dar-k8s.png}
    \caption{Análisis de Decisiones y Resolución aplicado a la selección del motor de Kubernetes}\label{tab:tabla-dar}
\end{table}
\ChapterImageStar[cap:disenio]{Diseño de la solución}{./images/fondo.png}\label{cap:disenio}
\mbox{}\\

El presente capítulo tiene como objetivo fundamental traducir la decisión estratégica —la implementación de los Universos Grid y Parallel— en un diseño técnico y estratégico concreto y ejecutable. Aquí se especifica la arquitectura de la solución, detallando los componentes de HTCondor que serán configurados o modificados, los flujos de trabajo establecidos para cada universo y los criterios de interoperabilidad con el Universo Vanilla preexistente. Este diseño busca establecer las bases técnicas y metodológicas para la fase de implementación, propendiendo por la robustez, escalabilidad e integridad de la infraestructura de computación distribuida resultante. Además, busca ayudar al Grupo \GRID con el cumplimiento de sus objetivos estratégicos y el cierre de la brecha identificada en la caracterización.

\section{Definición de estrategia}
La definición de la estrategia de diseño se fundamenta en una filosofía de desarrollo guiado por pruebas (\TDD), que es una disciplina de diseño y programación donde cada línea de código está escrita en respuesta a una prueba que el programador escribe justo antes de escribir el código \parencite{4163024}. Antes de especificar los componentes arquitectónicos, se establece un conjunto de casos de prueba estructurados que definen de manera formal y verificable el comportamiento esperado de los Universos \textit{Grid} y \textit{Parallel}. Estos casos, que funcionan como criterios de aceptación guiarán de manera iterativa el diseño de la solución, asegurando que cada decisión de implementación esté directamente alineada con la validación funcional del sistema.

Por último, se considera necesario establecer un nombre para el sistema. Esto con el fin de dar mayor claridad al lector y hacer más visibles los momentos en los que se habla del sistema resultado de este proyecto. Se opta por el "Grid App" haciendo referencia al Grupo \GRID, que es el sujeto nuclear de este proyecto; al Universo Grid, que es uno de los Universos que se decidió implementar para este proyecto de aplicación después de la toma de decisiones estructurada mediante el análisis \DAR y el modelo computacional Grid que es básicamente una infraestructura que provee gran capacidad computacional a un sistema distribuido haciendo uso de recursos geográficamente distribuidos \parencite{8974490}.

\section{Casos de prueba}

\subsection{Definición de requisitos}
\noindent
Para iniciar con la definición de casos de prueba, se establecen los requisitos que debe cumplir \textit{Grid App} que están fundamentados en el análisis preliminar hecho anteriormente, principalmente en la descripción de la oportunidad para el Grupo \GRID.

\subsubsection{Requisitos funcionales}
\noindent
Según \textcite{159342} un requisito funcional especifica una función que el sistema o que un componente del sistema es capaz de hacer. A continuación se definen los requisitos funcionales para \textit{Grid App}.
\begin{table}[H]
	\centering
	\sffamily\scriptsize
	\setlength{\tabcolsep}{4pt}
	\renewcommand{\arraystretch}{1.3}
	\caption{Requisitos funcionales para \textit{Grid App}}
	\label{table:requisitosFuncionales}
	\begin{tabular}{|p{0.1\textwidth}|p{0.2\textwidth}|p{0.7\textwidth}|}
		\toprule
		\textbf{ID}                              & \textbf{Título}                               & \textbf{Descripción} \\
		\midrule
		RF1 & Soporte para Universos adicionales & El sistema debe permitir la ejecución de trabajos en al menos dos Universos adicionales a Vanilla (Grid y Parallel) \\
		\midrule
		RF2 & Orquestación de múltiples clústeres & El sistema debe permitir enviar trabajos hacia diferentes clústeres (Parallel, Vanilla) a través del Universo Grid. \\
		\midrule
		RF3 & Selección de destino & El sistema debe permitir que el usuario especifique el clúster de destino. \\
		\midrule
		RF4 & Monitoreo de trabajos & El sistema debe permitir el monitoreo de los trabajos enviados a los clústeres, mostrando su estado al usuario. \\
		\midrule
		RF5 & Ejecución MPI en Parallel & Un clúster debe soportar ejecución de trabajos basados en MPI en múltiples nodos (≥ N nodos configurados). \\
		\midrule
		RF6 & Redirección Grid-Parallel & El Grid Manager debe aceptar trabajos enviados al Universo Grid y, si la naturaleza del trabajo es paralelo, redirigirlos correctamente al Universo Parallel. \\
        \midrule
		RF7 & Redirección Grid-Vanilla & El Grid Manager debe aceptar trabajos enviados al Universo Grid y, si la naturaleza del trabajo es distribuido, redirigirlos correctamente al Universo Vanilla. \\
        \midrule
		RF8 & Registro centralizado de errores & El Grid Manager debe centralizar \textit{logs} de fallos de ejecución provenientes de cada clúster. \\
		\bottomrule
	\end{tabular}
\end{table}

\subsubsection{Requisitos no-funcionales}
\noindent
Según \textcite{4384163} un requisito no-funcional es un atributo o una restricción de un sistema. A continuación se definen los requisitos no-funcionales para \textit{Grid App}.
\begin{table}[H]
	\centering
	\sffamily\scriptsize
	\setlength{\tabcolsep}{4pt}
	\renewcommand{\arraystretch}{1.3}
	\caption{Requisitos no-funcionales para \textit{Grid App}}
	\label{table:requisitosNoFuncionales}
	\begin{tabular}{|p{0.1\textwidth}|p{0.2\textwidth}|p{0.7\textwidth}|}
		\toprule
		\textbf{ID}                              & \textbf{Título}                               & \textbf{Descripción} \\
		\midrule
		RNF1 & Transparencia de uso & El usuario no debe preocuparse por las configuraciones internas de cada clúster; el Grid Manager abstrae el destino mediante una interfaz más amigable. \\
		\midrule
		RNF2 & Usabilidad & La configuración de los nuevos Universos debe estar documentada y ser accesible mediante manual de despliegue para los usuarios del Grupo GRID. \\
		\midrule
		RNF3 & Disponibilidad & En caso de que un clúster esté inactivo, el Grid Manager debe registrar el error y permitir redirección a otro clúster disponible si el Universo es compatible. \\
		\bottomrule
	\end{tabular}
\end{table}

\subsection{Pruebas}
\noindent
Para esta sección, se definen las pruebas y los pasos de cada prueba a ejecutar 

\section{Modelado del sistema en Archimate}
ArchiMate es un lenguaje de modelado estandarizado por~\textit{The Open Group} que permite representar de manera estructurada y clara las diferentes capas de una arquitectura empresarial: negocio, aplicación y tecnología. Su propósito es brindar una visión integrada que facilite la comunicación entre los distintos actores de un proyecto y que muestre cómo los procesos de negocio, los sistemas de información y la infraestructura tecnológica se relacionan entre sí.
En particular, ArchiMate se organiza en vistas que permiten enfocarse en aspectos específicos: la vista de negocio describe los procesos y actores implicados, la vista de aplicación se centra en los sistemas de software que apoyan esos procesos, y la vista de tecnología aborda la infraestructura que soporta todo el ecosistema. Gracias a este enfoque por capas, los diagramas ayudan a identificar dependencias, puntos de optimización y la coherencia general de la solución arquitectónica.

\subsection{Vista de negocio}


\subsection{Vista de aplicación}
%\begin{figure}[H]
    \centering
    \includegraphics[width=\textwidth]{tablas-images/cp6/Application-Cooperation-View.png}
    \caption{Vista de Cooperación de Aplicaciones}
\end{figure}
\begin{figure}[H]
    \centering
    \includegraphics[width=\textwidth]{tablas-images/cp6/Application-Behaviour-view.png}
    \caption{Vista de Comportamiento de Aplicaciones}
\end{figure}
\begin{figure}[H]
    \centering
    \includegraphics[width=\textwidth]{tablas-images/cp6/Application-Structure-View.png}
    \caption{Vista de Estructura de Aplicaciones}
\end{figure}

\subsection{Vista de tecnología}
%\begin{figure}[H]
    \centering
    \includegraphics[width=\textwidth]{tablas-images/cp6/Implementation-and-Installation-View.png}
    \caption{Vista de Implementación e Instalación}
\end{figure}
\begin{figure}[H]
    \centering
    \includegraphics[width=\textwidth]{tablas-images/cp6/Information-Structure-View.png}
    \caption{Vista de Estructura de Información}
\end{figure}x


\subsection{Vista general}
%\begin{figure}[H]
    \centering
    \includegraphics[scale=0.5]{tablas-images/cp6/Layered-View.png}
    \caption{Vista General en Capas}
\end{figure}

\section{Diseño por capas de la solución}

\section{Capa de infraestructura}

\begin{figure}[H]
    \centering
    %\includegraphics[width=\textwidth]{tablas-images/cp6/disenio-N1.png}
    \caption{Capa de Infraestructura}
\end{figure}

\section{Capa de virtualización}

\begin{figure}[H]
    \centering
    %\includegraphics[width=\textwidth]{tablas-images/cp6/disenio-N2.png}
    \caption{Capa de Virtualización}
\end{figure}

\section{Capa de aplicación}


\ChapterImageStar[cap:resultados]{Resultados}{./images/fondo.png}\label{cap:resultados}
\mbox{}\\

\section{Caracterización del Grupo \GRID e identificación de necesidades}
\noindent

La caracterización integral del Grupo GRID permitió identificar sus capacidades tecnológicas actuales, estructura organizacional y necesidades específicas. Se determinó que el grupo cuenta con una infraestructura de computación distribuida basada en HTCondor, compuesta por un clúster de máquinas Raspberry Pi y una infraestructura virtualizada utilizando el hipervisor XCP-ng. 

El análisis de \textit{stakeholders} reveló una estructura compleja de actores con distintos niveles de influencia, identificando al Grupo GRID como el interesado crítico, seguido por docentes de Ingeniería de Sistemas e investigadores. Se estableció que los principales beneficiarios serían estudiantes, grupos de investigación institucionales y la comunidad académica en general.

La caracterización evidenció la necesidad de ampliar las capacidades computacionales existentes mediante la incorporación de universos HTCondor adicionales que complementen la infraestructura Vanilla actual, específicamente orientados hacia aplicaciones que requieren federación de recursos (Grid) y paralelización fuertemente acoplada (Parallel).

\section{Revisión sistemática de la literatura}
\noindent

El estudio de mapeo sistemático (SMS) logró identificar y analizar 114 estudios relevantes de un total inicial de 847 documentos recuperados de cinco bases de datos académicas principales. El proceso metodológico riguroso aplicó criterios de inclusión y exclusión que redujeron el corpus inicial en un 43.44\%, seguido de la eliminación de duplicados y la aplicación de la técnica de bola de nieve.

Los resultados del SMS revelaron que la base de datos ACM fue la fuente más productiva con el 61.16\% de los artículos, mientras que IEEE y Taylor \& Francis mostraron menor productividad en las condiciones propuestas por este proyecto. El análisis temático identificó como tópicos predominantes: \textbf{Grid Computing, HPC, Cloud Computing, HTC y Parallel}, evidenciando una concentración de la investigación en estos dominios.

La distribución de universos HTCondor en la literatura mostró una clara predominancia del universo Vanilla (61 menciones), seguido por Grid (39 menciones) y Parallel (26 menciones), proporcionando evidencia cuantitativa para el proceso de toma de decisiones posterior.

\section{Análisis de Decisiones y Resolución (DAR)}
\noindent

La aplicación de la metodología DAR del modelo CMMI permitió evaluar objetivamente los universos HTCondor mediante criterios técnicos y organizacionales claramente definidos. Los criterios evaluados incluyeron: nivel de acoplamiento, capacidad de \textit{checkpointing}, popularidad académica, disponibilidad de documentación, división de roles, facilidad de implementación, compatibilidad con infraestructura existente, potencial educativo e investigativo, y soporte comunitario.

El proceso de evaluación sistemática arrojó un empate técnico entre los universos \textbf{Grid} y \textbf{Parallel}, ambos obteniendo las puntuaciones más altas. Esta situación llevó a la decisión estratégica de implementar ambos universos, maximizando así el impacto y las capacidades de la infraestructura resultante.

Los criterios que favorecieron estos universos incluyeron su alto potencial para investigación y docencia, amplia documentación disponible, capacidades técnicas diferenciadas y complementariedad con la infraestructura Vanilla existente.

\section{Diseño arquitectónico y especificación técnica}
\noindent

Se desarrolló una arquitectura integral que articula la integración de los universos Grid y Parallel con la infraestructura HTCondor existente del Grupo GRID. El diseño arquitectónico contempla:

\subsection{Arquitectura del universo Grid}
\noindent

Se especificó una arquitectura federada donde un \textit{grid manager} central coordina múltiples clústeres Vanilla independientes. La implementación incluye la configuración de un nodo coordinador (172.30.27.11) que ejecuta los \textit{daemons} especializados (SCHEDD, MASTER, GRIDMANAGER, COLLECTOR) y la adaptación de dos clústeres Vanilla (B y C) para funcionar como recursos Grid.

La arquitectura permite la transparencia de ubicación, balanceo automático de carga, tolerancia a fallos y gestión unificada de credenciales a través de diferentes dominios administrativos.

\subsection{Arquitectura del universo Parallel}
\noindent

Se diseñó una infraestructura virtualizada completamente nueva utilizando máquinas virtuales con AlmaLinux 9.6 sobre el hipervisor XCP-ng existente. Esta implementación proporciona un entorno moderno y optimizado para aplicaciones \MPI que requieren comunicación fuertemente acoplada.

La arquitectura incluye configuración especializada para OpenMPI, gestión de recursos paralelos y mecanismos de sincronización que garantizan la ejecución coordinada de procesos distribuidos.

\section{Implementación del Producto Mínimo Viable (PMV)}
\noindent

La materialización de la arquitectura propuesta se realizó mediante un PMV que integra todos los componentes del sistema diseñado:

\subsection{Grid App - Interfaz de usuario}
\noindent

Se desarrolló una aplicación web con Flask que funciona como interfaz unificada para la gestión de trabajos en los diferentes universos HTCondor. La aplicación incluye:

\begin{itemize}
    \item \textbf{Punto de envío}: Procesamiento inteligente de trabajos con generación de identificadores únicos, manejo de archivos \textit{multipart} y generación automática de \textit{submit files}.
    \item \textbf{Monitoreo de estado}: Monitoreo en tiempo real del estado de trabajos distribuidos.
    \item \textbf{Gestión de resultados}: Gestión y descarga de resultados de ejecución.
    \item \textbf{Selección de clústeres}: Selección automática de recursos basada en tipo de trabajo y disponibilidad.
\end{itemize}

\subsection{Infraestructura Grid implementada}
\noindent

La implementación del universo Grid materializó la arquitectura federada diseñada:

\begin{itemize}
    \item \textbf{\textit{Grid Manager}}: Configurado exitosamente con \textit{daemons} especializados y políticas de distribución.
    \item \textbf{Expansión de infraestructura}: Implementación de un segundo clúster Vanilla (Clúster C) demostrando escalabilidad.
    \item \textbf{Federación funcional}: Capacidad demostrada de distribuir trabajos transparentemente entre múltiples pools HTCondor.
\end{itemize}

\subsection{Infraestructura Parallel implementada}
\noindent

Se estableció exitosamente un entorno de ejecución paralelo completamente funcional:

\begin{itemize}
    \item \textbf{Clúster virtualizado}: Implementación de múltiples VMs con configuración MPI optimizada.
    \item \textbf{Integración OpenMPI}: Configuración y validación de comunicación entre procesos paralelos.
    \item \textbf{Gestión de recursos}: Asignación coordinada de recursos para trabajos fuertemente acoplados.
\end{itemize}

\section{Validación técnica y funcional}
\noindent

La validación integral del sistema implementado se realizó mediante tres escenarios de prueba específicos que cubrieron las funcionalidades críticas de ambos universos:

\subsection{Validación del universo Grid}
\noindent

\textbf{ESC-01}: Ejecución exitosa de trabajos Grid con repeticiones múltiples, validando:
\begin{itemize}
    \item Correcta distribución de trabajos entre clústeres.
    \item Gestión adecuada de archivos de entrada y salida.
    \item Monitoreo efectivo de estado de trabajos distribuidos.
\end{itemize}

\textbf{ESC-02}: Validación de trabajos parametrizables con variables dinámicas, demostrando:
\begin{itemize}
    \item Capacidad de procesamiento de parámetros variables.
    \item Flexibilidad en configuración de trabajos.
    \item Integración correcta con la interfaz \textit{Grid App}.
\end{itemize}

\subsection{Validación del universo Parallel}
\noindent

\textbf{ESC-03}: Ejecución exitosa de algoritmo \textit{quicksort} paralelo, confirmando:
\begin{itemize}
    \item Funcionamiento correcto de comunicación \MPI.
    \item Coordinación efectiva de procesos distribuidos.
    \item Capacidades de sincronización y paralelización.
\end{itemize}

Todos los casos de prueba se ejecutaron exitosamente, validando tanto los requisitos funcionales como no funcionales especificados. Las pruebas confirmaron que el sistema es estable, reproducible y adecuado para su uso en escenarios académicos y de investigación.

\section{Métricas de impacto y capacidades resultantes}
\noindent

La implementación exitosa del proyecto ha resultado en un incremento significativo de las capacidades computacionales del Grupo GRID:

\begin{itemize}
    \item \textbf{Expansión de recursos}: De un clúster Vanilla único a una infraestructura federada con tres \textit{pools} independientes.
    \item \textbf{Diversificación de capacidades}: Soporte para tres universos HTCondor (Vanilla, Grid, Parallel).
    \item \textbf{Escalabilidad demostrada}: Arquitectura probada capaz de incorporar recursos adicionales transparentemente.
    \item \textbf{Interfaz unificada}: Acceso simplificado a recursos computacionales distribuidos mediante \textit{Grid App}.
\end{itemize}

\section{Contribuciones al conocimiento}
\noindent

Este proyecto genera varias contribuciones significativas:

\begin{itemize}
    \item \textbf{Metodológica}: Marco reproducible combinando caracterización contextual, revisión sistemática, evaluación DAR y diseño arquitectónico.
    \item \textbf{Técnica}: Implementación funcional de múltiples universos HTCondor en infraestructura académica con recursos limitados.
    \item \textbf{Académica}: Documentación completa de configuraciones y procedimientos para replicación en contextos similares.
    \item \textbf{Institucional}: Fortalecimiento de capacidades de investigación y docencia en computación distribuida.
\end{itemize}

Los resultados demuestran que la incorporación de los universos Grid y Parallel en la infraestructura del Grupo \GRID representa una solución viable y eficiente para expandir el portafolio de servicios de computación distribuida, sin desestimar la inversión previa en tecnologías HTCondor y proporcionando una base sólida para el crecimiento futuro de las capacidades computacionales institucionales.

\ChapterImageStar[cap:conclusiones]{Conclusiones}{./images/fondo.png}\label{cap:conclusiones}
\mbox{}\\

El desarrollo de esta investigación ha permitido alcanzar exitosamente el objetivo general de proponer universos HTCondor para la ampliación de la infraestructura del Grupo de Investigación en Redes, Información y Distribución (GRID) de la Universidad del Quindío. A través de un enfoque metodológico riguroso que combinó caracterización contextual, revisión sistemática de literatura, análisis de decisiones estructurado y implementación práctica, se logró no solo seleccionar e implementar los universos más adecuados, sino también generar conocimiento transferible para contextos académicos similares.

\section{Sobre la metodología empleada}
\noindent

La metodología desarrollada, que integró caracterización del GRID, estudio de mapeo sistemático (SMS) y análisis de decisiones y resolución (DAR) del modelo CMMI, demostró ser efectiva para la toma de decisiones tecnológicas en entornos académicos con restricciones de recursos. El SMS proporcionó una base empírica sólida que identificó 114 estudios relevantes sobre universos HTCondor, mientras que el DAR permitió evaluar objetivamente las alternativas disponibles mediante criterios técnicos y organizacionales claramente definidos.

La caracterización integral del GRID reveló la importancia de comprender no solo las capacidades técnicas actuales, sino también la estructura de stakeholders, necesidades específicas y objetivos estratégicos antes de emprender proyectos de ampliación tecnológica. Este enfoque holístico garantizó que las soluciones implementadas se alinearan con las realidades operacionales y expectativas institucionales.

\section{Sobre la selección e implementación de universos}
\noindent

El empate técnico entre los universos Grid y Parallel en la evaluación DAR condujo a la decisión estratégica de implementar ambos, maximizando el impacto del proyecto. Esta decisión resultó acertada, ya que ambos universos abordan necesidades computacionales complementarias: el universo Grid proporciona capacidades de federación y distribución de recursos geográficamente distribuidos, mientras que el universo Parallel habilita la ejecución de aplicaciones fuertemente acopladas que requieren sincronización y comunicación intensiva.

La implementación exitosa demostró que es posible expandir infraestructuras HTCondor existentes de manera gradual e incremental, aprovechando inversiones previas mientras se incorporan nuevas capacidades. La arquitectura federada del universo Grid permitió duplicar los recursos computacionales disponibles sin requerir modificaciones disruptivas en la infraestructura vanilla preexistente.

\section{Sobre el impacto institucional y académico}
\noindent

La ampliación de la infraestructura HTCondor del GRID trasciende el ámbito técnico y genera impactos significativos en múltiples dimensiones institucionales. En el ámbito de la docencia, la disponibilidad de recursos de computación distribuida proporciona a los estudiantes de Ingeniería de Sistemas y Computación acceso a tecnologías de vanguardia que fortalecen su formación práctica en áreas críticas como computación paralela, sistemas distribuidos y administración de infraestructuras complejas.

Para la investigación, las nuevas capacidades habilitan proyectos que requieren procesamiento intensivo, simulaciones complejas y análisis de grandes volúmenes de datos, posicionando al GRID como un facilitador de investigación interdisciplinaria tanto a nivel interno como en colaboraciones interinstitucionales. La interfaz Grid App democratiza el acceso a estos recursos, eliminando barreras técnicas que tradicionalmente limitaban su utilización.

\section{Sobre la escalabilidad y sostenibilidad}
\noindent

La arquitectura implementada está diseñada con escalabilidad como principio fundamental. La demostración exitosa de la expansión del universo Grid de un clúster único a dos clústeres independientes valida la capacidad del sistema para incorporar recursos adicionales de manera transparente. Esta característica es crucial para el crecimiento futuro de la infraestructura según evolucionen las necesidades computacionales del grupo.

La sostenibilidad del proyecto se garantiza mediante la documentación exhaustiva de configuraciones, procedimientos y arquitecturas desarrolladas. La generación de un marco metodológico reproducible permite que otras instituciones con contextos similares adapten y repliquen estas soluciones, contribuyendo al fortalecimiento del ecosistema de computación distribuida en el ámbito académico.

\section{Sobre las limitaciones identificadas}
\noindent

Durante el desarrollo del proyecto se identificaron limitaciones que proporcionan direcciones para trabajo futuro. La implementación actual se basa en una infraestructura específica (Raspberry Pi y máquinas virtuales) que, aunque adecuada para los objetivos planteados, podría beneficiarse de la incorporación de recursos más heterogéneos y distribuidos geográficamente.

La interfaz Grid App, aunque funcional, representa una versión inicial que podría expandirse para incluir capacidades avanzadas como monitoreo predictivo, optimización automática de recursos y integración con sistemas de autenticación institucionales. Estas limitaciones no comprometen la funcionalidad actual, sino que definen oportunidades de mejora continua.

\section{Reflexiones sobre la computación distribuida académica}
\noindent

Este proyecto evidencia la importancia estratégica de la computación distribuida en contextos académicos contemporáneos. La democratización del acceso a recursos computacionales de alto rendimiento no solo beneficia directamente a estudiantes e investigadores, sino que contribuye al posicionamiento institucional y fortalecimiento de capacidades científicas y tecnológicas.

La experiencia desarrollada confirma que las tecnologías HTCondor, originalmente diseñadas para centros de investigación de gran escala, pueden adaptarse exitosamente a contextos académicos con recursos limitados, proporcionando beneficios significativos sin requerir inversiones prohibitivas en infraestructura.

\section{Aportes al estado del arte}
\noindent

Esta investigación contribuye al estado del arte en múltiples dimensiones. Metodológicamente, proporciona un marco estructurado que combina caracterización contextual, revisión sistemática de literatura y análisis de decisiones para la selección e implementación de tecnologías de computación distribuida en contextos académicos.

Técnicamente, documenta la implementación exitosa de múltiples universos HTCondor en una infraestructura académica real, proporcionando configuraciones, arquitecturas y procedimientos validados que pueden ser replicados en instituciones similares. La validación práctica mediante casos de prueba reales demuestra la viabilidad y efectividad de las soluciones propuestas.

\section{Conclusión general}
\noindent

El proyecto ha cumplido exitosamente sus objetivos, resultando en una infraestructura HTCondor ampliada que incorpora los universos Grid y Parallel, respaldada por una interfaz de usuario intuitiva y documentación completa. Los resultados trascienden el contexto específico del GRID, proporcionando un modelo replicable para el fortalecimiento de capacidades de computación distribuida en instituciones académicas.

La metodología empleada, los resultados obtenidos y las lecciones aprendidas constituyen una contribución valiosa al conocimiento en el área de computación distribuida aplicada a contextos educativos y de investigación. El proyecto establece una base sólida para el crecimiento futuro de las capacidades computacionales del GRID y proporciona un marco de referencia para iniciativas similares en otras instituciones.

La evidencia generada confirma que la inversión en infraestructuras de computación distribuida en contextos académicos produce retornos significativos en términos de fortalecimiento de capacidades docentes, investigativas y de extensión, justificando plenamente los esfuerzos requeridos para su implementación y sostenimiento a largo plazo.

\ChapterImageStar[cap:trabajos-futuros]{Trabajos Futuros}{./images/fondo.png}\label{cap:trabajos-futuros}
\mbox{}\\

El desarrollo exitoso de este proyecto de ampliación de la infraestructura HTCondor del GRID ha establecido una base sólida que abre múltiples direcciones para trabajo futuro. Las oportunidades identificadas se extienden desde mejoras técnicas específicas hasta iniciativas de mayor alcance que pueden potenciar significativamente el impacto de la infraestructura implementada.

\section{Expansión de la infraestructura distribuida}
\noindent

\subsection{Incorporación de recursos adicionales}
\noindent

La arquitectura Grid implementada ha demostrado su capacidad de escalabilidad al expandirse exitosamente de un clúster único a dos clústeres independientes. Esta característica habilita la incorporación de recursos computacionales adicionales que pueden incluir:

\textbf{Recursos heterogéneos}: Integración de diferentes tipos de hardware, incluyendo servidores de mayor capacidad, estaciones de trabajo de laboratorios y recursos de computación en la nube híbrida. Esta diversificación permitiría optimizar la ejecución de diferentes tipos de cargas de trabajo según sus requerimientos específicos.

\textbf{Federación interinstitucional}: Establecimiento de colaboraciones con otras universidades para crear una federación regional de recursos HTCondor. Esta iniciativa podría proporcionar acceso a capacidades computacionales significativamente mayores y facilitar proyectos de investigación colaborativa entre instituciones.

\subsection{Implementación de universos adicionales}
\noindent

El análisis DAR realizado identificó varios universos HTCondor que, aunque no fueron seleccionados inicialmente, presentan potencial para casos de uso específicos:

\textbf{Universo Container/Docker}: Implementación de capacidades de contenedorización que facilitarían el despliegue de aplicaciones con dependencias complejas y garantizarían reproducibilidad en entornos de investigación científica.

\textbf{Universo VM}: Integración con la infraestructura de virtualización existente (XCP-ng) para proporcionar aislamiento completo y soporte para sistemas operativos alternativos requeridos por aplicaciones específicas.

\section{Mejoras en Grid App}
\noindent

\subsection{Funcionalidades avanzadas de monitoreo}
\noindent

La interfaz actual de Grid App puede expandirse para incluir capacidades de monitoreo más sofisticadas:

\textbf{Dashboard analítico}: Desarrollo de visualizaciones interactivas que muestren métricas de utilización de recursos, tendencias de uso, patrones de ejecución y estadísticas de rendimiento en tiempo real.

\textbf{Alertas y notificaciones}: Implementación de un sistema de notificaciones que informe a los usuarios sobre cambios de estado en sus trabajos, completación de ejecuciones y disponibilidad de resultados.

\textbf{Monitoreo predictivo}: Incorporación de algoritmos de aprendizaje automático para predecir tiempos de ejecución, identificar posibles fallos y optimizar la asignación de recursos.

\subsection{Gestión avanzada de usuarios}
\noindent

\textbf{Autenticación institucional}: Integración con sistemas de autenticación y autorización de la Universidad del Quindío (LDAP/Active Directory) para proporcionar acceso controlado y auditoría de uso.

\textbf{Gestión de cuotas}: Implementación de un sistema de cuotas que permita administrar el uso de recursos por usuario o grupo, garantizando acceso equitativo y previniendo monopolización de recursos.

\textbf{Perfiles de usuario}: Desarrollo de perfiles personalizados que permitan a los usuarios guardar configuraciones frecuentes, mantener historial de trabajos y acceder a plantillas predefinidas.

\section{Optimización y automatización}
\noindent

\subsection{Balanceado inteligente de carga}
\noindent

\textbf{Algoritmos adaptativos}: Desarrollo de algoritmos de balanceado de carga que consideren no solo la disponibilidad actual de recursos, sino también características específicas de los trabajos, historial de rendimiento y predicciones de carga futura.

\textbf{Optimización dinámica}: Implementación de mecanismos que ajusten automáticamente la distribución de trabajos basándose en métricas de rendimiento en tiempo real y aprendizaje de patrones de uso.

\subsection{Automatización operacional}
\noindent

\textbf{Despliegue automatizado}: Desarrollo de scripts y herramientas de automatización (Ansible, Terraform) que faciliten la configuración y despliegue de nuevos nodos HTCondor, reduciendo la complejidad operacional.

\textbf{Mantenimiento predictivo}: Implementación de sistemas de monitoreo que identifiquen proactivamente problemas potenciales en la infraestructura y ejecuten acciones correctivas automáticas.

\section{Integración con ecosistemas científicos}
\noindent

\subsection{Workflows científicos}
\noindent

\textbf{Integración con sistemas de workflows}: Desarrollo de conectores para sistemas populares de gestión de workflows científicos como Nextflow, Snakemake o Galaxy, facilitando la ejecución de pipelines complejos de análisis de datos.

\textbf{Soporte para Jupyter}: Integración que permita ejecutar notebooks Jupyter de manera distribuida, facilitando el análisis de datos interactivo con acceso a recursos computacionales masivos.

\subsection{Repositorios de datos}
\noindent

\textbf{Gestión de datasets}: Implementación de capacidades de gestión de conjuntos de datos que faciliten el almacenamiento, versionado y distribución de datos de investigación entre diferentes trabajos y usuarios.

\textbf{Integración con almacenamiento distribuido}: Conexión con sistemas de almacenamiento distribuido como Ceph o GlusterFS para proporcionar acceso eficiente a grandes volúmenes de datos.

\section{Investigación y desarrollo}
\noindent

\subsection{Estudios de rendimiento}
\noindent

\textbf{Benchmarking comparativo}: Realización de estudios sistemáticos de rendimiento que comparen la eficiencia de diferentes universos HTCondor para tipos específicos de cargas de trabajo científicas.

\textbf{Optimización de algoritmos}: Investigación en algoritmos de planificación y distribución de trabajos específicamente optimizados para infraestructuras académicas con recursos heterogéneos.

\subsection{Nuevas aplicaciones}
\noindent

\textbf{Machine Learning distribuido}: Exploración de la aplicación de la infraestructura para entrenamiento distribuido de modelos de aprendizaje automático, aprovechando las capacidades paralelas implementadas.

\textbf{Simulaciones científicas}: Desarrollo de casos de uso específicos para simulaciones en áreas como física, química, biología y ingeniería que aprovechen las capacidades distribuidas disponibles.

\section{Transferencia de conocimiento}
\noindent

\subsection{Documentación y capacitación}
\noindent

\textbf{Recursos educativos}: Desarrollo de materiales didácticos, tutoriales interactivos y casos de estudio que faciliten la adopción de la infraestructura por parte de estudiantes e investigadores.

\textbf{Programa de capacitación}: Establecimiento de un programa estructurado de capacitación que incluya talleres, seminarios y certificaciones en computación distribuida usando HTCondor.

\subsection{Colaboraciones interinstitucionales}
\noindent

\textbf{Red de conocimiento}: Formación de una red de instituciones académicas que compartan experiencias, buenas prácticas y recursos relacionados con HTCondor y computación distribuida.

\textbf{Proyectos colaborativos}: Iniciación de proyectos de investigación colaborativa que aprovechen las capacidades distribuidas para abordar problemas científicos de gran escala.

\section{Sustentabilidad y gobierno}
\noindent

\subsection{Modelo de sostenibilidad}
\noindent

\textbf{Plan de sostenibilidad financiera}: Desarrollo de un modelo que garantice el financiamiento continuo para operación, mantenimiento y crecimiento de la infraestructura a largo plazo.

\textbf{Políticas de uso}: Establecimiento de políticas claras que regulen el uso de recursos, prioridades de acceso y procedimientos para resolución de conflictos.

\subsection{Mejora continua}
\noindent

\textbf{Ciclo de evaluación}: Implementación de un proceso sistemático de evaluación y mejora que considere retroalimentación de usuarios, métricas de rendimiento y evolución tecnológica.

\textbf{Adaptación tecnológica}: Establecimiento de mecanismos para evaluar e incorporar nuevas tecnologías emergentes en computación distribuida que puedan beneficiar a la infraestructura.

\section{Impacto a largo plazo}
\noindent

Las direcciones de trabajo futuro identificadas tienen el potencial de transformar significativamente el panorama de computación científica en la Universidad del Quindío y establecer un modelo replicable para otras instituciones académicas. La implementación gradual de estas iniciativas puede posicionar al GRID como un centro de referencia regional en computación distribuida, fortaleciendo su capacidad para atraer proyectos de investigación, establecer colaboraciones interinstitucionales y contribuir al desarrollo científico y tecnológico nacional.

El éxito en la implementación de estos trabajos futuros dependerá de la capacidad para mantener un balance entre innovación tecnológica y sostenibilidad operacional, garantizando que las mejoras propuestas se alineen con las necesidades reales de la comunidad académica y contribuyan efectivamente al cumplimiento de los objetivos misionales de docencia, investigación y extensión de la Universidad del Quindío.


\input{capitulos/cumplimientoObjetivos.tex}



% Referencias - con imagen de fondo específica
\cleardoublepage\thispagestyle{empty}
\refstepcounter{chapter}
\phantomsection\addcontentsline{toc}{chapter}{Referencias}\label{cap:referencias}

% Imagen de fondo específica para referencias
\AddToShipoutPicture*{%
  \put(0,0){\includegraphics[width=\paperwidth,height=\paperheight]{./images/referencias.png}}%
}

% Título posicionado con coordenadas absolutas (en píxeles desde esquina superior izquierda)
\begin{tikzpicture}[remember picture,overlay]
  \node[anchor=north west] at ([xshift=150pt,yshift=-370pt]current page.north west)
    {\Large\bfseries\textcolor{black}{Referencias}};
\end{tikzpicture}
\vspace{2cm}

% Bibliografía sin título automático
\begingroup
  \renewcommand{\bibname}{}
  \pagestyle{fancy} % Fuerza el estilo fancy para todas las páginas de bibliografía
  \bibliographystyle{apalike}
  \bibliography{bibliografia}
\endgroup

\appendix

\chapter{Fichas técnicas y búsqueda en bases de datos}\label{apendice:fichas-y-busquedas}

\section{Ficha técnica del recurso tecnológico}
\begin{figure}[H]
    \centering
    \includegraphics[width=\textwidth,height=0.85\textheight,keepaspectratio]{apendices/caracterizacionInfraestructura.png}
    \caption{Ficha técnica del recurso tecnológico}\label{fig:tabla-ficha-tecnica}
\end{figure}
\FloatBarrier\section*{Ficha técnica de servicios}

\begin{figure}[H]
    \centering
    \includegraphics[width=\textwidth,height=0.85\textheight,keepaspectratio]{apendices/caracterizacionServicios.png}
    \caption{Ficha técnica de servicios}\label{fig:tabla-ficha-servicios}
\end{figure}
\FloatBarrier\clearpage

\chapter{Búsquedas en bases de datos}

\section{Cadenas de búsqueda}\label{sec:cadenas-busqueda}


\begin{tcolorbox}[
  colback=gray!5, 
  colframe=black!60, 
  title=Cadena de búsqueda en ACM para educación, 
  fonttitle=\bfseries, 
  sharp corners=south
]
\scriptsize % o \footnotesize, \tiny según lo pequeño que lo quieras
\begin{verbatim}
(Title:("Container-based virtualization" OR "Application virtualization" OR "Docker" OR 
"Lightweight Virtualization") AND Title:("Education" OR "Education System" 
OR "Education Development" OR "Higher Education") ) 

OR

(Abstract:("Container-based virtualization" OR "Application virtualization" OR "Docker"
 OR "Lightweight Virtualization") AND Abstract:("Education" OR "Education System" 
 OR "Education Development" OR "Higher Education") )

OR

(Keyword:("Container-based virtualization" OR "Application virtualization" OR "Docker" OR 
"Lightweight Virtualization")
AND Keyword:("Education" OR "Education System" OR "Education Development" 
OR "Higher Education"))
\end{verbatim}
\end{tcolorbox}

\begin{tcolorbox}[
  colback=gray!5, 
  colframe=black!60, 
  title=Cadena de búsqueda en ACM para investigación, 
  fonttitle=\bfseries, 
  sharp corners=south
]
\scriptsize % puedes usar \tiny para hacerlo aún más pequeño
\begin{verbatim}
(Title:("Container-based virtualization" OR "Application virtualization" OR "Docker" OR 
"Lightweight Virtualization") AND Title:("Research" OR "Research Group" OR 
"Research Proposal"))

OR

(Abstract:("Container-based virtualization" OR "Application virtualization" OR "Docker" OR 
"Lightweight Virtualization") AND Abstract:("Research" OR "Research Group" OR 
"Research Proposal"))

OR

(Keyword:("Container-based virtualization" OR "Application virtualization" OR "Docker" OR 
"Lightweight Virtualization") AND Keyword:("Research" OR "Research Group" OR 
"Research Proposal"))
\end{verbatim}
\end{tcolorbox}

\begin{tcolorbox}[
  colback=gray!5, 
  colframe=black!60, 
  title=Cadena de búsqueda en ACM para extensión, 
  fonttitle=\bfseries, 
  sharp corners=south
]
\scriptsize % puedes usar \tiny para hacerlo aún más pequeño
\begin{verbatim}
(Title:("Container-based virtualization" OR "Application virtualization" OR "Docker" OR 
"Lightweight Virtualization") AND Title:("Industry" OR “IT Services” OR 
“Technology Infrastructure” OR “Cloud Computing”) ) 

OR

(Abstract:("Container-based virtualization" OR "Application virtualization" OR "Docker" 
OR "Lightweight Virtualization") AND Abstract:("Industry" OR “IT Services” OR 
“Technology Infrastructure” OR “Cloud Computing”) )

OR

(Keyword:("Container-based virtualization" OR "Application virtualization" OR "Docker" 
OR "Lightweight Virtualization")
AND Keyword:("Industry" OR “IT Services” OR “Technology Infrastructure” 
OR “Cloud Computing”))

\end{verbatim}
\end{tcolorbox}

\begin{tcolorbox}[
  colback=gray!5, 
  colframe=black!60, 
  title=Cadena de búsqueda en IEE para educación, 
  fonttitle=\bfseries, 
  sharp corners=south
]
\scriptsize % puedes usar \tiny para hacerlo aún más pequeño
\begin{verbatim}
(("Abstract":"Container-based virtualization" OR "Abstract":"Application virtualization" 
OR "Abstract":"Docker" OR "Abstract":"Lightweight Virtualization") AND ("Abstract":"Education" 
OR "Abstract":"Education System" OR "Abstract":"Education Development”  OR 
"Abstract":"Higher Education”)) 

OR (("Publication Title":"Container-based virtualization" OR "Publication 
Title":"Application virtualization" 
OR "Publication Title":"Docker" OR "Publication Title":"Lightweight Virtualization") 
AND ("Publication Title":"Education" 
OR "Publication Title":"Education System" OR "Publication Title":"Education Development”  
OR "Publication Title":"Higher Education” ))

OR (("Author Keywords":"Container-based virtualization" OR 
"Author Keywords":"Application virtualization" OR 
"Author Keywords":"Docker" OR "Author Keywords":"Lightweight Virtualization") AND 
("Author Keywords":"Education" 
OR "Author Keywords":"Education System" OR "Author Keywords":"Education Development”  
OR "Author Keywords":"Higher Education”))
\end{verbatim}
\end{tcolorbox}


\begin{tcolorbox}[
  colback=gray!5, 
  colframe=black!60, 
  title=Cadena de búsqueda en IEE para investigación, 
  fonttitle=\bfseries, 
  sharp corners=south
]
\scriptsize % puedes usar \tiny para hacerlo aún más pequeño
\begin{verbatim}
(("Abstract":"Container-based virtualization" OR "Abstract":"Application virtualization" 
OR "Abstract":"Docker" OR "Abstract":"Lightweight Virtualization") AND 
("Abstract":"Research Group" OR "Abstract":"Research Proposal")) 

OR (("Publication Title":"Container-based virtualization" OR 
"Publication Title":"Application virtualization" OR "Publication Title":"Docker" OR 
"Publication Title":"Lightweight Virtualization") AND 
("Publication Title":"Research Group" OR "Publication Title":"Research Proposal" ))

OR (("Author Keywords":"Container-based virtualization" OR 
"Author Keywords":"Application virtualization" OR "Author Keywords":"Docker" OR 
"Author Keywords":"Lightweight Virtualization") AND 
("Author Keywords":"Research Group" OR "Author Keywords":"Research Proposal"))
\end{verbatim}
\end{tcolorbox}

\begin{tcolorbox}[
  colback=gray!5, 
  colframe=black!60, 
  title=Cadena de búsqueda en IEE para extensión, 
  fonttitle=\bfseries, 
  sharp corners=south
]
\scriptsize % puedes usar \tiny para hacerlo aún más pequeño
\begin{verbatim}
(("Abstract":"Container-based virtualization" OR "Abstract":"Application virtualization" 
OR "Abstract":"Docker" OR "Abstract":"Lightweight Virtualization") AND 
("Abstract":"Industry" OR "Abstract":"IT Services" OR 
"Abstract":"Technology Infrastructure" OR "Abstract":"Cloud Computing")) 

OR (("Publication Title":"Container-based virtualization" OR 
"Publication Title":"Application virtualization" 
OR "Publication Title":"Docker" OR "Publication Title":"Lightweight Virtualization") AND 
("Publication Title":"Industry" OR "Publication Title":"IT Services" OR 
"Publication Title":"Technology Infrastructure" OR "Publication Title":"Cloud Computing"))

OR (("Author Keywords":"Container-based virtualization" OR 
"Author Keywords":"Application virtualization" OR "Author Keywords":"Docker" OR 
"Author Keywords":"Lightweight Virtualization") AND ("Author Keywords":"Industry" OR 
"Author Keywords":"IT Services" OR "Author Keywords":"Technology Infrastructure" OR 
"Author Keywords":"Cloud Computing"))
\end{verbatim}
\end{tcolorbox}

\begin{tcolorbox}[
  colback=gray!5, 
  colframe=black!60, 
  title=Cadena de búsqueda en Springer para educación, 
  fonttitle=\bfseries, 
  sharp corners=south
]
\scriptsize % puedes usar \tiny para hacerlo aún más pequeño
\begin{verbatim}
(title:("Container-based virtualization" OR "Application virtualization" OR 
"Docker" OR "Lightweight Virtualization") AND title:("Education" OR 
"Education System" OR "Education Development" OR "Higher Education"))

OR

(abstract:("Container-based virtualization" OR "Application virtualization" OR 
"Docker" OR "Lightweight Virtualization") AND abstract:("Education" OR 
"Education System" OR "Education Development" OR "Higher Education"))

OR 

(keyword:("Container-based virtualization" OR "Application virtualization" OR 
"Docker" OR "Lightweight Virtualization") AND keyword:("Education" OR 
"Education System" OR "Education Development" OR "Higher Education"))

\end{verbatim}
\end{tcolorbox}

\begin{tcolorbox}[
  colback=gray!5, 
  colframe=black!60, 
  title=Cadena de búsqueda en Springer para investigación, 
  fonttitle=\bfseries, 
  sharp corners=south
]
\scriptsize % puedes usar \tiny para hacerlo aún más pequeño
\begin{verbatim}
(title:("Container-based virtualization" OR "Application virtualization" OR 
"Docker" OR "Lightweight Virtualization") AND title:("research" OR 
"Research Group" OR "Research Proposal"))

OR

(abstract:("Container-based virtualization" OR "Application virtualization" 
OR "Docker" OR "Lightweight Virtualization") AND abstract:("research" 
OR "Research Group" OR "Research Proposal"))

OR 

(keyword:("Container-based virtualization" OR "Application virtualization"
 OR "Docker" OR "Lightweight Virtualization") AND keyword:("research" OR 
 "Research Group" OR "Research Proposal"))

\end{verbatim}
\end{tcolorbox}

\begin{tcolorbox}[
  colback=gray!5, 
  colframe=black!60, 
  title=Cadena de búsqueda en Springer para extensión, 
  fonttitle=\bfseries, 
  sharp corners=south
]
\scriptsize % puedes usar \tiny para hacerlo aún más pequeño
\begin{verbatim}
(title:("Container-based virtualization" OR "Application virtualization"
 OR "Docker" OR "Lightweight Virtualization") AND title:("Industry" OR 
 “IT Services” OR “Technology Infrastructure” OR “Cloud Computing”))

OR

(abstract:("Container-based virtualization" OR "Application virtualization" 
OR "Docker" OR "Lightweight Virtualization") AND abstract:("Industry" OR 
“IT Services” OR “Technology Infrastructure” OR “Cloud Computing”))

OR 

(keyword:("Container-based virtualization" OR "Application virtualization"
 OR "Docker" OR "Lightweight Virtualization") AND keyword:("Industry" 
 OR “IT Services” OR “Technology Infrastructure” OR “Cloud Computing”))

\end{verbatim}
\end{tcolorbox}

\begin{tcolorbox}[
  colback=gray!5, 
  colframe=black!60, 
  title=Cadena de búsqueda en Science Direct para educación, 
  fonttitle=\bfseries, 
  sharp corners=south
]
\scriptsize % puedes usar \tiny para hacerlo aún más pequeño
\begin{verbatim}
("Container-based virtualization" OR "Application virtualization" 
OR "Docker" OR "Lightweight Virtualization")  AND ("Education" OR 
"Education System" OR "Education Development" OR "Higher Education")
\end{verbatim}
\end{tcolorbox}


\begin{tcolorbox}[
  colback=gray!5, 
  colframe=black!60, 
  title=Cadena de búsqueda en Science Direct para investigación, 
  fonttitle=\bfseries, 
  sharp corners=south
]
\scriptsize % puedes usar \tiny para hacerlo aún más pequeño
\begin{verbatim}
("Container-based virtualization" OR "Application virtualization" OR 
"Docker" OR "Lightweight Virtualization")  AND ("Research" OR 
"Research Group" OR "Research Proposal")
\end{verbatim}
\end{tcolorbox}

\begin{tcolorbox}[
  colback=gray!5, 
  colframe=black!60, 
  title=Cadena de búsqueda en Science Direct para extensión, 
  fonttitle=\bfseries, 
  sharp corners=south
]
\scriptsize % puedes usar \tiny para hacerlo aún más pequeño
\begin{verbatim}
("Container-based virtualization" OR "Application virtualization" OR "Docker" OR 
"Lightweight Virtualization")  AND 
(“Industry” OR "IT Services" OR "Technology Infrastructure" OR "Cloud Computing")
\end{verbatim}
\end{tcolorbox}

\begin{tcolorbox}[
  colback=gray!5, 
  colframe=black!60, 
  title=Cadena de búsqueda en Taylor \& Francis para educación, 
  fonttitle=\bfseries, 
  sharp corners=south
]
\scriptsize % puedes usar \tiny para hacerlo aún más pequeño
\begin{verbatim}
("Application virtualization" OR "Docker" OR "Lightweight Virtualization" OR "Docker Container")   
AND   
("Education System" OR "Education Sector" OR "Education Development" OR "Higher Education")
\end{verbatim}
\end{tcolorbox}

\begin{tcolorbox}[
  colback=gray!5, 
  colframe=black!60, 
  title=Cadena de búsqueda en Taylor \& Francis para investigación, 
  fonttitle=\bfseries, 
  sharp corners=south
]
\scriptsize % puedes usar \tiny para hacerlo aún más pequeño
\begin{verbatim}
("Application virtualization" OR "Docker" OR "Lightweight Virtualization" OR "Docker Container")
AND   
("Specific Research Areas" OR "Research Group" OR "Research Proposal" OR "Research and Development")
\end{verbatim}
\end{tcolorbox}

\begin{tcolorbox}[
  colback=gray!5, 
  colframe=black!60, 
  title=Cadena de búsqueda en Taylor \& Francis para extensión, 
  fonttitle=\bfseries, 
  sharp corners=south
]
\scriptsize % puedes usar \tiny para hacerlo aún más pequeño
\begin{verbatim}
("Application virtualization" OR "Docker" OR "Lightweight Virtualization" OR "Docker Container")  
AND 
(“Industry” OR "IT Services" OR "Technology Infrastructure" OR "Cloud Computing")
\end{verbatim}
\end{tcolorbox}


\section{Búsqueda de artículos sin criterios de inclusión/exclusión}

\begin{figure}[H]
    \centering
    \includegraphics[width=\textwidth,keepaspectratio]{apendices/BD/sin-criterios/ACM-ed.png}
    \caption{Búsqueda de artículos de educación en ACM sin criterios de inclusión/exclusión \\
    Fecha de acceso: 12/03/25 9:13 pm
    }\label{fig:busqueda1}
\end{figure}
\FloatBarrier\begin{figure}[H]
    \centering
    \includegraphics[width=\textwidth,keepaspectratio]{apendices/BD/sin-criterios/ACM-inv.png}
    \caption{Búsqueda de artículos de investigación en ACM sin criterios de inclusión/exclusión \\
    Fecha de acceso: 12/03/25 8:23 pm
    }\label{fig:busqueda2}
\end{figure}
\FloatBarrier\begin{figure}[H]
    \centering
    \includegraphics[width=\textwidth,keepaspectratio]{apendices/BD/sin-criterios/ACM-ind.png}
    \caption{Búsqueda de artículos de extensión en ACM sin criterios de inclusión/exclusión \\
    Fecha de acceso: 12/03/25 9:20 pm
    }\label{fig:busqueda3}
\end{figure}
\FloatBarrier\begin{figure}[H]
    \centering
    \includegraphics[width=\textwidth,keepaspectratio]{apendices/BD/sin-criterios/IEEE-ed.png}
    \caption{Búsqueda de artículos de educación en IEEE sin criterios de inclusión/exclusión
    Fecha de acceso: 7/03/25 8:50 pm
    }\label{fig:busqueda4}
\end{figure}
\FloatBarrier\begin{figure}[H]
    \centering
    \includegraphics[width=\textwidth,keepaspectratio]{apendices/BD/sin-criterios/IEEE-inv.png}
    \caption{Búsqueda de artículos de investigación en IEEE sin criterios de inclusión/exclusión
    Fecha de acceso: 7/03/25 8:46 pm
    }\label{fig:busqueda5}
\end{figure}
\FloatBarrier\begin{figure}[H]
    \centering
    \includegraphics[width=\textwidth,keepaspectratio]{apendices/BD/sin-criterios/IEEE-ind.png}
    \caption{Búsqueda de artículos de extensión en IEEE sin criterios de inclusión/exclusión
    Fecha de acceso: 12/03/25 8:54 pm
    }\label{fig:busqueda6}
\end{figure}
\FloatBarrier\begin{figure}[H]
    \centering
    \includegraphics[width=\textwidth,keepaspectratio]{apendices/BD/sin-criterios/Springer-ed.png}
    \caption{Búsqueda de artículos de educación en Springer sin criterios de inclusión/exclusión
    Fecha de acceso: 12/03/25 9:58 pm
    }\label{fig:busqueda7}
\end{figure}
\FloatBarrier\begin{figure}[H]
    \centering
    \includegraphics[width=\textwidth,keepaspectratio]{apendices/BD/sin-criterios/Springer-inv.png}
    \caption{Búsqueda de artículos de investigación en Springer sin criterios de inclusión/exclusión
    Fecha de acceso: 13/03/25 12:40 pm
    }\label{fig:busqueda8}
\end{figure}
\FloatBarrier\begin{figure}[H]
    \centering
    \includegraphics[width=\textwidth,keepaspectratio]{apendices/BD/sin-criterios/Springer-ind.png}
    \caption{Búsqueda de artículos de extensión en Springer sin criterios de inclusión/exclusión
    Fecha de acceso: 13/03/25 12:48 pm}\label{fig:busqueda9}
\end{figure}
\FloatBarrier\begin{figure}[H]
    \centering
    \includegraphics[width=\textwidth,keepaspectratio]{apendices/BD/sin-criterios/SD-ed.png}
    \caption{Búsqueda de artículos de educación en Science Direct sin criterios de inclusión/exclusión
    Fecha de acceso: 13/03/25 1:03 pm}\label{fig:busqueda10}
\end{figure}
\FloatBarrier\begin{figure}[H]
    \centering
    \includegraphics[width=\textwidth,keepaspectratio]{apendices/BD/sin-criterios/SD-inv.png}
    \caption{Búsqueda de artículos de investigación en Science Direct sin criterios de inclusión/exclusión
    Fecha de acceso: 13/03/25 1:43 pm}\label{fig:busqueda11}
\end{figure}
\FloatBarrier\begin{figure}[H]
    \centering
    \includegraphics[width=\textwidth,keepaspectratio]{apendices/BD/sin-criterios/SD-ind.png}
    \caption{Búsqueda de artículos de extensión en Science Direct sin criterios de inclusión/exclusión
    Fecha de acceso: 13/03/25 1:48 pm}\label{fig:busqueda12}
\end{figure}
\FloatBarrier\begin{figure}[H]
    \centering
    \includegraphics[width=\textwidth,keepaspectratio]{apendices/BD/sin-criterios/TF-ed.png}
    \caption{Búsqueda de artículos de educación en Taylor \& Francis sin criterios de inclusión/exclusión
    Fecha de acceso: 16/03/25 9:21 pm}\label{fig:busqueda13}
\end{figure}
\FloatBarrier\begin{figure}[H]
    \centering
    \includegraphics[width=\textwidth,keepaspectratio]{apendices/BD/sin-criterios/TF-inv.png}
    \caption{Búsqueda de artículos de investigación en Taylor \& Francis sin criterios de inclusión/exclusión
    Fecha de acceso: 16/03/25 9:31 pm
    }\label{fig:busqueda14}
\end{figure}
\FloatBarrier\begin{figure}[H]
    \centering
    \includegraphics[width=\textwidth,keepaspectratio]{apendices/BD/sin-criterios/TF-ind.png}
    \caption{Búsqueda de artículos de extensión en Taylor \& Francis sin criterios de inclusión/exclusión
    Fecha de acceso: 16/03/25 9:34 pm
    }\label{fig:busqueda15}
\end{figure}
\FloatBarrier\section{Búsqueda de artículos usando criterios de inclusión/exclusión}
\begin{figure}[H]
    \centering
    \includegraphics[width=\textwidth,keepaspectratio]{apendices/BD/criterios/ACM-ed.png}
    \caption{Búsqueda de artículos de educación en ACM con criterios de inclusión/exclusión
    Fecha de acceso: 20/03/25 1:15 pm
    }\label{fig:busqueda16}
\end{figure}
\FloatBarrier\begin{figure}[H]
    \centering
    \includegraphics[width=\textwidth,keepaspectratio]{apendices/BD/criterios/ACM-inv.png}
    \caption{Búsqueda de artículos de investigación en ACM con criterios de inclusión/exclusión
    Fecha de acceso: 20/03/25 1:19 pm
    }\label{fig:busqueda17}
\end{figure}
\FloatBarrier\begin{figure}[H]
    \centering
    \includegraphics[width=\textwidth,keepaspectratio]{apendices/BD/criterios/ACM-ind.png}
    \caption{Búsqueda de artículos de extensión en ACM con criterios de inclusión/exclusión
    Fecha de acceso: 20/03/25 1:20 pm
    }\label{fig:busqueda18}
\end{figure}
\FloatBarrier\begin{figure}[H]
    \centering
    \includegraphics[width=\textwidth,keepaspectratio]{apendices/BD/criterios/IEEE-ed.png}
    \caption{Búsqueda de artículos de educación en IEEE con criterios de inclusión/exclusión
    Fecha de acceso: 20/03/25 1:27 pm
    }\label{fig:busqueda19}
\end{figure}
\FloatBarrier\begin{figure}[H]
    \centering
    \includegraphics[width=\textwidth,keepaspectratio]{apendices/BD/criterios/IEEE-ind.png}
    \caption{Búsqueda de artículos de extensión en IEEE con criterios de inclusión/exclusión
    Fecha de acceso: 20/03/25 1:37 pm
    }\label{fig:busqueda21}
\end{figure}
\FloatBarrier\begin{figure}[H]
    \centering
    \includegraphics[width=\textwidth,keepaspectratio]{apendices/BD/criterios/Springer-ed.png}
    \caption{Búsqueda de artículos de educación en Springer con criterios de inclusión/exclusión
    Fecha de acceso: 20/03/25 2:29 pm
    }\label{fig:busqueda22}
\end{figure}
\FloatBarrier\begin{figure}[H]
    \centering
    \includegraphics[width=\textwidth,keepaspectratio]{apendices/BD/criterios/Springer-inv.png}
    \caption{Búsqueda de artículos de investigación en Springer con criterios de inclusión/exclusión
    Fecha de acceso: 16/03/25 11:05 am
    }\label{fig:busqueda23}
\end{figure}
\FloatBarrier\begin{figure}[H]
    \centering
    \includegraphics[width=\textwidth,keepaspectratio]{apendices/BD/criterios/Springer-ind.png}
    \caption{Búsqueda de artículos de extensión en Springer con criterios de inclusión/exclusión
    Fecha de acceso: 16/03/25 11:07 am
    }\label{fig:busqueda24}
\end{figure}
\FloatBarrier\begin{figure}[H]
    \centering
    \includegraphics[width=\textwidth,keepaspectratio]{apendices/BD/criterios/SD-ed.png}
    \caption{Búsqueda de artículos de educación en Science Direct con criterios de inclusión/exclusión
    Fecha de acceso: 20/03/25 2:59 am
    }\label{fig:busqueda25}
\end{figure}
\FloatBarrier\begin{figure}[H]
    \centering
    \includegraphics[width=\textwidth,keepaspectratio]{apendices/BD/criterios/SD-inv.png}
    \caption{Búsqueda de artículos de investigación en Science Direct con criterios de inclusión/exclusión
    Fecha de acceso: 20/03/25 3:01 am
    }\label{fig:busqueda26}
\end{figure}
\FloatBarrier\begin{figure}[H]
    \centering
    \includegraphics[width=\textwidth,keepaspectratio]{apendices/BD/criterios/SD-ind.png}
    \caption{Búsqueda de artículos de extensión en Science Direct con criterios de inclusión/exclusión
    Fecha de acceso: 20/03/25 3:07 am
    }\label{fig:busqueda27}
\end{figure}
\FloatBarrier\begin{figure}[H]
    \centering
    \includegraphics[width=\textwidth,keepaspectratio]{apendices/BD/criterios/TF-ed.png}
    \caption{Búsqueda de artículos de educación en Taylor \& Francis con criterios de inclusión/exclusión
    Fecha de acceso: 20/03/25 4:46 am
    }\label{fig:busqueda28}
\end{figure}
\FloatBarrier\begin{figure}[H]
    \centering
    \includegraphics[width=\textwidth,keepaspectratio]{apendices/BD/criterios/TF-inv.png}
    \caption{Búsqueda de artículos de investigación en Taylor \& Francis con criterios de inclusión/exclusión
    Fecha de acceso: 20/03/25 4:49 am
    }\label{fig:busqueda29}
\end{figure}
\FloatBarrier\begin{figure}[H]
    \centering
    \includegraphics[width=\textwidth,keepaspectratio]{apendices/BD/criterios/TF-ind.png}
    \caption{Búsqueda de artículos de extensión en Taylor \& Francis con criterios de inclusión/exclusión
    Fecha de acceso: 20/03/25 4:50 am
    }\label{fig:busqueda30}
\end{figure}
\FloatBarrier\chapter{Plantilla del análisis DAR}
\section{Plantilla análisis DAR}

\begin{figure}[H]
    \centering
    \includegraphics[width=\textwidth,height=0.85\textheight,keepaspectratio]{apendices/plantilla-DAR.png}
    \caption{Plantilla del análisis DAR}\label{fig:tabla-plantilla-dar}
\end{figure}
\FloatBarrier\chapter{Eventos de difusión}
\section{Seminario GRID 2024-II}


\section{Seminario GRID 2025-I}

\section{ACOFI 2025}

En esta sección se presenta la ponencia presentada en el congreso ACOFI 2025 como resultado de la investigación desarrollada en este trabajo.

\subsection{Páginas de la ponencia ACOFI}

% Página 1
\begin{figure}[H]
    \centering
    \begin{tcolorbox}[
        colback=white,
        colframe=gray!50,
        boxrule=1pt,
        arc=2pt,
        boxsep=5pt,
        left=3pt,
        right=3pt,
        top=3pt,
        bottom=3pt,
        drop shadow
    ]
        \includegraphics[width=0.95\textwidth,keepaspectratio]{apendices/ACOFI/pagina_1.png}
    \end{tcolorbox}
    \caption{Ponencia ACOFI --- Página 1}\label{fig:acofi-pagina-1}
\end{figure}
\FloatBarrier% Página 2
\begin{figure}[H]
    \centering
    \begin{tcolorbox}[
        colback=white,
        colframe=gray!50,
        boxrule=1pt,
        arc=2pt,
        boxsep=5pt,
        left=3pt,
        right=3pt,
        top=3pt,
        bottom=3pt,
        drop shadow
    ]
        \includegraphics[width=0.95\textwidth,keepaspectratio]{apendices/ACOFI/pagina_2.png}
    \end{tcolorbox}
    \caption{Ponencia ACOFI --- Página 2}\label{fig:acofi-pagina-2}
\end{figure}
\FloatBarrier% Página 3
\begin{figure}[H]
    \centering
    \begin{tcolorbox}[
        colback=white,
        colframe=gray!50,
        boxrule=1pt,
        arc=2pt,
        boxsep=5pt,
        left=3pt,
        right=3pt,
        top=3pt,
        bottom=3pt,
        drop shadow
    ]
        \includegraphics[width=0.95\textwidth,keepaspectratio]{apendices/ACOFI/pagina_3.png}
    \end{tcolorbox}
    \caption{Ponencia ACOFI --- Página 3}\label{fig:acofi-pagina-3}
\end{figure}
\FloatBarrier% Página 4
\begin{figure}[H]
    \centering
    \begin{tcolorbox}[
        colback=white,
        colframe=gray!50,
        boxrule=1pt,
        arc=2pt,
        boxsep=5pt,
        left=3pt,
        right=3pt,
        top=3pt,
        bottom=3pt,
        drop shadow
    ]
        \includegraphics[width=0.95\textwidth,keepaspectratio]{apendices/ACOFI/pagina_4.png}
    \end{tcolorbox}
    \caption{Ponencia ACOFI --- Página 4}\label{fig:acofi-pagina-4}
\end{figure}
\FloatBarrier% Página 5
\begin{figure}[H]
    \centering
    \begin{tcolorbox}[
        colback=white,
        colframe=gray!50,
        boxrule=1pt,
        arc=2pt,
        boxsep=5pt,
        left=3pt,
        right=3pt,
        top=3pt,
        bottom=3pt,
        drop shadow
    ]
        \includegraphics[width=0.95\textwidth,keepaspectratio]{apendices/ACOFI/pagina_5.png}
    \end{tcolorbox}
    \caption{Ponencia ACOFI --- Página 5}\label{fig:acofi-pagina-5}
\end{figure}
\FloatBarrier% Página 6
\begin{figure}[H]
    \centering
    \begin{tcolorbox}[
        colback=white,
        colframe=gray!50,
        boxrule=1pt,
        arc=2pt,
        boxsep=5pt,
        left=3pt,
        right=3pt,
        top=3pt,
        bottom=3pt,
        drop shadow
    ]
        \includegraphics[width=0.95\textwidth,keepaspectratio]{apendices/ACOFI/pagina_6.png}
    \end{tcolorbox}
    \caption{Ponencia ACOFI --- Página 6}\label{fig:acofi-pagina-6}
\end{figure}
\FloatBarrier% Página 7
\begin{figure}[H]
    \centering
    \begin{tcolorbox}[
        colback=white,
        colframe=gray!50,
        boxrule=1pt,
        arc=2pt,
        boxsep=5pt,
        left=3pt,
        right=3pt,
        top=3pt,
        bottom=3pt,
        drop shadow
    ]
        \includegraphics[width=0.95\textwidth,keepaspectratio]{apendices/ACOFI/pagina_7.png}
    \end{tcolorbox}
    \caption{Ponencia ACOFI --- Página 7}\label{fig:acofi-pagina-7}
\end{figure}
\FloatBarrier% Página 8
\begin{figure}[H]
    \centering
    \begin{tcolorbox}[
        colback=white,
        colframe=gray!50,
        boxrule=1pt,
        arc=2pt,
        boxsep=5pt,
        left=3pt,
        right=3pt,
        top=3pt,
        bottom=3pt,
        drop shadow
    ]
        \includegraphics[width=0.95\textwidth,keepaspectratio]{apendices/ACOFI/pagina_8.png}
    \end{tcolorbox}
    \caption{Ponencia ACOFI --- Página 8}\label{fig:acofi-pagina-8}
\end{figure}
\FloatBarrier% Página 9
\begin{figure}[H]
    \centering
    \begin{tcolorbox}[
        colback=white,
        colframe=gray!50,
        boxrule=1pt,
        arc=2pt,
        boxsep=5pt,
        left=3pt,
        right=3pt,
        top=3pt,
        bottom=3pt,
        drop shadow
    ]
        \includegraphics[width=0.95\textwidth,keepaspectratio]{apendices/ACOFI/pagina_9.png}
    \end{tcolorbox}
    \caption{Ponencia ACOFI --- Página 9}\label{fig:acofi-pagina-9}
\end{figure}
\FloatBarrier% Página 10
\begin{figure}[H]
    \centering
    \begin{tcolorbox}[
        colback=white,
        colframe=gray!50,
        boxrule=1pt,
        arc=2pt,
        boxsep=5pt,
        left=3pt,
        right=3pt,
        top=3pt,
        bottom=3pt,
        drop shadow
    ]
        \includegraphics[width=0.95\textwidth,keepaspectratio]{apendices/ACOFI/pagina_10.png}
    \end{tcolorbox}
    \caption{Ponencia ACOFI --- Página 10}\label{fig:acofi-pagina-10}
\end{figure}
\FloatBarrier\section{CEIFI 2025}

\section{Artículo de revista}

En esta sección se presenta el artículo de revista publicado en Journal of Information Systems and Applications (JISA) como resultado de la investigación desarrollada en este trabajo.

\subsection{Páginas del artículo JISA}

% Página 1
\begin{figure}[H]
    \centering
    \begin{tcolorbox}[
        colback=white,
        colframe=gray!50,
        boxrule=1pt,
        arc=2pt,
        boxsep=5pt,
        left=3pt,
        right=3pt,
        top=3pt,
        bottom=3pt,
        drop shadow
    ]
        \includegraphics[width=0.95\textwidth,keepaspectratio]{apendices/JISA/pagina_1.png}
    \end{tcolorbox}
    \caption{Artículo JISA --- Página 1}\label{fig:jisa-pagina-1}
\end{figure}
\FloatBarrier% Página 2
\begin{figure}[H]
    \centering
    \begin{tcolorbox}[
        colback=white,
        colframe=gray!50,
        boxrule=1pt,
        arc=2pt,
        boxsep=5pt,
        left=3pt,
        right=3pt,
        top=3pt,
        bottom=3pt,
        drop shadow
    ]
        \includegraphics[width=0.95\textwidth,keepaspectratio]{apendices/JISA/pagina_2.png}
    \end{tcolorbox}
    \caption{Artículo JISA --- Página 2}\label{fig:jisa-pagina-2}
\end{figure}
\FloatBarrier% Página 3
\begin{figure}[H]
    \centering
    \begin{tcolorbox}[
        colback=white,
        colframe=gray!50,
        boxrule=1pt,
        arc=2pt,
        boxsep=5pt,
        left=3pt,
        right=3pt,
        top=3pt,
        bottom=3pt,
        drop shadow
    ]
        \includegraphics[width=0.95\textwidth,keepaspectratio]{apendices/JISA/pagina_3.png}
    \end{tcolorbox}
    \caption{Artículo JISA --- Página 3}\label{fig:jisa-pagina-3}
\end{figure}
\FloatBarrier% Página 4
\begin{figure}[H]
    \centering
    \begin{tcolorbox}[
        colback=white,
        colframe=gray!50,
        boxrule=1pt,
        arc=2pt,
        boxsep=5pt,
        left=3pt,
        right=3pt,
        top=3pt,
        bottom=3pt,
        drop shadow
    ]
        \includegraphics[width=0.95\textwidth,keepaspectratio]{apendices/JISA/pagina_4.png}
    \end{tcolorbox}
    \caption{Artículo JISA --- Página 4}\label{fig:jisa-pagina-4}
\end{figure}
\FloatBarrier% Página 5
\begin{figure}[H]
    \centering
    \begin{tcolorbox}[
        colback=white,
        colframe=gray!50,
        boxrule=1pt,
        arc=2pt,
        boxsep=5pt,
        left=3pt,
        right=3pt,
        top=3pt,
        bottom=3pt,
        drop shadow
    ]
        \includegraphics[width=0.95\textwidth,keepaspectratio]{apendices/JISA/pagina_5.png}
    \end{tcolorbox}
    \caption{Artículo JISA --- Página 5}\label{fig:jisa-pagina-5}
\end{figure}
\FloatBarrier% Macro para crear páginas del artículo de forma automática
% Páginas 6-34
\newcounter{jisampage}
\setcounter{jisampage}{6}
\loop%
    \begin{figure}[H]
        \centering
        \begin{tcolorbox}[
            colback=white,
            colframe=gray!50,
            boxrule=1pt,
            arc=2pt,
            boxsep=5pt,
            left=3pt,
            right=3pt,
            top=3pt,
            bottom=3pt,
            drop shadow
        ]
            \includegraphics[width=0.95\textwidth,keepaspectratio]{apendices/JISA/pagina_\thejisampage.png}
        \end{tcolorbox}
        \caption{Artículo JISA --- Página \thejisampage}\label{fig:jisa-pagina-\thejisampage}
    \end{figure}
    \FloatBarrier\stepcounter{jisampage}
    \ifnum\value{jisampage}<35
\repeat%

\section{Slides de CEIFI}
\foreach \n in {1,...,16} {
    \begin{figure}[H]
        \centering
        \fbox{\includegraphics[width=0.9\textwidth]{apendices/CEIFI/\n.png}}
        \caption{Slide \n de CEIFI}
    \end{figure}
}

\section{Slides de GRID 2025-I}
\foreach \n in {1,...,13} {
    \begin{figure}[H]
        \centering
        \fbox{\includegraphics[width=0.9\textwidth]{apendices/GRID-2025/\n.png}}
        \caption{Slide \n de GRID 2025-I}
    \end{figure}
}

\section{Poster GRID 2024-II}
\begin{figure}[H]
    \centering
    \fbox{\includegraphics[width=0.9\textwidth]{apendices/GRID-2024/image.png}}
    \caption{Póster de GRID 2024-II}
\end{figure}

\end{document}