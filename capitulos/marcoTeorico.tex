\ChapterImageStar[cap:Marco-Teorico]{Marco Teórico}{./images/fondo.png}\label{cap:marcoTeorico}
\mbox{}\\
En el contexto de la gestión de proyectos tecnológicos y el desarrollo de software, los marcos de referencia resultan fundamentales para afrontar los desafíos actuales con metodologías claras y estructuras probadas. Estos marcos permiten ubicar el proyecto dentro de corrientes de pensamiento aceptadas y, al mismo tiempo, ofrecen herramientas prácticas para su aplicación en contextos reales. Para el Grupo de Investigación en Redes, Información y Distribución (GRID), su utilización busca mejorar la organización, la calidad y la pertinencia de las soluciones tecnológicas, particularmente en el diseño de arquitecturas basadas en contenedores.

\section{PMBOK}
Uno de los referentes más reconocidos en la gestión de proyectos es el \PMBOK\ (\textit{Project Management Body of Knowledge}), establecido por el Project Management Institute. Este estándar reúne un conjunto amplio de buenas prácticas aplicables a la mayoría de los proyectos, organizando el trabajo en áreas clave como el alcance, tiempo, costos, calidad, riesgos y recursos\citep{project2017guia}. La utilización del \PMBOK\ no solo mejora la gestión y control de los proyectos, sino que también permite alinearlos con los objetivos estratégicos de la organización, propendiendo la entrega de valor y la reducción de riesgos durante su ejecución\citep{Monday2022}.

\section{ISO 9000}
Complementariamente, la norma \ISO\ 9000 aporta una perspectiva centrada en la calidad, promoviendo la estandarización de procesos y la mejora continua\citep{ISO9001}. Esta serie de normas internacionales busca garantizar que las organizaciones respondan de manera consistente a las expectativas de los clientes, mediante la implementación de principios que abarcan desde el liderazgo hasta la gestión de la información y el conocimiento. Aplicar este marco no solo mejora la operación, sino que también fortalece la confianza del cliente y asegura la calidad en los productos y servicios ofrecidos\citep{Gray2022}. Así, se establece una conexión directa entre la gestión de proyectos y los sistemas de calidad, lo que resulta especialmente útil cuando se busca garantizar la sostenibilidad de los resultados.

\section{Modelo por capas}
Para abordar la complejidad técnica de los sistemas desarrollados, se recurre al modelo por capas, una arquitectura que permite dividir el sistema en distintos niveles con funciones específicas y autónomas. Esta forma de organización contribuye a una mayor claridad y modularidad, permitiendo que los componentes de una capa puedan ser modificados sin afectar el resto del sistema\citep{Spray2023}. De este modo, se facilita el mantenimiento, la escalabilidad y la gestión de cambios, cualidades esenciales en el desarrollo de software moderno. La interoperabilidad también se ve fortalecida, dado que esta arquitectura permite una integración más fluida entre distintos módulos y servicios.

\section{CNCF}
En ese mismo sentido, la Cloud Native Computing Foundation (\CNCF) introduce un enfoque moderno para el desarrollo de aplicaciones, orientado a tecnologías nativas de la nube. Este marco promueve prácticas como el uso de contenedores, microservicios y la automatización continua, con el objetivo de construir soluciones más eficientes, escalables y resilientes\citep{CNCF2023}. La \CNCF\ también proporciona herramientas que buscan la portabilidad y la interoperabilidad entre diferentes entornos de nube, lo que permite a las organizaciones adaptarse con mayor agilidad a un entorno cambiante y competitivo. Su enfoque abierto e interoperable lo convierte en un aliado clave para iniciativas que busquen aprovechar al máximo las capacidades de la nube.

\section{Design thinking}
Junto a estas herramientas técnicas y de gestión, el Design Thinking aporta una perspectiva centrada en las personas, enfocándose en comprender profundamente las necesidades del usuario para proponer soluciones innovadoras\citep{CombellesC.LucenaP.2020}. Esta metodología fomenta la empatía, la experimentación y la colaboración interdisciplinaria, promoviendo la creación de productos y servicios que se ajusten con mayor precisión a las demandas reales del contexto. Su inclusión en proyectos tecnológicos no solo impulsa la innovación, sino que también fortalece la toma de decisiones ágiles y adaptativas, favoreciendo entornos flexibles en constante evolución.

\section{TOGAF}
Por su parte, \TOGAF\ (The Open Group Architecture Framework) complementa este conjunto de marcos al enfocarse en la alineación entre la estrategia del negocio y los procesos de tecnología de la información. Mediante su enfoque estructurado por fases —que abarca desde la planificación hasta la implementación y el monitoreo— \TOGAF\ permite gestionar arquitecturas empresariales de forma coherente y flexible. Su aplicación ayuda en el uso recursos, integración de sistemas y toma de decisiones estratégicas con una visión holística de la organización\citep{Mumtaza2025}.

\section{ISO/IEC 25010}
Finalmente, la norma \ISO/\IEC\ 25010 establece un modelo integral para la evaluación de la calidad del software, considerando atributos como la funcionalidad, usabilidad, seguridad, mantenibilidad y portabilidad\citep{ISO25010}. Este marco teórico es fundamental para asegurar que los sistemas desarrollados cumplan con los requisitos tanto del negocio como del usuario final, proporcionando un enfoque riguroso que permite identificar áreas de mejora en las distintas etapas del ciclo de vida del software. Su adopción permite fortalecer la confianza en los productos desarrollados y garantizar su robustez en contextos dinámicos.\\

Todos estos marcos, aunque distintos en su enfoque, se complementan entre sí y permiten establecer una base para la formulación y ejecución de proyectos tecnológicos. Su integración permite abordar los retos desde múltiples dimensiones —estratégica, técnica, organizacional y humana—, ayudando al diseño de soluciones innovadoras y sostenibles.