\ChapterImageStar[cap:justificacion]{Justificación}{./images/fondo.png}\label{cap:justificacion}
\mbox{}\\


En la actualidad, según~\cite{Bianchi2013} la computación distribuida prueba ser una herramienta valiosa para la experimentación tanto científica, experimental como práctica, en cualquiera de las especialidades de ciencias donde se las desee enmarcar. Un claro ejemplo es el análisis de grandes cantidades de datos (\textit{Big Data Analytics}), que según~\cite{Tsai2015} puede utilizar modelos de computación distribuida como uno de sus métodos para gestionar enormes cantidades de información. Además, algunas aplicaciones de software modernas dedicadas a la investigación y el \textit{deep learning} requieren más recursos de los que un solo nodo computacional puede proporcionar; en el caso de esta última, como argumentan~\cite{Thomson2023}, si la carga computacional que requiere el \textit{deep learning} continúa, este terminará por volverse técnica y económicamente prohibitivo.
\\
Además de lo anterior, debe tenerse en cuenta que \GRID~realiza investigaciones en temas afines con las redes de computadoras, sistemas distribuidos, seguridad informática, entre otros. La investigación interdisciplinaria y los proyectos con universidades de primer nivel, tanto nacionales como internacionales, son también objetivos alineados con la misión heredada del \GRID~como lo manifiestan sus integrantes. Así, ampliar la infraestructura HTCondor existente para que soporte un universo adicional y de interés para el \GRID~representa un paso estratégico para alcanzar estos objetivos.
\\
En este contexto, surge una pregunta relevante para el entorno de la Universidad del Quindío: ¿Es justificable ampliar la implementación de la infraestructura HTCondor de \GRID? Se considera que la respuesta es afirmativa. Factores como el incremento en la capacidad investigativa, el aprovechamiento de recursos computacionales subutilizados, la colaboración con otros grupos de investigación tanto locales como de otras universidades y la mejora en la competitividad de la Universidad del Quindío son razones alineadas con la misión institucional así como del \GRID.
