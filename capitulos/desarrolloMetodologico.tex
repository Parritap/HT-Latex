\ChapterImageStar[cap:desarrollo-metodologico]{Desarrollo metodológico}{./images/fondo.png}\label{cap:desarrolloMetodologico}
\mbox{}\\
A continuación, se describe el procedimiento metodológico seguido para alcanzar los objetivos planteados en esta investigación. La metodología se estructuró en fases sucesivas y complementarias que permiten pasar de la caracterización del contexto institucional y tecnológico, hacia la selección, diseño, implementación y validación de una arquitectura basada en tecnologías de virtualización por contenedores (\VBC).

\section{Caracterización del GRID}
El Grupo de Investigación en Redes, Información y Distribución (\GRID) de la Universidad del Quindío desarrolla actividades en los ejes misionales de la institución: educación, investigación y extensión. En el marco de esta investigación, se caracterizó el \GRID\ con el propósito de identificar sus capacidades actuales, necesidades y oportunidades relacionadas con la adopción de tecnologías de virtualización. Este diagnóstico inicial permitió contextualizar la pertinencia de las \VBC\ como una alternativa tecnológica para fortalecer los servicios académicos y de investigación, especialmente en beneficio de los estudiantes de Ingeniería de Sistemas y Computación.

\section{Revisión de la literatura}
Con el fin de fundamentar la investigación, se realizó un mapeo sistemático de estudios (\SMS). Este consistió en la búsqueda, filtrado, selección y análisis de literatura académica, artículos técnicos y reportes de caso relacionados con las \VBC. El objetivo fue obtener una visión global y estructurada sobre las tecnologías disponibles, sus tendencias de adopción y las principales dimensiones de análisis empleadas en la comunidad científica y profesional.

\section{Identificación y caracterización de tecnologías VBC}
A partir de los resultados del \SMS, se seleccionaron las tecnologías de \VBC\ con mayor relevancia e impacto en la literatura y la práctica. Para cada una de ellas se realizó una caracterización técnica, evaluando aspectos como arquitectura interna, facilidad de integración, integración con la nube y comunidad de soporte. Esta fase permitió construir un marco comparativo preliminar que orienta la elección de herramientas candidatas para el \GRID.

\section{Benchmarking de tecnologías VBC}
Posteriormente, se diseñó y ejecutó un proceso de \textit{benchmarking} enfocado en medir y contrastar el desempeño de un conjunto de tecnologías seleccionadas bajo condiciones controladas. Los criterios de evaluación incluyeron consumo de \CPU, uso de memoria, throughput de red y operaciones de entrada/salida (I/O). Los resultados permitieron establecer métricas que evidencian fortalezas y limitaciones de cada tecnología, facilitando la selección informada de la alternativa adecuada para el contexto institucional.

\section{Análisis de Decisión y Resolución (DAR)}
Con base en los resultados del \textit{benchmarking}, se aplicó un análisis de Decisión y Resolución (\DAR). Este método permitió ponderar los beneficios, riesgos y oportunidades asociados con la adopción de las \VBC\ en el \GRID. El \DAR\ integró tanto los criterios técnicos como los organizacionales, priorizando aquellos que propenden por la sostenibilidad de la solución a mediano y largo plazo.

\section{Diseño de la solución arquitectónica}
En esta fase se elaboró la propuesta de arquitectura tecnológica que articula la infraestructura existente en el \GRID\ con las capacidades de la tecnología seleccionada. El diseño incluyó la definición de componentes, interacciones, flujos de información y políticas de gestión, buscando escalabilidad, resiliencia y facilidad de administración de la solución.

\section{Implementación de la solución}
Con el diseño arquitectónico como guía, se procedió a implementar un producto mínimo viable (\PMV) que materializa la adopción de la tecnología seleccionada. La implementación se llevó a cabo en el entorno del \GRID, integrando las configuraciones necesarias y desplegando servicios básicos que permiten evaluar la funcionalidad del sistema en condiciones reales.

\section{Validación de la solución}
Finalmente, se realizó la validación del \PMV\ mediante pruebas de desempeño, disponibilidad y escalabilidad, contrastando los resultados con los requerimientos definidos en la fase de caracterización del \GRID. Adicionalmente, se consideraron percepciones de los usuarios del grupo de investigación como insumo para verificar la pertinencia y aplicabilidad de la solución propuesta.