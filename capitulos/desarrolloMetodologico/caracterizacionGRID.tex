\ChapterImageStar[cap:caracterizacionGRID]{Caracterización del GRID}{./images/fondo.png}\label{cap:caracterizacionGRID}
\mbox{}\\
\noindent
El Grupo de Investigación en Redes, Información y Distribución (\GRID) de la Universidad del Quindío se enmarca en los objetivos misionales de la institución: educación, investigación y extensión. Uno del los intereses particulares del \GRID es ofrecer servicios tecnológicos avanzados a la comunidad académica, con énfasis en los estudiantes de Ingeniería de Sistemas y Computación, quienes encuentran en este grupo un espacio de formación e innovación en temas de infraestructura, ingeniería de software y tecnologías emergentes.\\
La caracterización del \GRID\ resulta esencial para comprender su estructura, capacidades y necesidades en relación con la expansión de un nuevo universo HTCondor. A continuación, se presenta un análisis de los diferentes aspectos que definen el contexto institucional y tecnológico del grupo.

\section{Análisis de stakeholders del GRID}
\noindent
Con el fin de identificar los actores internos y externos que influyen en el desarrollo de las actividades del grupo, se realizó un análisis de \textit{stakeholders}. Este ejercicio permitió reconocer los diferentes intereses, roles y niveles de influencia que cada actor tiene en relación con la expansión de un nuevo universo HTCondor. Los principales \textit{stakeholders} identificados incluyen: investigadores del grupo, estudiantes de pregrado y posgrado, docentes de la Facultad de Ingeniería, y en un nivel más amplio, la comunidad académica de la Universidad del Quindío.
\\
\noindent
La tabla~\ref{tab:stakeholders} presenta el análisis de los principales interesados en la expansión de los universos HTCondor dentro del contexto institucional. Se identifican actores internos y externos, especificando su rol, el tipo de relación con el proyecto, el nivel de impacto esperado, así como su poder de influencia, interés y compromiso frente a la iniciativa.

El análisis de interesados revela una estructura compleja de actores con distintos niveles de influencia y compromiso frente al proyecto. El Grupo de Investigación GRID emerge como el interesado crítico, concentrando simultáneamente el mayor impacto, poder de influencia, interés y compromiso. Esta posición central le otorga un rol determinante como beneficiario principal y tomador de decisiones, lo que subraya la necesidad de mantener una comunicación estrecha y continua con este grupo para asegurar la alineación del proyecto con sus expectativas estratégicas y operativas.

Los docentes de Ingeniería de Sistemas y los investigadores locales y externos constituyen el segundo nivel de importancia, caracterizándose por un alto interés en la solución y un potencial significativo como usuarios finales. Aunque su poder de decisión es limitado, su adopción efectiva de la tecnología será fundamental para validar el éxito del proyecto. Este segmento requiere especial atención en términos de usabilidad, documentación y capacitación, dado que su compromiso está condicionado a la utilidad práctica y los beneficios tangibles que puedan obtener. La diferencia principal entre ambos grupos radica en que los investigadores externos representarían el segmento de usuarios más activo e intensivo de la infraestructura.

El Programa de Ingeniería de Sistemas y Computación desempeña un rol estratégico como facilitador institucional con alto poder de influencia, particularmente en la provisión de recursos y la posibilidad de escalar la solución hacia otros programas académicos. Sin embargo, su compromiso relativamente bajo sugiere que será necesario demostrar claramente el valor estratégico e institucional del proyecto para asegurar su respaldo sostenido. Por otro lado, los estudiantes de pregrado presentan bajo impacto, poder e interés debido a la naturaleza especializada de la computación distribuida en el contexto académico de pregrado, lo que los posiciona como beneficiarios secundarios que validarán la solución más por uso ocasional que por dependencia operativa.

Finalmente, la comunidad HTCondor y los investigadores en HTC y HPC representan un interesado externo con una relación simbiótica particular: aunque no tienen poder de decisión sobre el proyecto y su compromiso es bajo, su rol como proveedores de conocimiento técnico es de alto impacto. La documentación y experiencias generadas por el proyecto podrían contribuir al fortalecimiento del ecosistema HTCondor, creando un valor agregado que trasciende los objetivos locales del proyecto. Esta relación sugiere la conveniencia de establecer canales de comunicación con esta comunidad para compartir hallazgos y mejores prácticas.

\begin{table}[H]
\centering
\scriptsize % reduce el tamaño de la letra en 2 puntos aprox.
\renewcommand{\arraystretch}{1.3} % espacio entre filas
\begin{tabularx}{\textwidth}{%
    >{\raggedright\arraybackslash}X
    >{\raggedright\arraybackslash}X
    >{\raggedright\arraybackslash}X
    >{\raggedright\arraybackslash}X
    >{\raggedright\arraybackslash}X
    >{\raggedright\arraybackslash}X
    >{\raggedright\arraybackslash}X}
\toprule
\textbf{Interesado} & \textbf{Rol} & \textbf{Relación} & \textbf{Impacto} & \textbf{Poder de influencia} & \textbf{Interés} & \textbf{Compromiso} \\
\midrule
Grupo de Investigación GRID & Beneficiario principal & Provee infraestructura, evalúa la solución y su impacto & Alto & Alto, decide la adopción de la tecnología & Alto, busca mejorar sus servicios & Alto, ya que su infraestructura será potenciada \\
\midrule
Docentes de Ingeniería de Sistemas & Usuarios clave & Harán uso de los entregables para proyectos y enseñanza & Medio-Alto & Medio, pueden sugerir mejoras pero no decidir implementación & Medio-Alto, esperan decisiones sobre tecnología para enseñanza e investigación & Medio, dependerá de la utilidad de la solución \\
\midrule
Estudiantes de Ingeniería de Sistemas & Usuarios finales & Usarán los servicios en sus cursos y proyectos. Podrán informarse sobre el estudio. & Medio & Bajo, no tienen poder de decisión, pero su uso validará la solución & Alto, necesitan un entorno estable y eficiente & Medio-Alto, dependiendo de la accesibilidad y usabilidad \\
\midrule
Programa de ingeniería de sistemas y computación & Facilitador & Puede apoyar con recursos y normativas para la adopción & Alto & Alto, puede aprobar recursos & Medio, su interés es institucional y estratégico & Bajo-Medio, si la solución no afecta directamente a su gestión \\
\midrule
Proveedores de tecnología (Docker, Kubernetes, etc.) & Proveedores de herramientas & Proveen la tecnología de virtualización a utilizar & Bajo & Bajo, la decisión de uso recae en el GRID y la universidad & Bajo aunque buscan ampliar su base de usuarios & Bajo, su involucramiento es indirecto \\
\midrule
Investigadores y otros grupos de investigación & Potenciales beneficiarios & Pueden usar los resultados en búsqueda de mejoras para sus proyectos & Medio & Medio, pueden influir con solicitudes de mejora & Medio, dependiendo de su relación con GRID & Bajo, solo si ven beneficios concretos \\
\midrule
Sector empresarial & Potencial inversor o socio & Podría apoyar la solución si ve ventajas en la adopción de TVBC & Bajo-Medio & Bajo, no decide en la universidad, pero puede ofrecer incentivos & Bajo-Medio, si la tecnología ofrece valor comercial & Bajo, depende de la alineación con sus intereses \\
\bottomrule
\end{tabularx}
\caption{Análisis de stakeholders}\label{tab:stakeholders}
\end{table}

\section{Priorización de stakeholders}
Una vez realizada la identificación de los \textit{stakeholders}, se emprendió un proceso de priorización para determinar cuáles poseen mayor impacto y poder de decisión en el proyecto. Esta clasificación resulta crucial para establecer estrategias de comunicación, gestión de expectativas y participación activa en la definición de requerimientos. De esta manera, se busca que los actores más influyentes en la toma de decisiones y en la adopción tecnológica sean atendidos de forma prioritaria, aumentando las probabilidades de éxito en la implementación.

\begin{figure}[H]
    \centering
    \includegraphics[width=\textwidth] {tablas-images/cp1/priorizacionStakeholders.png}
    \caption{Priorización de stakeholders del proyecto}\label{fig:tabla-priorizacion-stakeholders}
\end{figure}

\section{Integrantes y áreas de trabajo del GRID}
El \GRID\ está conformado por un equipo multidisciplinario de investigadores y profesionales que, desde sus diferentes áreas de experticia, contribuyen al avance en campos como computación de alto rendimiento, \textit{big data}, inteligencia artificial, redes de computadoras y desarrollo de software. A continuación, se presenta una lista de los integrantes actuales del grupo, junto con sus respectivas líneas de trabajo y áreas de especialización:
\begin{itemize}
	\item \href{https://scienti.minciencias.gov.co/cvlac/visualizador/generarCurriculoCv.do?cod_rh=0000210897}{\underline{{\textbf{Christian Andrés Candela Uribe}}}}: Microservicios, desarrollo de software, minería de datos, infraestructura TI.\@
	\item \href{https://scienti.minciencias.gov.co/cvlac/visualizador/generarCurriculoCv.do?cod_rh=0001383939}{\underline{{\textbf{Luis Eduardo Sepúlveda Rodríguez}}}}: Infraestructura de TI, HPC, computación paralela.
	\item \href{https://scienti.minciencias.gov.co/cvlac/visualizador/generarCurriculoCv.do?cod_rh=0001638854}{\underline{{\textbf{Carlos Andrés Flórez Villarraga}}}}: Programación y algoritmia, inteligencia artificial.
	\item \href{https://scienti.minciencias.gov.co/cvlac/visualizador/generarCurriculoCv.do?cod_rh=0001343801}{\underline{{\textbf{Carlos Eduardo Gómez Montoya}}}}: Redes, ingeniería de software, cloud computing.
	\item \href{https://scienti.minciencias.gov.co/cvlac/visualizador/generarCurriculoCv.do?cod_rh=0001398775}{\underline{{\textbf{Sergio Augusto Cardona Torres}}}}: Big data y análisis de datos, ingeniería de software, sistemas adaptativos, informática educativa.
	\item \href{https://scienti.minciencias.gov.co/cvlac/visualizador/generarCurriculoCv.do?cod_rh=0000193550}{\underline{{\textbf{Sonia Jaramillo Valbuena}}}}: Big data, electroquímica, inteligencia artificial.
	\item \href{https://scienti.minciencias.gov.co/cvlac/visualizador/generarCurriculoCv.do?cod_rh=0000283495}{\underline{{\textbf{Julián Esteban Gutiérrez Posada}}}}: Microservicios, desarrollo de software, minería de datos, infraestructura TI, HPC, computación paralela, redes, ingeniería de software.
\end{itemize}

La diversidad de líneas de trabajo de los integrantes fortalece la capacidad del grupo para abordar proyectos de carácter transversal y multidisciplinario, lo cual resulta particularmente relevante para el diseño e implementación de soluciones arquitectónicas soportadas en tecnologías de virtualización.

\section{Misión del GRID}
La misión del GRID es heredada de la Universidad del Quindío. A continuación se presenta la misión del GRID:\@

\begin{quote}
	\textit{La Universidad del Quindío contribuye a la transformación de la sociedad, mediante la formación integral desde el ser, el saber y el hacer, de líderes reflexivos y gestores del cambio; con estándares de calidad, a través de una oferta de formación en diferentes metodologías, que responda a una sociedad basada en el conocimiento; una investigación pertinente, que aporte a la solución de las problemáticas del desarrollo e integrada con la extensión y proyección social; educando en tiempos del posconflicto y de la consolidación de la paz, apoyada en una gestión creativa y con estándares de calidad.}
\end{quote}

A partir de esta misión, se identifican los siguientes pilares fundamentales:

\begin{itemize}
	\item \textbf{Docencia:} La Universidad del Quindío contribuye a la transformación de la sociedad, mediante la formación integral desde el ser, el saber y el hacer, de líderes reflexivos y gestores del cambio; con estándares de calidad, a través de una oferta de formación en diferentes metodologías, que responda a una sociedad basada en el conocimiento.

	\item \textbf{Investigación:} Una investigación pertinente, que aporte a la solución de las problemáticas del desarrollo e integrada con la extensión y proyección social.

	\item \textbf{Extensión y Desarrollo Social:} Apoyada en una gestión creativa y con estándares de calidad.

	\item \textbf{Responsabilidad Social:} Educando en tiempos del posconflicto y de la consolidación de la paz.
\end{itemize}

\section{Visión del GRID}
La misión de la Universidad del Quindío se complementa con su visión institucional, la cual también es adoptada por el \GRID. A continuación se presenta la visión del \GRID:\@

\begin{quote}
	\textit{En el año 2025, la Universidad del Quindío estará consolidada como una institución \textit{Pertinente --- Creativa --- Integradora}, acreditada de alta calidad, con reconocimiento nacional e internacional en sus procesos de formación a través de diferentes metodologías, de investigación, extensión, proyección y responsabilidad social.}
\end{quote}

A partir de esta visión, se destacan los siguientes enfoques estratégicos:

\begin{itemize}
	\item \textbf{Gestión:} La Universidad del Quindío estará consolidada como una institución \textit{Pertinente --- Creativa --- Integradora}.

	\item \textbf{Docencia:} Acreditada de alta calidad en sus procesos de formación a través de diferentes metodologías.

	\item \textbf{Investigación:} Consolidada como pertinente y de alta calidad en sus procesos de investigación.

	\item \textbf{Extensión y Desarrollo Social:} Procesos creativos e integradores en proyección social.

	\item \textbf{Responsabilidad Social:} Reconocimientos en sus procesos de responsabilidad social.
\end{itemize}

\section{Impacto del proyecto en el GRID}

El proyecto tiene como objetivo apoyar los procesos de docencia, investigación
y extensión mediante la especificación de una arquitectura de tecnologías de
virtualización basada en contenedores (\VBC).
Este trabajo se enfoca en la identificación de una tecnología de contenerización que \textbf{agregue valor a los procesos del grupo}, beneficiando a docentes, estudiantes y cualquier parte interesada que participe en los proyectos y actividades desarrolladas por este grupo de investigación.

\section{Caracterización de la infraestructura tecnológica del GRID}
En la presente sección se van a especificar las características técnicas de la infraestructura tecnológica del \GRID\ disponible para temas de virtualización. Esta caracterización se construyó usando una \href{https://docs.google.com/spreadsheets/d/14NBv72ucVTrLqGIldYdIsjdBGt3QlgwcblcVRis-DaQ/edit?usp=sharing}{macro} que permite facilitar la recolección de información técnica de los servidores y equipos de cómputo disponibles en el \GRID.

Los cuadros~\ref{tab:torre-hp-1},~\ref{tab:torre-2},~\ref{tab:torre-3} y~\ref{tab:torre-4} presentan las fichas técnicas de cuatro servidores tipo torre pertenecientes al inventario del Grupo de investigación. Se trata de equipos marca HP, sin modelo específico identificado, pero con características homogéneas en su configuración física y técnica: disponen de ocho puertos USB (cuatro frontales y cuatro posteriores), entradas y salidas de audio y micrófono, conexión HDMI, lector de DVD, tres interfaces Ethernet, puerto DisplayPort y conectores PS/2 para teclado y ratón. Su propósito principal es funcionar como hipervisores con XCP-ng, y se encuentran adaptados para soportar procesos de virtualización. Estos recursos, identificados con diferentes códigos de inventario y números en el \CPD\, hacen parte del \textit{cluster} actual, usados para el desarrollo de proyectos dentro del \GRID.
% Archivo de caracterización de infraestructura corregido

% Torre HP 1
\begin{table}[H]
\centering
\scriptsize % Tamaño de fuente más pequeño
\setlength{\tabcolsep}{2pt} % Menor espacio entre columnas
\renewcommand{\arraystretch}{1.0} % Espaciado más ajustado
\caption{Ficha técnica --- Torre 1}\label{tab:torre-hp-1}
\begin{tabular}{|p{0.5\textwidth}|p{0.2\textwidth}|} % Columnas más estrechas
\hline
\multicolumn{2}{|l|}{\textbf{DESCRIPCIÓN FÍSICA:} Servidor tipo torre} \\ \hline
\textbf{TIPO DE RECURSO:} Torre &
\multirow{5}{*}{\includegraphics[width=0.18\textwidth,keepaspectratio]{tablas-images/cp1/torres/torre-1.png}} \\ \cline{1-1}
\textbf{MODELO:} Desconocido & \\ \cline{1-1}
\textbf{MARCA:} HP & \\ \cline{1-1}
\textbf{CÓDIGO DE INVENTARIO:} 7 24390 49867 3 & \\ \cline{1-1}
\textbf{NÚMERO EN CPD:} 14 & \\ \hline
\multicolumn{2}{|l|}{\textbf{ESPECIFICACIONES TÉCNICAS}} \\ \hline
\multicolumn{2}{|p{0.7\textwidth}|}{ % Ancho ajustado
- 8 USB (4 frontal, 4 trasera)
- Audio y micrófono
- HDMI
- Lector DVD
- 3 Ethernet
- DisplayPort
- PS/2 (Teclado/Ratón)
} \\ \hline
\multicolumn{2}{|l|}{\textbf{PROPÓSITO:} Hipervisor XCP-ng} \\ \hline
\multicolumn{2}{|l|}{\textbf{OPORTUNIDAD DE USO:} Proyectos del \GRID} \\ \hline
\multicolumn{2}{|p{0.7\textwidth}|}{\textbf{OBSERVACIONES:} Sin modelo. Equipo adaptado para virtualización.} \\ \hline
\end{tabular}
\end{table}

% Torre 2
\begin{table}[H]
\centering
\scriptsize
\setlength{\tabcolsep}{2pt}
\renewcommand{\arraystretch}{1.0}
\caption{Ficha técnica --- Torre 2}\label{tab:torre-2}
\begin{tabular}{|p{0.5\textwidth}|p{0.2\textwidth}|}
\hline
\multicolumn{2}{|l|}{\textbf{DESCRIPCIÓN FÍSICA:} Servidor tipo torre} \\ \hline
\textbf{TIPO DE RECURSO:} Torre & 
\multirow{5}{*}{\includegraphics[width=0.18\textwidth,keepaspectratio]{tablas-images/cp1/torres/torre-1.png}} \\ \cline{1-1}
\textbf{MODELO:} Desconocido & \\ \cline{1-1}
\textbf{MARCA:} HP & \\ \cline{1-1}
\textbf{CÓDIGO DE INVENTARIO:} 7 24390 49861 1 & \\ \cline{1-1}
\textbf{NÚMERO EN CPD:} 12 & \\ \hline
\multicolumn{2}{|l|}{\textbf{ESPECIFICACIONES TÉCNICAS}} \\ \hline
\multicolumn{2}{|p{0.7\textwidth}|}{
- 8 USB (4 frontal, 4 trasera)
- Audio y micrófono
- HDMI
- Lector DVD
- 3 Ethernet
- DisplayPort
- PS/2 (Teclado/Ratón)
} \\ \hline
\multicolumn{2}{|l|}{\textbf{PROPÓSITO:} Hipervisor XCP-ng} \\ \hline
\multicolumn{2}{|l|}{\textbf{OPORTUNIDAD DE USO:} Proyectos del \GRID} \\ \hline
\multicolumn{2}{|p{0.7\textwidth}|}{\textbf{OBSERVACIONES:} Sin modelo. Equipo adaptado para virtualización.} \\ \hline
\end{tabular}
\end{table}

% Torre 3
\begin{table}[H]
\centering
\scriptsize
\setlength{\tabcolsep}{2pt}
\renewcommand{\arraystretch}{1.0}
\caption{Ficha técnica -- Torre 3}\label{tab:torre-3}
\begin{tabular}{|p{0.5\textwidth}|p{0.2\textwidth}|}
\hline
\multicolumn{2}{|l|}{\textbf{DESCRIPCIÓN FÍSICA:} Servidor tipo torre} \\ \hline
\textbf{TIPO DE RECURSO:} Torre & 
\multirow{5}{*}{\includegraphics[width=0.18\textwidth,keepaspectratio]{tablas-images/cp1/torres/torre-1.png}} \\ \cline{1-1}
\textbf{MODELO:} Desconocido & \\ \cline{1-1}
\textbf{MARCA:} HP & \\ \cline{1-1}
\textbf{CÓDIGO DE INVENTARIO:} 7 24390 49969 4 & \\ \cline{1-1}
\textbf{NÚMERO EN CPD:} 13 & \\ \hline
\multicolumn{2}{|l|}{\textbf{ESPECIFICACIONES TÉCNICAS}} \\ \hline
\multicolumn{2}{|p{0.7\textwidth}|}{
- 8 USB (4 frontal, 4 trasera)
- Audio y micrófono
- HDMI
- Lector DVD
- 3 Ethernet
- DisplayPort
- PS/2 (Teclado/Ratón)
} \\ \hline
\multicolumn{2}{|l|}{\textbf{PROPÓSITO:} Hipervisor XCP-ng} \\ \hline
\multicolumn{2}{|l|}{\textbf{OPORTUNIDAD DE USO:} Proyectos del \GRID} \\ \hline
\multicolumn{2}{|p{0.7\textwidth}|}{\textbf{OBSERVACIONES:} Sin modelo. Equipo adaptado para virtualización.} \\ \hline
\end{tabular}
\end{table}

% Torre 4
\begin{table}[H]
\centering
\scriptsize
\setlength{\tabcolsep}{2pt}
\renewcommand{\arraystretch}{1.0}
\caption{Ficha técnica --- Torre 4}\label{tab:torre-4}
\begin{tabular}{|p{0.5\textwidth}|p{0.2\textwidth}|}
\hline
\multicolumn{2}{|l|}{\textbf{DESCRIPCIÓN FÍSICA:} Servidor tipo torre} \\ \hline
\textbf{TIPO DE RECURSO:} Torre & 
\multirow{5}{*}{\includegraphics[width=0.18\textwidth,keepaspectratio]{tablas-images/cp1/torres/torre-1.png}} \\ \cline{1-1}
\textbf{MODELO:} Desconocido & \\ \cline{1-1}
\textbf{MARCA:} HP & \\ \cline{1-1}
\textbf{CÓDIGO DE INVENTARIO:} 7 24390 49879 4 & \\ \cline{1-1}
\textbf{NÚMERO EN CPD:} 14 & \\ \hline
\multicolumn{2}{|l|}{\textbf{ESPECIFICACIONES TÉCNICAS}} \\ \hline
\multicolumn{2}{|p{0.7\textwidth}|}{
- 8 USB (4 frontal, 4 trasera)
- Audio y micrófono
- HDMI
- Lector DVD
- 3 Ethernet
- DisplayPort
- PS/2 (Teclado/Ratón)
} \\ \hline
\multicolumn{2}{|l|}{\textbf{PROPÓSITO:} Hipervisor XCP-ng} \\ \hline
\multicolumn{2}{|l|}{\textbf{OPORTUNIDAD DE USO:} Proyectos del \GRID} \\ \hline
\multicolumn{2}{|p{0.7\textwidth}|}{\textbf{OBSERVACIONES:} Sin modelo. Equipo adaptado para virtualización.} \\ \hline
\end{tabular}
\end{table}


Los cuadros~\ref{tab:torre-5} y~\ref{tab:torre-6} detallan las características de dos servidores tipo torre HP, modelo G9, destinados a operar como hipervisores bajo XCP-ng. Ambos equipos, identificados en el inventario con los códigos 72992 y 72976 y ubicados en los puestos 22 y 21 del CPD, respectivamente, cuentan con especificaciones técnicas similares: nueve puertos USB (cuatro frontales y cinco posteriores), entradas y salidas de audio y micrófono, conexión HDMI, lector de DVD, una interfaz Ethernet, dos puertos DisplayPort y un procesador Intel vPro i9, lo que les otorga un mayor rendimiento frente a los servidores del mismo entorno. Estos equipos han sido adaptados para virtualización, reforzando la infraestructura tecnológica del GRID y ampliando las capacidades de cómputo necesarias para el desarrollo de entornos virtualizados de alto desempeño.
% Torre 5
\begin{table}[H]
\centering
\scriptsize
\setlength{\tabcolsep}{2pt}
\renewcommand{\arraystretch}{1.0}
\caption{Ficha técnica --- Torre 5}\label{tab:torre-5}
\begin{tabular}{|p{0.5\textwidth}|p{0.2\textwidth}|}
\hline
\multicolumn{2}{|l|}{\textbf{DESCRIPCIÓN FÍSICA:} Servidor tipo torre} \\ \hline
\textbf{TIPO DE RECURSO:} Torre & 
\multirow{5}{*}{\includegraphics[width=0.18\textwidth,keepaspectratio]{tablas-images/cp1/torres/torre-2.png}} \\ \cline{1-1}
\textbf{MODELO:} G9 & \\ \cline{1-1}
\textbf{MARCA:} HP & \\ \cline{1-1}
\textbf{CÓDIGO DE INVENTARIO:} 72992 & \\ \cline{1-1}
\textbf{NÚMERO EN CPD:} 22 & \\ \hline
\multicolumn{2}{|l|}{\textbf{ESPECIFICACIONES TÉCNICAS}} \\ \hline
\multicolumn{2}{|p{0.7\textwidth}|}{
- 9 USB (4 frontal, 5 trasera)
- Audio y micrófono
- HDMI
- Lector DVD
- 1 Ethernet
- 2 DisplayPort
- Procesador Intel vPro i9
} \\ \hline
\multicolumn{2}{|l|}{\textbf{PROPÓSITO:} Hipervisor XCP-ng} \\ \hline
\multicolumn{2}{|l|}{\textbf{OPORTUNIDAD DE USO:} Proyectos del \GRID} \\ \hline
\multicolumn{2}{|p{0.7\textwidth}|}{\textbf{OBSERVACIONES:} Equipo adaptado para virtualización.} \\ \hline
\end{tabular}
\end{table}

% Torre 6
\begin{table}[H]
\centering
\scriptsize
\setlength{\tabcolsep}{2pt}
\renewcommand{\arraystretch}{1.0}
\caption{Ficha técnica --- Torre 6}
\label{tab:torre-6}
\begin{tabular}{|p{0.5\textwidth}|p{0.2\textwidth}|}
\hline
\multicolumn{2}{|l|}{\textbf{DESCRIPCIÓN FÍSICA:} Servidor tipo torre} \\ \hline
\textbf{TIPO DE RECURSO:} Torre & 
\multirow{5}{*}{\includegraphics[width=0.18\textwidth,keepaspectratio]{tablas-images/cp1/torres/torre-2.png}} \\ \cline{1-1}
\textbf{MODELO:} G9 & \\ \cline{1-1}
\textbf{MARCA:} HP & \\ \cline{1-1}
\textbf{CÓDIGO DE INVENTARIO:} 72976 & \\ \cline{1-1}
\textbf{NÚMERO EN CPD:} 21 & \\ \hline
\multicolumn{2}{|l|}{\textbf{ESPECIFICACIONES TÉCNICAS}} \\ \hline
\multicolumn{2}{|p{0.7\textwidth}|}{
- 9 USB (4 frontal, 5 trasera)
- Audio y micrófono
- HDMI
- Lector DVD
- 1 Ethernet
- 2 DisplayPort
- Procesador Intel vPro i9
} \\ \hline
\multicolumn{2}{|l|}{\textbf{PROPÓSITO:} Hipervisor XCP-ng} \\ \hline
\multicolumn{2}{|l|}{\textbf{OPORTUNIDAD DE USO:} Proyectos del \GRID} \\ \hline
\multicolumn{2}{|p{0.7\textwidth}|}{\textbf{OBSERVACIONES:} Equipo adaptado para virtualización.} \\ \hline
\end{tabular}
\end{table}

El cuadro~\ref{tab:torre-7} presenta la ficha técnica del servidor tipo torre identificado como Torre 7, perteneciente a la marca Argom Tech y sin código de inventario asignado, ubicado en el puesto 11 del \CPD\. Este equipo dispone de un procesador Intel Pentium 62030 de 3.00 GHz con arquitectura x64, memoria RAM de 16 GB, disco duro de 1 TB, unidad de CD/DVD y tarjetas de video y sonido integradas. Su propósito principal es operar como hipervisor bajo la plataforma XCP-ng. Cabe destacar que en este servidor se encuentra implementada la solución arquitectónica propuesta, lo que lo convierte en un recurso fundamental para la validación y consolidación del entorno de contenedores diseñado para el \GRID.
% Torre 7
\begin{table}[H]
\centering
\scriptsize
\setlength{\tabcolsep}{2pt}
\renewcommand{\arraystretch}{1.0}
\caption{Ficha técnica --- Torre 7}\label{tab:torre-7}
\begin{tabular}{|p{0.5\textwidth}|p{0.2\textwidth}|}
\hline
\multicolumn{2}{|l|}{\textbf{DESCRIPCIÓN FÍSICA:} Servidor tipo torre} \\ \hline
\textbf{TIPO DE RECURSO:} Torre & 
\multirow{5}{*}{\includegraphics[width=0.18\textwidth,keepaspectratio]{tablas-images/cp1/torres/ATX.png}} \\ \cline{1-1}
\textbf{MODELO:} Argom tech & \\ \cline{1-1}
\textbf{MARCA:} Argom tech & \\ \cline{1-1}
\textbf{CÓDIGO DE INVENTARIO:} Sin código & \\ \cline{1-1}
\textbf{NÚMERO EN CPD:} 11 & \\ \hline
\multicolumn{2}{|l|}{\textbf{ESPECIFICACIONES TÉCNICAS}} \\ \hline
\multicolumn{2}{|p{0.7\textwidth}|}{
- Procesador: Intel Pentium 62030 3.00GHz
- Arquitectura: X64
- RAM: 16GB
- Disco: 1024GB
- Unidad CD/DVD: Sí
- Tarjeta video: Integrada
- Tarjeta sonido: Integrada
} \\ \hline
\multicolumn{2}{|l|}{\textbf{PROPÓSITO:} Hipervisor XCP-ng} \\ \hline
\multicolumn{2}{|l|}{\textbf{OPORTUNIDAD DE USO:} Proyectos del \GRID} \\ \hline
\multicolumn{2}{|p{0.7\textwidth}|}{\textbf{OBSERVACIONES:} Equipo adaptado para virtualización.} \\ \hline
\end{tabular}
\end{table}

Los cuadros~\ref{tab:rack-1} al~\ref{tab:rack-5} describen cinco servidores tipo rack, modelo System x3250 M4 de la marca IBM, los cuales  se ubican en el rack 1 del \CPD. Estos equipos, identificados con diferentes códigos dentro de la topología de red se ubica en los puestos 55, 54, 53 y 52. Presentan características técnicas homogéneas: procesador Intel Xeon E3\-1220v2, controlador SATA integrado, dos ranuras PCI Express, cuatro unidades SAS/SATA con capacidad de intercambio en caliente, fuente redundante de 460W, sistema de gestión integrado, cuatro puertos USB (dos frontales y dos posteriores) y lector de DVD.\@Su propósito principal es la prestación de servicios de cómputo y la provisión de recursos tecnológicos para estudiantes, además de la generación de máquinas virtuales orientadas a prácticas académicas.
% Rack 1
\begin{table}[H]
\centering
\scriptsize
\setlength{\tabcolsep}{2pt}
\renewcommand{\arraystretch}{1.0}
\caption{Ficha técnica --- Rack 1}\label{tab:rack-1}
\begin{tabular}{|p{0.5\textwidth}|p{0.2\textwidth}|}
\hline
\multicolumn{2}{|l|}{\textbf{DESCRIPCIÓN FÍSICA:} Servidor tipo rack} \\ \hline
\textbf{TIPO DE RECURSO:} Servidor & 
\multirow{5}{*}{\includegraphics[width=0.18\textwidth,keepaspectratio]{tablas-images/cp1/racks/rack-1.png}} \\ \cline{1-1}
\textbf{MODELO:} System x3250 M4 & \\ \cline{1-1}
\textbf{MARCA:} IBM & \\ \cline{1-1}
\textbf{CÓDIGO DE INVENTARIO:} 7 24390 50981 & \\ \cline{1-1}
\textbf{NUMERO EN CPD:} 55 & \\ \hline
\multicolumn{2}{|l|}{\textbf{ESPECIFICACIONES TÉCNICAS}} \\ \hline
\multicolumn{2}{|p{0.7\textwidth}|}{
- Procesador Intel Xeon E3-1220v2
- Controlador SATA integrado
- 2 ranuras PCI Express
- 4 SAS/SATA intercambio en caliente
- Fuente redundante 460W
- Gestión del sistema
- 4 puertos USB (2 frontal, 2 trasero)
- Lector DVD
} \\ \hline
\multicolumn{2}{|l|}{\textbf{PROPÓSITO:} Prestación de servicios de cómputo} \\ \hline
\multicolumn{2}{|p{0.7\textwidth}|}{\textbf{IMPACTO:} 
- Servicios a estudiantes
- Máquinas virtuales para prácticas} \\ \hline
\multicolumn{2}{|p{0.7\textwidth}|}{\textbf{OBSERVACIONES:} Ninguna} \\ \hline
\end{tabular}
\end{table}

% Rack 2
\begin{table}[H]
\centering
\scriptsize
\setlength{\tabcolsep}{2pt}
\renewcommand{\arraystretch}{1.0}
\caption{Ficha técnica --- Rack 2}
\label{tab:rack-2}
\begin{tabular}{|p{0.5\textwidth}|p{0.2\textwidth}|}
\hline
\multicolumn{2}{|l|}{\textbf{DESCRIPCIÓN FÍSICA:} Servidor tipo rack} \\ \hline
\textbf{TIPO DE RECURSO:} Servidor & 
\multirow{5}{*}{\includegraphics[width=0.18\textwidth,keepaspectratio]{tablas-images/cp1/racks/rack-1.png}} \\ \cline{1-1}
\textbf{MODELO:} System x3250 M4 & \\ \cline{1-1}
\textbf{MARCA:} IBM & \\ \cline{1-1}
\textbf{CÓDIGO DE INVENTARIO:} 7 24390 50980 & \\ \cline{1-1}
\textbf{NUMERO EN CPD:} 54 & \\ \hline
\multicolumn{2}{|l|}{\textbf{ESPECIFICACIONES TÉCNICAS}} \\ \hline
\multicolumn{2}{|p{0.7\textwidth}|}{
- Procesador Intel Xeon E3-1220v2
- Controlador SATA integrado
- 2 ranuras PCI Express
- 4 SAS/SATA intercambio en caliente
- Fuente redundante 460W
- Gestión del sistema
- 4 puertos USB (2 frontal, 2 trasero)
- Lector DVD
} \\ \hline
\multicolumn{2}{|l|}{\textbf{PROPÓSITO:} Prestación de servicios de cómputo} \\ \hline
\multicolumn{2}{|p{0.7\textwidth}|}{\textbf{IMPACTO:} 
- Servicios a estudiantes
- Máquinas virtuales para prácticas} \\ \hline
\multicolumn{2}{|p{0.7\textwidth}|}{\textbf{OBSERVACIONES:} Ninguna} \\ \hline
\end{tabular}
\end{table}

% Rack 3
\begin{table}[H]
\centering
\scriptsize
\setlength{\tabcolsep}{2pt}
\renewcommand{\arraystretch}{1.0}
\caption{Ficha técnica --- Rack 3}
\label{tab:rack-3}
\begin{tabular}{|p{0.5\textwidth}|p{0.2\textwidth}|}
\hline
\multicolumn{2}{|l|}{\textbf{DESCRIPCIÓN FÍSICA:} Servidor tipo rack} \\ \hline
\textbf{TIPO DE RECURSO:} Servidor & 
\multirow{5}{*}{\includegraphics[width=0.18\textwidth,keepaspectratio]{tablas-images/cp1/racks/rack-1.png}} \\ \cline{1-1}
\textbf{MODELO:} System x3250 M4 & \\ \cline{1-1}
\textbf{MARCA:} IBM & \\ \cline{1-1}
\textbf{CÓDIGO DE INVENTARIO:} 7 24390 50980 & \\ \cline{1-1}
\textbf{NUMERO EN CPD:} 53 & \\ \hline
\multicolumn{2}{|l|}{\textbf{ESPECIFICACIONES TÉCNICAS}} \\ \hline
\multicolumn{2}{|p{0.7\textwidth}|}{
- Procesador Intel Xeon E3-1220v2
- Controlador SATA integrado
- 2 ranuras PCI Express
- 4 SAS/SATA intercambio en caliente
- Fuente redundante 460W
- Gestión del sistema
- 4 puertos USB (2 frontal, 2 trasero)
- Lector DVD
} \\ \hline
\multicolumn{2}{|l|}{\textbf{PROPÓSITO:} Prestación de servicios de cómputo} \\ \hline
\multicolumn{2}{|p{0.7\textwidth}|}{\textbf{IMPACTO:} 
- Servicios a estudiantes
- Máquinas virtuales para prácticas} \\ \hline
\multicolumn{2}{|p{0.7\textwidth}|}{\textbf{OBSERVACIONES:} Ninguna} \\ \hline
\end{tabular}
\end{table}

% Rack 4
\begin{table}[H]
\centering
\scriptsize
\setlength{\tabcolsep}{2pt}
\renewcommand{\arraystretch}{1.0}
\caption{Ficha técnica --- Rack 4}
\label{tab:rack-4}
\begin{tabular}{|p{0.5\textwidth}|p{0.2\textwidth}|}
\hline
\multicolumn{2}{|l|}{\textbf{DESCRIPCIÓN FÍSICA:} Servidor tipo rack} \\ \hline
\textbf{TIPO DE RECURSO:} Servidor & 
\multirow{5}{*}{\includegraphics[width=0.18\textwidth,keepaspectratio]{tablas-images/cp1/racks/rack-1.png}} \\ \cline{1-1}
\textbf{MODELO:} System x3250 M4 & \\ \cline{1-1}
\textbf{MARCA:} IBM & \\ \cline{1-1}
\textbf{CÓDIGO DE INVENTARIO:} 7 24390 48735 & \\ \cline{1-1}
\textbf{NUMERO EN CPD:} 52 & \\ \hline
\multicolumn{2}{|l|}{\textbf{ESPECIFICACIONES TÉCNICAS}} \\ \hline
\multicolumn{2}{|p{0.7\textwidth}|}{
- Procesador Intel Xeon E3-1220v2
- Controlador SATA integrado
- 2 ranuras PCI Express
- 4 SAS/SATA intercambio en caliente
- Fuente redundante 460W
- Gestión del sistema
- 4 puertos USB (2 frontal, 2 trasero)
- Lector DVD
} \\ \hline
\multicolumn{2}{|l|}{\textbf{PROPÓSITO:} Prestación de servicios de cómputo} \\ \hline
\multicolumn{2}{|p{0.7\textwidth}|}{\textbf{IMPACTO:} 
- Servicios a estudiantes
- Máquinas virtuales para prácticas} \\ \hline
\multicolumn{2}{|p{0.7\textwidth}|}{\textbf{OBSERVACIONES:} Ninguna} \\ \hline
\end{tabular}
\end{table}

% Rack 5
\begin{table}[H]
\centering
\scriptsize
\setlength{\tabcolsep}{2pt}
\renewcommand{\arraystretch}{1.0}
\caption{Ficha técnica --- Rack 5}\label{tab:rack-5}
\begin{tabular}{|p{0.5\textwidth}|p{0.2\textwidth}|}
\hline
\multicolumn{2}{|l|}{\textbf{DESCRIPCIÓN FÍSICA:} Servidor tipo rack} \\ \hline
\textbf{TIPO DE RECURSO:} Servidor & 
\multirow{5}{*}{\includegraphics[width=0.18\textwidth,keepaspectratio]{tablas-images/cp1/racks/rack-1.png}} \\ \cline{1-1}
\textbf{MODELO:} System x3250 M4 & \\ \cline{1-1}
\textbf{MARCA:} IBM & \\ \cline{1-1}
\textbf{CÓDIGO DE INVENTARIO:} 51474 & \\ \cline{1-1}
\textbf{NUMERO EN CPD:} 52 & \\ \hline
\multicolumn{2}{|l|}{\textbf{ESPECIFICACIONES TÉCNICAS}} \\ \hline
\multicolumn{2}{|p{0.7\textwidth}|}{
- Procesador Intel Xeon E3-1220v2
- Controlador SATA integrado
- 2 ranuras PCI Express
- 4 SAS/SATA intercambio en caliente
- Fuente redundante 460W
- Gestión del sistema
- 4 puertos USB (2 frontal, 2 trasero)
- Lector DVD
} \\ \hline
\multicolumn{2}{|l|}{\textbf{PROPÓSITO:} Prestación de servicios de cómputo} \\ \hline
\multicolumn{2}{|p{0.7\textwidth}|}{\textbf{IMPACTO:} 
- Servicios a estudiantes
- Máquinas virtuales para prácticas} \\ \hline
\multicolumn{2}{|p{0.7\textwidth}|}{\textbf{OBSERVACIONES:} Ninguna} \\ \hline
\end{tabular}
\end{table}

El cuadro~\ref{tab:nas-1} presenta la ficha técnica del NAS, un sistema de almacenamiento en red modelo TS-832PX-4G de la marca QNAP, el cual se integra como un recurso estratégico en la infraestructura tecnológica. Este dispositivo cuenta con un procesador Annapurna Labs Alpine AL-324 de cuatro núcleos, memoria RAM de 4 GB DDR4 expandible hasta 16 GB, ocho bahías para discos SATA de 3.5 o 2.5 pulgadas, dos puertos de red RJ45 de 2.5 GbE y dos de 10 GbE, además de tres puertos USB 3.2 Gen 1 y ranuras PCIe para expansión. Su consumo energético es de 50.8 W en funcionamiento y 27 W en reposo, lo que permite operación continua. El propósito principal de este NAS es ofrecer almacenamiento compartido y redundante en red, ofreciendo disponibilidad de respaldos y acceso confiable a archivos para proyectos académicos.
% NAS 1
\begin{table}[H]
\centering
\scriptsize
\setlength{\tabcolsep}{2pt}
\renewcommand{\arraystretch}{1.0}
\caption{Ficha técnica --- NAS 1}\label{tab:nas-1}
\begin{tabular}{|p{0.5\textwidth}|p{0.2\textwidth}|}
\hline
\multicolumn{2}{|l|}{\textbf{DESCRIPCIÓN FÍSICA:} Sistema de almacenamiento en red} \\ \hline
\textbf{TIPO DE RECURSO:} NAS & 
\multirow{5}{*}{\includegraphics[width=0.18\textwidth,keepaspectratio]{tablas-images/cp1/NAS/nas-1.png}} \\ \cline{1-1}
\textbf{MODELO:} TS-832PX-4G & \\ \cline{1-1}
\textbf{MARCA:} QNAP & \\ \cline{1-1}
\textbf{CÓDIGO DE INVENTARIO:} Por definir & \\ \cline{1-1}
\textbf{FECHA DE ADQUISICIÓN:} & \\ \hline
\multicolumn{2}{|l|}{\textbf{ESPECIFICACIONES TÉCNICAS}} \\ \hline
\multicolumn{2}{|p{0.7\textwidth}|}{
- Procesador: Annapurna Labs Alpine AL-324, 4 núcleos
- RAM: 4 GB DDR4 (exp. a 16 GB)
- Bahías: 8 para discos SATA 3.5"/2.5"
- Puertos Red: 2 x RJ45 2.5GbE, 2 x 10GbE
- Puertos USB: 3 x USB 3.2 Gen 1
- Consumo: 50.8 W (func.), 27 W (reposo)
- Expansión: Ranuras PCIe
} \\ \hline
\multicolumn{2}{|p{0.7\textwidth}|}{\textbf{PROPÓSITO:} Almacenamiento compartido y redundante en red} \\ \hline
\multicolumn{2}{|p{0.7\textwidth}|}{\textbf{IMPACTO:} - Sin NAS: no hay backups ni acceso a archivos} \\ \hline
\multicolumn{2}{|p{0.7\textwidth}|}{\textbf{OBSERVACIONES:} Ninguna} \\ \hline
\end{tabular}
\end{table}

El cuadro~\ref{tab:firewall-1} corresponde a la ficha técnica de un firewall implementado en el \CPD, el cual funciona como sistema de seguridad de red. Se trata de un equipo DELL, modelo PowerEdge T100, con código de inventario 7 24390 46288 9 y chasis en formato torre. Sus especificaciones técnicas incluyen un procesador Intel Xeon E3110 de 3 GHz, 4 GB de memoria DDR2 (2 x 2 GB), un disco duro de 1 TB tipo HDD de 3.5 pulgadas, fuente de alimentación de 305 W, unidad óptica DVD-ROM y conectividad Ethernet. Su propósito principal es la seguridad de la red de servidores, evitando accesos no autorizados y protegiendo los recursos computacionales. Actúa como primera barrera de defensa en el ecosistema de virtualización y servicios del grupo, contribuyendo a la integridad de los datos y a la continuidad de los procesos misionales.
% Firewall 1
\begin{table}[H]
\centering
\scriptsize
\setlength{\tabcolsep}{2pt}
\renewcommand{\arraystretch}{1.0}
\caption{Ficha técnica --- Firewall}\label{tab:firewall-1}
\begin{tabular}{|p{0.5\textwidth}|p{0.2\textwidth}|}
\hline
\multicolumn{2}{|l|}{\textbf{DESCRIPCIÓN FÍSICA:} Sistema de seguridad de red} \\ \hline
\textbf{TIPO DE RECURSO:} Firewall & 
\multirow{5}{*}{\includegraphics[width=0.18\textwidth,keepaspectratio]{tablas-images/cp1/firewall/firewall.png}} \\ \cline{1-1}
\textbf{MODELO:} PowerEdge T100 & \\ \cline{1-1}
\textbf{MARCA:} DELL & \\ \cline{1-1}
\textbf{CÓDIGO DE INVENTARIO:} 7 24390 46288 9 & \\ \cline{1-1}
\textbf{NÚMERO EN CPF:} No especificado & \\ \hline
\multicolumn{2}{|l|}{\textbf{ESPECIFICACIONES TÉCNICAS}} \\ \hline
\multicolumn{2}{|p{0.7\textwidth}|}{
- Procesador: Intel Xeon E3110 3 GHz
- Memoria: 4 GB DDR2 (2 x 2 GB)
- Almacenamiento: 1 TB HDD 3.5"
- Fuente alimentación: 305 W
- Unidad óptica: DVD-ROM
- Tipo chasis: Torre
- Ethernet
} \\ \hline
\multicolumn{2}{|l|}{\textbf{PROPÓSITO:} Seguridad de la red de servidores} \\ \hline
\multicolumn{2}{|p{0.7\textwidth}|}{\textbf{IMPACTO:} - Protege equipos de usuarios no autorizados} \\ \hline
\multicolumn{2}{|p{0.7\textwidth}|}{\textbf{OBSERVACIONES:} Ninguna} \\ \hline
\end{tabular}
\end{table}

\section{Caracterización de servicios del GRID}
El \GRID\ ofrece una variedad de servicios tecnológicos a la comunidad académica, especialmente a los estudiantes de Ingeniería de Sistemas y Computación. Estos servicios incluyen:

\subsection{Servicios actuales}
Los servicios actuales del GRID se centran en la provisión de infraestructura de \TI\, incluyendo máquinas virtuales, almacenamiento y redes. Estos servicios son utilizados principalmente por estudiantes y docentes, los cuales se especifican en el cuadro~\ref{tab:servicios-actuales}.

\begin{table}[H]
	\centering
	\renewcommand{\arraystretch}{1.2}
	\setlength{\tabcolsep}{3pt}
	\tiny
	\begin{tabularx}{\textwidth}{|>{\raggedright\arraybackslash}p{0.25\textwidth}|X|}
		\hline
		\textbf{NOMBRE DEL SERVICIO}    & Máquinas Virtuales para estudiantes y docentes                                                                    \\
		\hline
		\textbf{TIPO DE SERVICIO}       & Servicio educativo                                                                                                \\
		\hline
		\textbf{PROPÓSITO}              & Proveer máquinas virtuales a profesores, estudiantes y administrativos para prácticas académicas mediante XCP-ng. \\
		\hline
		\textbf{HORARIO DISPONIBILIDAD} & 24/7                                                                                                              \\
		\hline
		\textbf{TIEMPO FUNCIONAMIENTO}  & 3 años                                                                                                            \\
		\hline
		\textbf{RECURSOS}               & Servidores torre y rack                                                                                           \\
		\hline
		\textbf{TECNOLOGÍAS}            & Hipervisor tipo I (XCP-ng)                                                                                        \\
		\hline
		\textbf{IMPACTO}                & Indisponibilidad afecta actividades misionales del grupo de investigación y programa de Ingeniería de sistemas.   \\
		\hline
	\end{tabularx}
	\caption{Caracterización de los servicios actuales del GRID}\label{tab:servicios-actuales}
\end{table}

\subsection{Servicios esperados}
Los servicios esperados por el GRID se orientan al aprovisionamiento de contenedores a traves de tecnologias VBC. Se espera que los usuarios de distintos dominios ( educación, investigación, extensión ) puedan beneficiarse de este nuevo servicio que se especifica en el cuadro ~\ref{tab:servicios-esperados}

\begin{table}[H]
	\centering
	\renewcommand{\arraystretch}{1.2}
	\setlength{\tabcolsep}{3pt}
	\tiny
	\begin{tabularx}{\textwidth}{|>{\raggedright\arraybackslash}p{0.25\textwidth}|X|}
		\hline
		\textbf{NOMBRE DEL SERVICIO}    & Ambientes computacionales basados en \VBC                                                                       \\
		\hline
		\textbf{TIPO DE SERVICIO}       & Servicio de educación, investigación y extensión                                                                \\
		\hline
		\textbf{PROPÓSITO}              & Proveer ambientes computacionales mediante tecnologías \VBC\ y mecanismo de orquestación                        \\
		\hline
		\textbf{HORARIO DISPONIBILIDAD} & 24/7                                                                                                            \\
		\hline
		\textbf{RECURSOS}               & Servidores torre y rack                                                                                         \\
		\hline
		\textbf{TECNOLOGÍAS}            & Por determinar                                                                                                  \\
		\hline
		\textbf{IMPACTO}                & Indisponibilidad afecta actividades misionales del grupo de investigación y programa de Ingeniería de sistemas. \\
		\hline
	\end{tabularx}
	\caption{Caracterización de los servicios esperados del GRID}\label{tab:servicios-esperados}
\end{table}

\section{Descripción de la oportunidad}

Actualmente el Grupo de Investigación en Redes, Información y Distribución (\GRID) presenta diversas necesidades y oportunidades con relación a los servicios tecnológicos que ofrece a la Universidad del Quindío, en apoyo a sus objetivos misionales de docencia, investigación y extensión.

En este contexto, el \GRID\ busca identificar tecnologías emergentes que permitan potenciar su capacidad de brindar servicios tecnológicos avanzados, tanto para su propio beneficio como para la comunidad académica dentro de su ámbito de influencia.

Con relación a lo anterior, las \textbf{tecnologías de virtualización basadas en procesos} se presentan como una alternativa para optimizar la gestión de recursos y servicios de tecnología informática (\TI). Aunque el \GRID\ cuenta con una infraestructura basada en máquinas virtuales, gestionadas mediante el hipervisor tipo I XCP-ng, además de iniciativas de computación \textit{Desktop Cloud}, aún requiere de instancias computacionales más livianas para ampliar su oferta de servicios computacionales dirigidos a la comunidad académica, especialmente a los estudiantes del programa de Ingeniería de Sistemas y Computación de la Universidad del Quindío.

Como mencionan \textit{Sepúlveda et al.} (2022), las tecnologías de virtualización han proliferado en los últimos años y constituyen la base subyacente de infraestructuras modernas como el \textit{cloud computing}. A partir de esta proliferación, las tecnologías de \VBC\ se presentan como una alternativa que podría potenciar la gestión de los recursos relacionados con la infraestructura de \TI\ del \GRID.

Las tecnologías de \VBC\ representan una opción de virtualización que requiere menos recursos computacionales para su operación \citep{Xavier2013}, y que, en conjunto con las máquinas virtuales ya existentes en el \GRID\, podrían constituir una oferta de servicios de \TI\ con mayor diversificación, escalabilidad, flexibilidad y mantenibilidad, para satisfacer los requerimientos del contexto académico del grupo de investigación.

\section{Resumen de la entrevista con el cliente}

Para comprender mejor las necesidades y expectativas del \GRID\, se realizó una entrevista con el cliente.

\begin{itemize}
	\item \textbf{Entrevistado:} Luis Eduardo Sepúlveda Rodríguez
	\item \textbf{Fecha:} 6 de febrero de 2025
	\item \textbf{Duración:} 25 minutos
	\item \textbf{Link:} \href{https://drive.google.com/file/d/1rIc9xOsyDqumlTV-QXcw0inPyIbSEHLz/view?usp=sharing}{click aquí}
	\item \textbf{Asistentes:} Anubis Haxard Correa Urbano, José Alejandro Arias Pinzón
\end{itemize}

\subsection{Misión del grupo \GRID}
En el minuto 1:01 se menciona que: el grupo de investigación no declara una misión y visión distinta a la de su organización, la Universidad del Quindío. En consecuencia, estos elementos se heredan directamente de la institución.

\subsection{Actividades del grupo de investigación}
En el minuto 1:10 se menciona que: Aunque su nombre podría llevar al sesgo de pensar que se dedica exclusivamente a la investigación, el \GRID se desarrolla en los tres pilares misionales: docencia, investigación y extensión. Participa en actividades académicas como la enseñanza en el programa de Ingeniería de Sistemas y Computación, desarrolla investigaciones mediante el método científico, y realiza actividades de proyección social y formación complementaria.

\subsection{La virtualización basada en contenedores como una oportunidad}
En el minuto 3:01 se menciona que: Las tecnologías de \VBC pueden aportar al cumplimiento de la misión institucional. Actualmente se utiliza Docker por ser un estándar de facto, no por una evaluación formal. Existen alternativas como Podman, ContainerD y LXC que también podrían apoyar los tres pilares institucionales.

\subsection{El problema de la multitud de herramientas}
En el minuto 3:44 se menciona que: Existen muchas herramientas que podrían cumplir los objetivos institucionales. Escoger una tecnología adecuada no es trivial y requiere comprender el contexto organizacional. Por eso, este proyecto busca ofrecer una solución arquitectónica basada en \VBC\, que sirva a estudiantes y docentes para comprender el estado y las tendencias de estas tecnologías.

\subsection{Difusión para apoyar a otros grupos e instituciones}
En el minuto 5:32 se menciona que: Aunque el proyecto se enmarca en el \GRID\, sus resultados podrían ser útiles para otras universidades, grupos de investigación o incluso la industria. Elegir tecnologías \VBC\@estratégicamente puede tener gran valor, por lo que se plantea la necesidad de difundir los avances y resultados del proyecto.

\subsection{Restricción en los recursos}
En el minuto 7:08 se menciona que: El \GRID\ cuenta con infraestructura de \TI\, pero debe considerar su contexto y limitaciones. Soluciones que requieran licencias costosas o hardware especializado no son viables. Por tanto, las opciones de código abierto cobran especial relevancia.

\subsection{Impacto del proyecto en los campos de estudio del \GRID}
En el minuto 14:50 se menciona que: Los pilares misionales abarcan muchas actividades. El \GRID se enfoca en áreas como desarrollo de software, pensamiento computacional, computación paralela, análisis de datos, inteligencia artificial, redes, infraestructura de \TI, y HPC.\@Este proyecto busca fortalecer esas áreas mediante el uso de tecnologías \VBC.\@

\subsection{Necesidad de orquestación entre máquinas virtuales y contenedores}
Fuera del audio se menciona que: Actualmente los servicios se ejecutan sobre máquinas virtuales con XCP-ng. Se considera deseable —aunque no obligatorio— que la solución propuesta permita integrar contenedores con máquinas virtuales completas mediante un clúster, para maximizar el aprovechamiento de la infraestructura existente.

\bigskip\noindent \textit{Nota:} Este documento es solo un resumen de la entrevista. El audio incluye una explicación adicional del mapeo SMS que no se encuentra transcrita aquí.
