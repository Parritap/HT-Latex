\ChapterImageStar[cap:validacion]{Implementación y validación de la solución}{./images/fondo.png}\label{cap:validacion}
\mbox{}\\
En este capítulo se detalla el proceso de validación de la solución propuesta, que incluye tanto pruebas técnicas como la aceptación por parte del cliente.
\section{Pruebas de validacion técnica}\label{sec:functional-tests}

%=========================functional-test/backup=========================
\subsection{functional-test/backup}
\noindent
La siguiente sección presenta los distintos manifiestos y scripts que conforman la estrategia de creación de respaldo en Kubernetes. Cada archivo cumple un rol específico dentro del despliegue, desde la definición del pod y sus volúmenes. A continuación, se describen brevemente los archivos incluidos en este paquete:
\subsubsection{deployment.yaml}
\noindent
\input{tablas-images/cp6/src/functional-test/backup/deployment.tex}
\subsubsection{pv.yaml}
\noindent
\input{tablas-images/cp6/src/functional-test/backup/pv.tex}
\subsubsection{service.yaml}
\noindent
\input{tablas-images/cp6/src/functional-test/backup/service.tex}
\subsubsection{writer-pod.yaml}
\noindent
\input{tablas-images/cp6/src/functional-test/backup/writer-pod.tex}
\subsubsection{run-tests.sh}
\noindent
\input{tablas-images/cp6/src/functional-test/backup/run-tests.tex}


%=========================functional-test/cluster-bootup=========================

\subsection{functional-test/cluster-bootup}
\noindent

\subsubsection{run-tests.sh}
\noindent
\input{tablas-images/cp6/src/functional-test/cluster-bootup/run-tests.tex}


%=========================functional-test/get-ip=========================

\subsection{functional-test/get-ip}
\noindent

\subsubsection{run-tests.sh}
\noindent
\input{tablas-images/cp6/src/functional-test/get-ip/run-tests.tex}



%=========================functional-test/aplicar-yml=========================/

\subsection{functional-test/upload-yaml}

\subsubsection{test-deployment.yaml}
\noindent
\input{tablas-images/cp6/src/functional-test/upload-yaml/test-deployment.tex}


\subsubsection{run-tests.sh}
\noindent
\input{tablas-images/cp6/src/functional-test/upload-yaml/run-tests.tex}


\section{Demostración de la validación técnica}
\noindent
Se han documentado y registrado las pruebas técnicas realizadas para validar la solución propuesta. Esta documentación son elementos audiovisuales que evidencian la correcta implementación y funcionamiento de los componentes desarrollados. Esta Demostración puede encontrarse en el siguiente enlace: \href{https://drive.google.com/drive/folders/1UgpmFmz7E2uYxv_FCGCpYArZDIMc0nxv?usp=sharing}{aquí}


\section{Aceptación de la solución por parte del cliente}
\noindent
En la reunión de

\section{Conclusiones de la validación}
