\ChapterImageStar[cap:caracterizacion-universos]{Caracterización de los Universos HTCondor}{./images/fondo.png}\label{cap:caracterizacion-universos}
\mbox{}\\

En esta sección se presenta una descripción detallada de los distintos universos que conforman HTCondor. Cada universo representa un entorno de ejecución específico, diseñado para satisfacer diferentes necesidades computacionales y facilitar la integración de diversas aplicaciones científicas y técnicas. Se analizan sus principales características, ventajas y limitaciones, proporcionando una visión integral que permita seleccionar el universo más adecuado según los requerimientos de cada proyecto.



\section{Universos disponibles}

Como se ha mencionado anteriormente, HTCondor es un sofisticado sistema de software diseñado para la \textbf{Computación de Alto Rendimiento} (\HTC) que gestiona vastos recursos computacionales al emparejar a los propietarios de recursos con los consumidores de recursos \citep{HTCondor}. Un concepto fundamental dentro de este software es el de los \textit{Universos}, los cuales definen el entorno de ejecución específico, las reglas de gestión de recursos y la semántica de fallos.

La selección del universo se define en el archivo de descripción de envío (también conocido como \textit{submit file}) y determina principalmente la forma en que el trabajo interactúa con el pool de recursos computacionales. Esta elección especifica aspectos como por ejemplo si la ejecución será local o remota, tal y como ocurre en el universo \textit{Grid}, o si se debe buscar algún binario específico para la ejecución, como en el caso del universo \textit{java}. Hasta la fecha de redacción de este documento, los principales universos soportados son los siguientes \citep{HTCondor-choosing-universe}:

\begin{itemize}
	\item \textbf{\textit{Vanilla}}
	\item \textbf{Grid}
	\item \textbf{Java}
	\item \textbf{\textit{Scheduler}}
	\item \textbf{Local}
	\item \textbf{Parallel}
	\item \textbf{VM}
	\item \textbf{Container}
	\item \textbf{Docker}
\end{itemize}

\subsection{\textit{Vanilla}}

El universo \textit{Vanilla} sirve como el entorno de ejecución \textbf{predeterminado} y más ampliamente utilizado en HTCondor. Está destinado específicamente a la gran mayoría de los programas de usuario, ofreciendo las menores restricciones y el más alto nivel de compatibilidad general. Si un administrador o usuario no especifica explícitamente un universo en el archivo \textit{submit}, se asume automáticamente que el universo escogido es el \textit{Vanilla} \citep{CERNBatchDocs}.



La flexibilidad inherente del universo \textit{Vanilla} significa que, para cargas de trabajo nuevas o especializadas, los administradores y desarrolladores frecuentemente priorizan adaptarlas para que se ejecuten dentro de este entorno. Por ejemplo, algunos trabajos requieren de envolver aplicaciones paralelas o pilas de dependencias complejas dentro de contenedores (\textit{Docker o Singularity}) y lanzarlas como un simple \textit{wrapper} ejecutable bajo el universo \textit{Vanilla}, estableciendo efectivamente el universo \textit{Vanilla} como la capa de compatibilidad universal para los flujos de trabajo modernos de alto rendimiento \citep{Emilio_DockerHTCondor, HTCondor_Parallel}.


\subsection{Local}

El universo \textit{Local} es aquel con el cual es posible ejecutar trabajos en la máquina \textit{submit}.

Los trabajos en el universo local presentan las siguientes características \citep{HTCondor-choosing-universe}:

\begin{itemize}
	\item Se ejecutan \textbf{inmediatamente} tras el envío
	\item No entran en el ciclo de negociación
	\item Omiten el proceso típico de emparejamiento
\end{itemize}


\subsection{\textit{Scheduler}}

Según la documentación oficial de \cite{HTCondor-choosing-universe}, el universo \textit{Scheduler} es esencialmente el mismo que el universo \textit{Local}, sin embargo, a diferencia de este último, el universo \textit{Scheduler} no utiliza un demonio \texttt{condor\_starter} para gestionar el trabajo y por lo tanto, ofrece funciones y soporte de políticas limitados.

\section{Universos con entornos de ejecución especializados}

Estos universos están diseñados para cargas de trabajo que requieren entornos de tiempo de ejecución de programación específicos o esquemas complejos de asignación de recursos que involucran múltiples máquinas concurrentes.

\subsection{Universo Java}

El universo Java está ajustado para trabajos escritos en este lenguaje, diseñados para ejecutarse en la Máquina Virtual de Java (JVM). Aunque es posible ejecutar aplicaciones Java en el universo \textit{Vanilla}, el universo Java incluye características especializadas que simplifican y facilitan su ejecución. Por ejemplo, como señalan \cite{Thain2002}, anteriormente los usuarios debían enviar el binario de la JVM bajo el universo \textit{standard} (un universo antiguo de HTCondor que ha sido reemplazado por el universo \textit{Vanilla}) para ejecutar programas Java. Actualmente, basta con enviar el archivo \texttt{.class} en el archivo \textit{submit}, permitiendo que el trabajo se ejecute directamente en los nodos ejecutores que cumplen con los requisitos necesarios para ejecutar un trabajo Java.

\subsection{Parallel}

El universo \textit{Parallel} está diseñado para dar soporte a trabajos que requieren ejecución síncrona en múltiples máquinas de ejecución dedicadas. Es esencial para aplicaciones estrechamente acopladas, más comúnmente aquellas que utilizan el paradigma \textbf{\textit{Message Passing Interface} (MPI)}, y reemplaza al universo \textbf{mpi} más antiguo \citep{HTCondor-env-services}.


Según la documentación oficial de \citep{HTCondor-env-services} en el universo \textit{Parallel}:

\begin{itemize}
	\item El envío del trabajo especifica el número total de \textit{slots} (o nodos ejecutores) requeridos usando \texttt{machine\_count}
	\item El demonio negociador de HTCondor asegura que todos los recursos necesarios se reclamen \textbf{concurrentemente} antes de que la ejecución pueda comenzar.
	\item HTCondor designa uno de los nodos como ``\textbf{nodo 0}'', que típicamente ejecuta el proceso maestro.
\end{itemize}

HTCondor gestiona la configuración de la comunicación, incluida la generación y copia de claves SSH a través de los nodos asignados para permitir la comunicación sin contraseña necesaria para iniciar la aplicación paralela. También requiere una configuración específica del comportamiento de terminación del trabajo a través de atributos como \texttt{+ParallelShutdownPolicy}~\citep{HTCondor-env-services}.




\section{Entornos de Alto Aislamiento: Contenedores y Máquinas Virtuales}

La necesidad de entornos de ejecución que encapsulen las dependencias de la aplicación, aseguren la reproducibilidad y proporcionen distintos niveles de aislamiento de seguridad ha llevado al desarrollo de universos dedicados para contenedores y virtualización.

\subsection{Docker}

El universo \textit{Docker} permite que un trabajo HTCondor se ejecute directamente dentro de un contenedor Docker. Este universo proporciona aislamiento de procesos a nivel de sistema operativo, utilizando la funcionalidad estándar de Docker para ejecutar imágenes definidas por el usuario.

\subsubsection{Requisitos e Integración}

La ejecución en el universo \textit{Docker} requiere:

\begin{itemize}
	\item El servicio Docker debe estar preinstalado y configurado en los nodos ejecutores.
	\item El envío del trabajo debe especificar el nombre de la imagen del contenedor usando \texttt{docker\_image}.
\end{itemize}


\subsection{Container}

El universo \textit{Container} ofrece un enfoque más flexible que el universo Docker, abstrayendo la tecnología de contenerización subyacente para dar soporte a sistemas como Singularity (ahora Apptainer). Esencialmente, lo que el universo \textit{Container} procura es ofrecer una sintaxis que no dé privilegio a un \textit{runtime} específico (como sí ocurre en el universo \textit{Docker}), de manera que cualquier máquina con un \textit{runtime} compatible con la imagen especificada por el usuario pueda ser usada \citep{HTCondor-env-services}.


\subsection{VM}

El universo \textit{VM} facilita la ejecución de una imagen de disco de máquina virtual como un trabajo HTCondor.
\subsubsection{Mecanismo e Integración con el Hipervisor}

En este universo:

\begin{itemize}
	\item El trabajo ya no es un simple ejecutable, sino una \textbf{imagen de disco completa}
	\item HTCondor soporta la integración con hipervisores como Xen, KVM y VMware \citep{HTCondor_vm_universe_wiki}
\end{itemize}

\section{Universos Distribuidos y Federados}

Estos universos abordan la necesidad de integrar y gestionar recursos computacionales que residen fuera de los límites inmediatos del clúster HTCondor, soportando la federación de recursos externos  \citep{Padmanabhan2011}.

\subsection{Grid}

El universo \textit{Grid} permite a los usuarios de HTCondor enviar trabajos a sistemas de gestión remotos, actuando como un puente hacia sistemas de lotes externos como Slurm, LSF u otros \textit{pools} de HTCondor a través del demonio \texttt{condor\_gridmanager}. La gestión de trabajos se realiza mediante protocolos de comunicación especializados \citep{HTCondor-Grid-universe}, lo que permite a las organizaciones utilizar recursos distribuidos a través de dominios administrativos dispares. Esta capacidad facilita colaboraciones científicas a gran escala y la federación de clústeres universitarios o nacionales \citep{Padmanabhan2011}.
