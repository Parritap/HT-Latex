\ChapterImageStar[cap:caracterizacion-universos]{Caracterización de los Universos HTCondor}{./images/fondo.png}\label{cap:caracterizacion-universos}
\mbox{}\\

En esta sección se presenta una descripción detallada de los distintos universos que conforman HTCondor. Cada universo representa un entorno de ejecución específico, diseñado para satisfacer diferentes necesidades computacionales y facilitar la integración de diversas aplicaciones científicas y técnicas. Se analizan sus principales características, ventajas y limitaciones, proporcionando una visión integral que permitirá seleccionar el universo más adecuado según los requerimientos de cada proyecto.



\section{Universos disponibles}

Como se ha mencionado anteriormente, HTCondor es un sofisticado sistema de software diseñado para la \textbf{Computación de Alto Rendimiento} (\HTC) que gestiona vastos recursos computacionales al emparejar a los propietarios de recursos con los consumidores de recursos \citep{HTCondor}. Un concepto fundamental dentro de este software es el de los \textit{Universos}, los cuales definen el entorno de ejecución específico, las reglas de gestión de recursos y la semántica de fallos.

La selección del universo se define en el archivo de descripción de envío (también conocido como \textit{submit file}) y determina principalmente la forma en que el trabajo interactúa con el pool de recursos computacionales. Esta elección especifica aspectos como por ejemplo si la ejecución será local o remota, como ocurre en el universo \textit{GRID}, o si se debe buscar algún binario específico para la ejecución, como en el caso del universo \textit{java}. Hasta la fecha de redacción de este documento, los principales universos soportados son los siguientes \citep{HTCondor-choosing-universe}:

\begin{itemize}
	\item \textbf{Vanilla}
	\item \textbf{Grid}
	\item \textbf{Java}
	\item \textbf{Scheduler}
	\item \textbf{Local}
	\item \textbf{Parallel}
	\item \textbf{VM}
	\item \textbf{Container}
	\item \textbf{Docker}
\end{itemize}

\subsection{Vanilla}

El universo Vanilla sirve como el entorno de ejecución \textbf{predeterminado} y más ampliamente utilizado en HTCondor. Está destinado específicamente a la gran mayoría de los programas de usuario, ofreciendo las menores restricciones y el más alto nivel de compatibilidad general. Si un administrador o usuario no especifica explícitamente un universo en el archivo submit, se asume automáticamente que el universo escogido es el  \texttt{universe = vanilla} \citep{CERNBatchDocs}.

\subsubsection{Mecanismo e Interacción con el Daemon}

Los trabajos enviados al universo Vanilla se ejecutan como procesos estándar gestionados por los demonios centrales de HTCondor. La ejecución implica la interacción entre:

\begin{itemize}
	\item El daemon \texttt{condor\_starter} en el nodo worker
	\item El daemon \texttt{condor\_shadow} en el nodo submit
\end{itemize}

Esta relación establecida facilita la gestión, el seguimiento del estado y la aplicación de políticas de recursos. La flexibilidad del universo Vanilla se extiende a la compatibilidad con diversos tipos de ejecutables, incluidos programas compilados y scripts de shell complejos.



La flexibilidad inherente del universo Vanilla significa que, para cargas de trabajo nuevas o especializadas, los administradores y desarrolladores frecuentemente priorizan adaptarlas para que se ejecuten dentro de este entorno. Por ejemplo, algunos trabajos requieren de envolver aplicaciones paralelas o pilas de dependencias complejas dentro de contenedores (Docker o Singularity) y lanzarlas como un simple \textit{wrapper} ejecutable bajo el universo Vanilla, estableciendo efectivamente el universo Vanilla como la capa de compatibilidad universal para los flujos de trabajo modernos de alto rendimiento \citep{Emilio_DockerHTCondor, HTCondor_Parallel}.


\subsection{Local}

El universo Local es aquel con el cual es posible ejecutar trabajos en la máquina submit.

Los trabajos en el universo local presentan las siguientes características \citep{HTCondor-choosing-universe}:
\\\\
\begin{itemize}
	\item Se ejecutan \textbf{inmediatamente} tras el envío
	\item No entran en el ciclo de negociación
	\item Omiten el proceso típico de emparejamiento
\end{itemize}


\subsection{Scheduler}

Según la documentación oficial de \cite{HTCondor-choosing-universe}, el universo Scheduler es esencialmente el mismo que el universo Local, sin embargo, a diferencia de este último, el universo scheduler no utiliza un demonio \texttt{condor\_starter} para gestionar el trabajo y, por lo tanto, ofrece funciones y soporte de políticas limitados.

\section{Universos con entornos de ejecución especializados}

Estos universos están diseñados para cargas de trabajo que requieren entornos de tiempo de ejecución de programación específicos o esquemas complejos de asignación de recursos que involucran múltiples máquinas concurrentes.

\subsection{Universo Java}

El universo Java está ajustado para trabajos escritos en este lenguaje, diseñados para ejecutarse en la Máquina Virtual de Java (JVM). Aunque es posible ejecutar aplicaciones Java en el universo Vanilla, el universo Java incluye características especializadas que simplifican y facilitan su ejecución. Por ejemplo, como señalan \cite{Thain2002}, anteriormente los usuarios debían enviar el binario de la JVM bajo el universo \textit{standard} (un universo antiguo de HTCondor que ha sido reemplazado por el universo Vanilla) para ejecutar programas Java. Actualmente, basta con enviar el archivo \texttt{.class} en el archivo submit, permitiendo que el trabajo se ejecute directamente en los workers que cumplen con los requisitos necesarios para ejecutar un trabajo Java.

\subsection{Parallel}

El universo Parallel está diseñado para soportar trabajos que requieren ejecución síncrona en múltiples máquinas de ejecución dedicadas. Es esencial para aplicaciones estrechamente acopladas, más comúnmente aquellas que utilizan el paradigma \textbf{Message Passing Interface (MPI)}, y reemplaza al universo \texttt{mpi} más antiguo \citep{HTCondor-env-services}.


según la documentación oficial de \citep{HTCondor-env-services} en el universo Parallel:

\begin{itemize}
	\item El envío del trabajo específica el número total de slots requeridos usando \texttt{machine\_count}
	\item El Negociador de HTCondor asegura que todos los recursos necesarios se reclamen \textbf{concurrentemente} antes de que la ejecución pueda comenzar
	\item HTCondor designa uno de los nodos como ``nodo 0'', que típicamente ejecuta el proceso maestro
\end{itemize}

HTCondor gestiona la configuración de la comunicación, incluida la generación y copia de claves SSH a través de los nodos asignados para permitir la comunicación sin contraseña necesaria para iniciar la aplicación paralela. También requiere una configuración específica del comportamiento de terminación del trabajo a través de atributos como \texttt{+ParallelShutdownPolicy}~\citep{HTCondor-env-services}.




\section{Entornos de Alto Aislamiento: Contenedores y Máquinas Virtuales}

La necesidad de entornos de ejecución que encapsulen las dependencias de la aplicación, aseguren la reproducibilidad y proporcionen distintos niveles de aislamiento de seguridad ha llevado al desarrollo de universos dedicados para contenedores y virtualización.

\subsection{Docker}

El universo Docker permite que un trabajo HTCondor se ejecute directamente dentro de un contenedor Docker. Este universo proporciona aislamiento de procesos a nivel de sistema operativo, utilizando la funcionalidad estándar de Docker para ejecutar imágenes definidas por el usuario.

\subsubsection{Requisitos e Integración}

La ejecución en el universo Docker requiere:

\begin{itemize}
	\item El servicio Docker debe estar preinstalado y configurado en los nodos worker
	\item El envío del trabajo debe especificar el nombre de la imagen del contenedor usando \texttt{docker\_image}
	\item Los trabajos se mapean fundamentalmente al marco de ejecución del universo Vanilla
\end{itemize}


\subsection{Container}

El universo Container ofrece un enfoque más flexible que el universo Docker, abstrayendo la tecnología de contenerización subyacente para soportar sistemas como Singularity (ahora Apptainer). Esencialmente, lo que el universo Container procura es proponer una sintaxis que no de privilegio a un \textit{runtime} en específico  (como sí pasa en el universo Docker), de manera que cualquier máquina con un runtime compatible con la imagen de especificada por el usuario pueda ser usada \citep{HTCondor-env-services}.


\subsection{VM}

El universo VM facilita la ejecución de una imagen de disco de máquina virtual como un trabajo HTCondor.
\subsubsection{Mecanismo e Integración con el Hipervisor}

En este universo:

\begin{itemize}
	\item El trabajo ya no es un simple ejecutable, sino una \textbf{imagen de disco completa}
	\item HTCondor soporta la integración con hipervisores como Xen, KVM y VMware \citep{HTCondor_vm_universe_wiki}
	\item La coordinación es gestionada por el Punto de Acceso de Grid de Máquina Virtual (VM-GAHP)
	\item El VM-GAHP maneja el ciclo de vida de la VM: inicialización, pausa, reanudación y terminación
\end{itemize}

Dado que la VM está completamente aislada, la transferencia de archivos debe gestionarse explícitamente utilizando los mecanismos estándar de transferencia de archivos de HTCondor, listando todos los archivos necesarios usando atributos como \texttt{transfer\_input\_files}.

\subsubsection{Espectro de Aislamiento y Casos de Uso}

El universo VM proporciona un aislamiento crítico a nivel de núcleo (kernel), necesario al ejecutar:

\begin{itemize}
	\item Sistemas operativos de terceros completamente no confiables
	\item Sistemas operativos altamente específicos (por ejemplo, trabajos de Windows dentro de un clúster Linux)
\end{itemize}

Esta virtualización completa contrasta con los universos de contenedores, que dependen del núcleo del host para los recursos compartidos. Mientras que los contenedores ofrecen un mejor rendimiento y una menor latencia de inicio debido a la reducción de la sobrecarga, el universo VM garantiza la máxima fidelidad de seguridad.

Más allá de la seguridad, el universo VM también se utiliza como un \textbf{mecanismo de aprovisionamiento de recursos}: la máquina virtual resultante puede ejecutar una tarea dedicada o configurarse para generar recursos adicionales que, a su vez, se añaden de nuevo al pool de HTCondor, creando efectivamente un pool virtual.

\section{Universos Distribuidos y Federados}

Estos universos abordan la necesidad de integrar y gestionar recursos computacionales que residen fuera de los límites inmediatos del clúster HTCondor, soportando la federación y la explosión (bursting) de recursos externos.

\subsection{Universo Grid: Gestión de Recursos Externos}

El universo Grid permite a los usuarios de HTCondor enviar trabajos a sistemas de gestión remotos---a menudo denominados ``recursos de grid''---mientras mantienen la interfaz familiar de envío y monitoreo de HTCondor.

\subsubsection{Meta-Programación y Manejo de Protocolos}

El universo Grid funciona como un \textbf{meta-scheduler}, enviando trabajos a sistemas de lotes externos como Slurm, LSF u otros pools de HTCondor a través del daemon \texttt{condor\_gridmanager}. La gestión de trabajos se basa en protocolos de comunicación especializados:

\begin{itemize}
	\item \textbf{GASS (General Access Service)}: Para transferir archivos esenciales (ejecutable, flujos de E/S) hacia y desde el sitio de ejecución remoto

	\item \textbf{GRAM (Grid Resource Allocation Manager)}: Para contactar al gatekeeper remoto para iniciar y monitorear el trabajo
\end{itemize}

Este universo proporciona una tolerancia a fallos esencial al monitorear el progreso del trabajo en el sistema remoto, manejando inteligentemente escenarios en los que el recurso remoto falla y manteniendo las credenciales de usuario que de otro modo podrían expirar durante un trabajo de larga duración.

El universo Grid permite a las organizaciones utilizar recursos agrupados a través de dominios administrativos dispares, facilitando colaboraciones científicas a gran escala y la federación de clústeres universitarios o nacionales.

\subsection{Contexto de Computación en la Nube: Adquisición Dinámica de Recursos}

Aunque no es un universo de trabajo en el mismo sentido arquitectónico que Vanilla o Parallel, HTCondor ofrece herramientas especializadas para aprovechar dinámicamente la infraestructura de nube comercial. Esta funcionalidad es crucial para extender los clústeres locales o manejar explosiones inmediatas de capacidad.

\subsubsection{HTCondor Annex y Explosión de Capacidad}

La herramienta \texttt{htcondor annex} permite a los administradores o usuarios:

\begin{itemize}
	\item Adquirir recursos en la nube (actualmente centrándose en Amazon Web Services EC2)
	\item Integrarlos dinámicamente en un pool HTCondor existente
	\item Representa una adaptación estratégica de gestionar colas de Grid compartidas a gestionar capacidad de nube dedicada
\end{itemize}

Los casos de uso principales incluyen:

\begin{itemize}
	\item Escenarios donde las demandas de recursos son intermitentes o inmediatas
	\item Cumplir plazos críticos
	\item Adquirir hardware especializado (GPU específicas, instancias con terabytes de memoria)
	\item Manejar escaseces repentinas de capacidad
\end{itemize}

Al aprovisionar recursos (a menudo instancias Spot de EC2 para eficiencia de costos) a través de la herramienta \texttt{annex}, las máquinas de la nube se añaden como slots de ejecución estándar de HTCondor, permitiéndoles aceptar trabajos del universo Vanilla o Container, utilizando efectivamente la nube como una extensión fluida y explosiva del entorno local.

\section{Síntesis y Soporte para la Decisión Arquitectónica}

El conjunto completo de universos HTCondor proporciona un marco altamente especializado y adaptable para la computación de alto rendimiento. El análisis revela varias dinámicas clave que dan forma al ecosistema moderno de HTCondor.

\subsection{Análisis Comparativo de Características Técnicas}

La siguiente tabla resume los mecanismos arquitectónicos centrales y las características operativas que definen cada universo HTCondor.

\begin{table}[h]
	\centering
	\small
	\caption{Comparación de Características Centrales de los Universos HTCondor}
	\begin{tabular}{@{}p{2cm}p{2.2cm}p{2.2cm}p{2.5cm}p{3cm}p{2cm}@{}}
		\toprule
		\textbf{Universo} & \textbf{Alcance de Ejecución} & \textbf{Soporte de Políticas} & \textbf{Interacción Daemon}  & \textbf{Característica Clave}                & \textbf{Estado}          \\
		\midrule
		Vanilla           & Remoto, Máquina Única         & Completo                      & Starter/Shadow/Startd        & Mayor compatibilidad, predeterminado         & Predeterminado Moderno   \\
		\addlinespace
		Standard          & Remoto, Máquina Única         & Completo, restrictivo         & Starter/Shadow/Startd        & Toma de puntos de control automática         & Legado/Obsoleto          \\
		\addlinespace
		Local             & Host de Envío                 & Rico                          & Starter bifurcado            & Ejecución inmediata, no preemptible          & Preferido Actual         \\
		\addlinespace
		Scheduler         & Host de Envío                 & Limitado                      & Generación directa           & Sobrecarga mínima, funcionalidades limitadas & Pronta Retirada          \\
		\addlinespace
		Parallel          & Remoto, Multimáquina          & Completo                      & Scheduler/Múltiples Starters & Coasignación multimáquina (MPI)              & HTC Especializado        \\
		\addlinespace
		Docker            & Remoto, Máquina Única         & Mapeo a Vanilla               & Starter vía Docker           & Aislamiento a nivel SO                       & Aislamiento Contenedores \\
	\end{tabular}
\end{table}
