\ChapterImageStar[cap:conclusiones]{Conclusiones}{./images/fondo.png}\label{cap:conclusiones}
\mbox{}\\

El desarrollo de esta investigación ha permitido alcanzar exitosamente el objetivo general de proponer universos HTCondor para la ampliación de la infraestructura del Grupo de Investigación en Redes, Información y Distribución (GRID) de la Universidad del Quindío. A través de un enfoque metodológico riguroso que combinó caracterización contextual, revisión sistemática de literatura, análisis de decisiones estructurado y implementación práctica, se logró no solo seleccionar e implementar los universos más adecuados, sino también generar conocimiento transferible para contextos académicos similares.

\section{Sobre la metodología empleada}
\noindent

La metodología desarrollada, que integró caracterización del GRID, estudio de mapeo sistemático (SMS) y análisis de decisiones y resolución (DAR) del modelo CMMI, demostró ser efectiva para la toma de decisiones tecnológicas en entornos académicos con restricciones de recursos. El SMS proporcionó una base empírica sólida que identificó 114 estudios relevantes sobre universos HTCondor, mientras que el DAR permitió evaluar objetivamente las alternativas disponibles mediante criterios técnicos y organizacionales claramente definidos.

La caracterización integral del GRID reveló la importancia de comprender no solo las capacidades técnicas actuales, sino también la estructura de stakeholders, necesidades específicas y objetivos estratégicos antes de emprender proyectos de ampliación tecnológica. Este enfoque holístico garantizó que las soluciones implementadas se alinearan con las realidades operacionales y expectativas institucionales.

\section{Sobre la selección e implementación de universos}
\noindent

El empate técnico entre los universos Grid y Parallel en la evaluación DAR condujo a la decisión estratégica de implementar ambos, maximizando el impacto del proyecto. Esta decisión resultó acertada, ya que ambos universos abordan necesidades computacionales complementarias: el universo Grid proporciona capacidades de federación y distribución de recursos geográficamente distribuidos, mientras que el universo Parallel habilita la ejecución de aplicaciones fuertemente acopladas que requieren sincronización y comunicación intensiva.

La implementación exitosa demostró que es posible expandir infraestructuras HTCondor existentes de manera gradual e incremental, aprovechando inversiones previas mientras se incorporan nuevas capacidades. La arquitectura federada del universo Grid permitió duplicar los recursos computacionales disponibles sin requerir modificaciones disruptivas en la infraestructura vanilla preexistente.

\section{Sobre el impacto institucional y académico}
\noindent

La ampliación de la infraestructura HTCondor del GRID trasciende el ámbito técnico y genera impactos significativos en múltiples dimensiones institucionales. En el ámbito de la docencia, la disponibilidad de recursos de computación distribuida proporciona a los estudiantes de Ingeniería de Sistemas y Computación acceso a tecnologías de vanguardia que fortalecen su formación práctica en áreas críticas como computación paralela, sistemas distribuidos y administración de infraestructuras complejas.

Para la investigación, las nuevas capacidades habilitan proyectos que requieren procesamiento intensivo, simulaciones complejas y análisis de grandes volúmenes de datos, posicionando al GRID como un facilitador de investigación interdisciplinaria tanto a nivel interno como en colaboraciones interinstitucionales. La interfaz Grid App democratiza el acceso a estos recursos, eliminando barreras técnicas que tradicionalmente limitaban su utilización.

\section{Sobre la escalabilidad y sostenibilidad}
\noindent

La arquitectura implementada está diseñada con escalabilidad como principio fundamental. La demostración exitosa de la expansión del universo Grid de un clúster único a dos clústeres independientes valida la capacidad del sistema para incorporar recursos adicionales de manera transparente. Esta característica es crucial para el crecimiento futuro de la infraestructura según evolucionen las necesidades computacionales del grupo.

La sostenibilidad del proyecto se garantiza mediante la documentación exhaustiva de configuraciones, procedimientos y arquitecturas desarrolladas. La generación de un marco metodológico reproducible permite que otras instituciones con contextos similares adapten y repliquen estas soluciones, contribuyendo al fortalecimiento del ecosistema de computación distribuida en el ámbito académico.

\section{Sobre las limitaciones identificadas}
\noindent

Durante el desarrollo del proyecto se identificaron limitaciones que proporcionan direcciones para trabajo futuro. La implementación actual se basa en una infraestructura específica (Raspberry Pi y máquinas virtuales) que, aunque adecuada para los objetivos planteados, podría beneficiarse de la incorporación de recursos más heterogéneos y distribuidos geográficamente.

La interfaz Grid App, aunque funcional, representa una versión inicial que podría expandirse para incluir capacidades avanzadas como monitoreo predictivo, optimización automática de recursos y integración con sistemas de autenticación institucionales. Estas limitaciones no comprometen la funcionalidad actual, sino que definen oportunidades de mejora continua.

\section{Reflexiones sobre la computación distribuida académica}
\noindent

Este proyecto evidencia la importancia estratégica de la computación distribuida en contextos académicos contemporáneos. La democratización del acceso a recursos computacionales de alto rendimiento no solo beneficia directamente a estudiantes e investigadores, sino que contribuye al posicionamiento institucional y fortalecimiento de capacidades científicas y tecnológicas.

La experiencia desarrollada confirma que las tecnologías HTCondor, originalmente diseñadas para centros de investigación de gran escala, pueden adaptarse exitosamente a contextos académicos con recursos limitados, proporcionando beneficios significativos sin requerir inversiones prohibitivas en infraestructura.

\section{Aportes al estado del arte}
\noindent

Esta investigación contribuye al estado del arte en múltiples dimensiones. Metodológicamente, proporciona un marco estructurado que combina caracterización contextual, revisión sistemática de literatura y análisis de decisiones para la selección e implementación de tecnologías de computación distribuida en contextos académicos.

Técnicamente, documenta la implementación exitosa de múltiples universos HTCondor en una infraestructura académica real, proporcionando configuraciones, arquitecturas y procedimientos validados que pueden ser replicados en instituciones similares. La validación práctica mediante casos de prueba reales demuestra la viabilidad y efectividad de las soluciones propuestas.

\section{Conclusión general}
\noindent

El proyecto ha cumplido exitosamente sus objetivos, resultando en una infraestructura HTCondor ampliada que incorpora los universos Grid y Parallel, respaldada por una interfaz de usuario intuitiva y documentación completa. Los resultados trascienden el contexto específico del GRID, proporcionando un modelo replicable para el fortalecimiento de capacidades de computación distribuida en instituciones académicas.

La metodología empleada, los resultados obtenidos y las lecciones aprendidas constituyen una contribución valiosa al conocimiento en el área de computación distribuida aplicada a contextos educativos y de investigación. El proyecto establece una base sólida para el crecimiento futuro de las capacidades computacionales del GRID y proporciona un marco de referencia para iniciativas similares en otras instituciones.

La evidencia generada confirma que la inversión en infraestructuras de computación distribuida en contextos académicos produce retornos significativos en términos de fortalecimiento de capacidades docentes, investigativas y de extensión, justificando plenamente los esfuerzos requeridos para su implementación y sostenimiento a largo plazo.
