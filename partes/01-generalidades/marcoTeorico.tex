\ChapterImageStar[cap:Marco-Teorico]{Marco Teórico}{./images/fondo.png}\label{cap:marcoTeorico}
\mbox{}\\
\noindent
En el ámbito de la gestión de proyectos y la computación de alto rendimiento, es importante contar con marcos de referencia que ofrezcan una estructura robusta y metodologías para enfrentar los retos actuales. Los conceptos a continuación proporcionan las bases metodológicas, técnicas y estructurales necesarias para el desarrollo e implementación de la ampliación de la infraestructura HTCondor del grupo \GRID.

\section{\PMBOK}
\noindent
En el ámbito de la gestión de proyectos y la computación de alto rendimiento, es importante contar con marcos de referencia que ofrezcan una estructura robusta y metodologías para enfrentar los retos actuales. Este proyecto se fundamenta en el \PMBOK~(\textit{Project Management Body of Knowledge}), que es un estándar reconocido a nivel mundial en la gestión de proyectos y proporciona un conjunto exhaustivo de buenas prácticas aplicables a la mayoría de los proyectos \citep{PMI2019}. Este marco ayuda a la gestión de aspectos importantes como el alcance, tiempo, costo, calidad, recursos humanos, riesgos y comunicación, entre otros.

\section{Normas ISO 9000}
\noindent
Complementando esta perspectiva, las normas ISO 9000 aportan un conjunto de principios y directrices internacionales que permiten a las organizaciones satisfacer las expectativas y necesidades de sus clientes de manera consistente. Estas normas no solo ofrecen un marco sistemático para la mejora continua de los procesos, sino que también propenden por la eficiencia operativa y la calidad de los productos y servicios ofrecidos. En el contexto de este proyecto, su adopción resulta pertinente con el fin de estructurar de forma adecuada las fases de diseño y validación, ayudando así a que el resultado del proyecto sea de calidad y aporte valor al cliente.

\section{Modelo por Capas}
\noindent
Otro pilar conceptual relevante es el modelo por capas, conocido también como arquitectura en capas, que se refiere a una estructura de diseño en la que el sistema se divide en diferentes niveles o capas, cada una con una función específica e independiente de las demás \citep{Spray2023}. Este enfoque resulta especialmente útil en el ámbito de la computación, ya que facilita la escalabilidad, el mantenimiento y la interoperabilidad de los sistemas involucrados. Además, contribuye a reducir la complejidad del diseño y a facilitar la implementación de trabajos futuros inspirados en el presente.

\section{TOGAF (\textit{The Open Group Architecture Framework})}
\noindent
Asimismo, el marco TOGAF, que se ha convertido en uno de los marcos de referencia más utilizados para el desarrollo y gestión de arquitecturas empresariales, proporciona un enfoque estructurado que permite a las organizaciones alinear la estrategia de negocio con los procesos de \TI, facilitando la toma de decisiones y reduciendo el uso de recursos \citep{Mumtaza2025}. Su arquitectura por fases, que abarca la planificación, diseño, implementación y monitoreo, no solo facilita la integración de sistemas y tecnologías, sino que también permite flexibilidad y adaptabilidad en un entorno empresarial en constante evolución.

\section{Estándar ISO/IEC 25010}
\noindent
Además de lo anterior, el estándar ISO/IEC 25010 se incorpora como un estándar internacional que establece un modelo de calidad del software y sistemas, ampliamente utilizado para evaluar la calidad de productos y servicios de \TI. Según lo explica \cite{ISO25010}, este modelo define características como funcionalidad, eficiencia, seguridad, mantenibilidad y usabilidad, proporcionando un marco integral para asegurar que los sistemas cumplan con los requisitos del usuario y del negocio. La implementación de ISO/IEC 25010 en este proyecto responde al deseo de establecer criterios objetivos para validar la utilidad del Universo HTCondor propuesto.


\section{Análisis de Decisiones y Resolución} %!TODO
LOREM IPSUM



\section{Conclusión del Marco de Referencia}
\noindent
En general, el marco de referencia que se propone para este proyecto se basa en teorías y, principalmente, estándares reconocidos internacionalmente, ofreciendo así un soporte metodológico, técnico y estratégico robusto para el diseño y desarrollo de la ampliación de la infraestructura HTCondor del grupo \GRID, permitiéndole a estos avanzar hacia una infraestructura más madura y escalable, alineada con las necesidades actuales de la computación distribuida y de alto rendimiento.
